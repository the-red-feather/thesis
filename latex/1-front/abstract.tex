\chapter*{Abstract}

In the field of \ac{GIS}, geodata transformation and analysis tools often take the shape of software libraries written in system level programming languages. 
% A number of these libraries, like PROJ, GDAL, and CGAL, play a foundational role in almost all \ac{GIS} software applications.
% However, when these libraries are referenced in to other languages and used in applications, 
% functionality often get 'lost in translation'.
However, the societal impact of these tools is often limited, as most end users only access these libraries via indirect means: Through bindings in other languages, through plugin in applications, or both. 
Additionally, the tools end-users end up with are often not composable, and may contain other hurdles like installation or configuration. 
% Performance and features get 'lost in translation' when using bindings for a different programming language, when using a plugin within an application, or worse, both.  
% Additionally, the application end-users are left with are often not composable with other applications, leading to labor intensive workflows.  
% At the same time, maintainers of certain \ac{GIS} libraries often find themselves in the situation of having to maintain and synchronize a great number of bindings and plugins, which limits innovation.

The goal of this study is to make core \ac{GIS} libraries more directly available and composable to end-users.
% This study presents and prototypes a novel method which allows \ac{GIS} practitioners without a background in software development, to access the full potential of core transformation and analysis capabilities found in native \ac{GIS} libraries.
% That is based on visual programming, and static web applications. 
% The study attempts to meet this goal by designing and implementing a 
% novel method based on the fields of visual programming, and static web applications. 
% The prior works on static geo-web applications and \ac{GIS} \ac{VPL}s indicate that a strong theoretical framework is in place.
% But, and this is especially evident in the prior studies regarding Browser-based geocomputation, the practical implementation of these theories were only partially successful, and limited in scope. 
% This necessitates a practical approach in response. 
This study presents and prototypes a novel method, centered around a visual programming language to host the functionalities of \ac{GIS} libraries from within an application, and in a composable manner. 
Additionally, the visual language is used to connect these libraries to a user-definable \ac{GUI}. 
This prototype is implemented as a static web application, so that these libraries are directly accessible to end users without installation.
% Geodata used within the application is also exclusively statically hosted geodata, or user-submitted data. 
\ac{GIS} libraries are compiled to WebAssembly, making the library usable in any language, including this web based visual language by using a 'no boilerplate' plugin system. 
% WebAssembly is also used to run  resorting to active backend web services. 
Finally, both scalability to handle sizable datasets, and a rich \ac{GUI} (3D viewers, file inputs, sliders), are primary design considerations and assessment criteria.
% be used without
% Using this, ...
% indeed 
 
The results show that this specific web-based VPL appears to be a feasible method for providing direct access to some native \ac{GIS} libraries, and does offer a unique set of features not found in comparable visual languages. 
The significance of this method, compared to other web-based geometry VPLs, lies in the fact that it offers a lenient plugin system, in combination with a range of different \ac{GUI} nodes, certain "dataflow VPL" properties, and a proposed zero-cost abstraction runtime. 
All of these features combined lead to a VPL which is able to directly connect \ac{GUI} components with native \ac{GIS} libraries, all while remaining scalable in principle.

On a practical level, more work remains to proof this feasibility.
The methodology developed by this study is only \emph{theoretically} accessible and composable, based on achieved features. 
User-testing is required to confirm if this method indeed improves workflows, and actually saves time and energy of developers and end users. 
Moreover, the prototypical software implementation used is limited and not production ready.
Both the fact that the 'no-boilerplate' plugin system cannot be used with C / C++ \ac{GIS} libraries, and that the zero-cost abstraction runtime is not functional, must be improved upon in future work. Despite this, visual programing, distribution using WebAssembly, and Rust-based geocomputation, all proved to be valuable directions of future \ac{GIS} research.

% This method is thereafter used to load various \ac{GIS} libraries, and used in demo applications, after which an assessment can be made on its quality and the extend of its achieved functionalities. 
% The study concludes by addressing if the method meets the overarching goal.


% Geocomputation is a cornerstone of \ac{GIS} \& modern mapping needs.
% While the last decades have seen major advancements, writing a geocomputational pipeline remains a complex, non-trivial exercise. 
% Performing geocomputations by utilizing a visual programming language (VPL) within a web browser is a novel development which seeks to simplify this process, allowing more people to write these types of pipelines more successfully.
% It would, in theory, allow end-users to author geocomputation pipelines, without the need of a software development background, and without the need to install software.
% utilizing a dataflow model.
% This model is the standard within VPLs dealing with geometry processing, and contains many advantages in terms of performance and reliability.

% In this context, the portability of existing geocomputation libraries is a common problem.
% This is often addressed by maintaining duplicate JavaScript alternatives to libraries such as GDAL, PROJ, and GEOS, complicating innovation.  

% This study seeks to improve the state of browser-based geocomputation VPLs by attempting to bring industry-standard geo-libraries to these environments. 
% The study poses that this can only be done if native geocomputation libraries can be \emph{compiled} to WebAssembly, \emph{loaded} into a VPL, and \emph{utilized} in a browser-based dataflow-VPL format.
% Discovering if and how these steps can be performed is the central question of this study. 
% % This study seeks to improve the state of web-based geocomputation VPLs 
% % by discovering if the library portability problem can be overcome, and if so, how.

% This question is answered by testing a possible solution.
% A proof of concept web-based VPL is designed, implemented and evaluated, which makes use of a novel plugin system, as well as a directed, acyclic graph-based data model \& interface.

% Using this proof of concept, the study was able to demonstrate that the web platform was sufficiently capable of representing a dataflow VPL capable of constructing geocomputation pipelines.
% The functional programming-properties of this dataflow VPL also makes geo-libraries sufficiently \emph{usable}, albeit with some well-known caveats of dataflow-VPLs, like the representation of conditionals and iteration.  

% The current methods of \emph{compiling} existing C++ geocomputation libraries to the web turned out to be insufficient for the purposes of this study.  
% This is due to the Emscripten compiler's focus on compiling full C++ applications instead of separate libraries. 
% Despite this, the study wás able to demonstrate how a novel method can be used to sufficiently \emph{compile} and \emph{load} multiple Rust-libraries for usage in the VPL, thanks to more feature-rich WebAssembly tools in the Rust ecosystem. 
% While Rusts geocomputation libraries are young, the study presents this method to either offer emscripten contributors a blueprint of a desired workflow, or to offer geocomputation library contributors a powerful use-case for the Rust language. 

% All in all, this means that either if the geocomputation libraries found in the Rust ecosystem mature, or if Emscripten's capabilities improve, then the code portability problem \& dataflow problem of existing web-based geocomputation VPLS can be solved. 


% \begin{note}

%     "make a simple narrative of what you did and what are the results."
%     "make it not be a very rough draft"
%     "6.2: deliver the things you promised."
%     "make it more formal: replace: "crucial", "by far" or "impressive""

%     Stelios Draft Comments
%     ======================
    
%     In general, I think it's going well. I think the intro is fine, although it can be improved. I like the conclusions quite a lot, tbh. You are giving proper and interesting answers to the raised questions. The rest is a bit of a very rough draft, as I see. So, some specific points:
    
%     TODO: Tone down stuff
%     -  It seems like you have placed several bits here and there that seem like either introduction or conclusions. I feel your enthusiasm and excitement in your writing, but I think you are bringing questions and answers in places were people would normally just expect a simple narrative of what you did and what are the results.
    
%     TODO: Write 6.2
%     -  Speaking of results, we are missing the experiments so please put this at your first priority. This is because it feels like you promise things that you do not deliver.
    
%     TODO: Find & Replace
%     -  In general, you are using some very bold statements and informal expressions. The latter is kinda okay, but the first one is very dangerous. I've noticed words like "crucial", "by far" or "impressive" which seem very biased and subjective, so better be avoided. You can still use them here and there (mostly in the conclusions), but I think you use them way  too much (again, I can see your enthusiasm).
    
%     TODO: Find & Replace 
%     -  Minor thing, but I think you have confused \citet with \citep. The first one is used when the citation is part of the sentence and the latter when it's at the end of a sentence (or in a parenthesis, anyway).
    
% \end{note}

% This might allow web based VPLs for geocomputation to be 
% Geodata computation is important.
% Geodata computation is difficult.
% geo-web-vpls could help, but have seen little research
% This study: design, implement, analyze a new prototype geo-web-vpl. 
% Design-> utilize existing, native libraries written in C++ \& Rust on the web, in the format of a VPL.

% results -> it works.

% -> study shows that interplay between textual and visual programming is possible


% \begin{note}
%  - Performance intensive: (Big data, O(n^2) problems) 
%  - Heterogenous data (type, quality, scale, criteria, crs) 
%  - Complex (geometric) operations (linalg, bundle adjustment, procedural modelling) 
% \end{note}

% All of this makes the process of geocomputation difficult. 


% The full flowchart runs client-side in a browser, and both end results and intermediate products can be inspected in a 3D viewer.

% GeoFront offers several functionalities such as the parametric creation of 2D and 3D primitives, triangulation, isocurve extraction, and more. 
% These functionalities can be expanded upon though a plugin system which utilizes the existing "Node Package Manager" infrastructure.
% Together with WebAssembly, this enables the utilization of industry standard geoprocessing libraries such as `CGAL`, `GDAL` and `PROJ`, and data parsing libraries such as `IFC.js` and `laz-rs`.



% Following the implementation, the project was tested by simulating use-case scenarios. 
% The tests demonstrate the feasibility of [...]
% At the same time some key parameters of [...] identified which if tuned properly they can optimize the performance, behavior and robustness of the geo-web-vpl.
% With the project being a prototype solution, the VPL is far from operational and there is certainly a lot of space for improvement regarding both components. 

% 1. Tryout (ACTUAL)
%    - A-la wapm WebAssembly Package Manager allows packages to be run from within the package-page itself. 
%   - Just meant to quickly try out some features.

% 2. Educational (ACTUAL)
%    - interactive educational tool
%    - (What does a delaunay triangulation look like? how does it behave? What happens if you lower the radius of inverse distance weighting ? )

% 3. Rapid-Prototyping (POSSIBLE)
%    - Setting up pipelines which can be consumed by cloud-based geoprocessing services. 
%    - Future work: export flowchart to a process which can be run natively or server side.

% 4. Publishing (POSSIBLE)
%    - Geotiff.io / ModelLab
%    - Web FME 
%    - Publish full web apps in and off themselves, making use of zero, one or multiple wasm-compiled libraries.  
%    - Future work: export to web-app (without flowchart)

% % Safesoft's FME, but web based \& open source 

% % CONCLUSION
% By creating geofront, this thesis was able to discover .............

% - The web is able to facilitate a visual proregramming language.

% - The web is able to be used for geoprocessing, albeit with some caveats
%   - TypedArrays,
%   - Geometric predicates 
%   - Rounding
%   - ETC.

% - Many of these things can be fixed with webassembly, but webassembly itself has other shortcomings
%   - Differences between Rust \& C++

% - reasonable performance 
%   (- great considering the platform)

% - would not be possible without these modern web features
%   - Web Assembly 
%   - Typed Array's 
%   - Web Workers
%   - Web Components,
%   - 2D Canvas API
%   - Web GL

% \todo{[JF]: I need to add more critical notes on promises of accessibility. Is a webapp truly accessible? Is a flowchart interactive, or does it hinder interactiveness? }

% I believe that such a web application can make geoprocessing more accessible to practitioners.
% This empowers users to create small geoprocessing demo's, and share these 
% With geofront, geoprocessing libraries can be loaded and used interactively. Users are also able to create and share flowcharts.
%%%%%%%%%%%%%%%%%%%%%%%%%%%%%%%%%%%%%%%%%%%%%%%%%%%%%%%%%%%%%%%%%%%%%%

% 


%%%%%
