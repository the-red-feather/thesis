
\chapter*{Abstract}
Map renderers play a crucial role in Web, desktop, and mobile applications. In this context, code portability is a common problem, often addressed by maintaining multiple code bases: one for theWeb, usually written in JavaScript, and one for desktop and mobile, often written in C/C++. The maintenance of several code bases slows down innovation and makes evolution time-consuming. In this paper, we review existing open-source map renderers, examine how they address this problem, and identify the downsides of the current strategies. With a proof of concept, we demonstrate that Rust, WebAssembly, and WebGPU are now sufficiently mature to address this problem. Our new open-source map renderer written in Rust runs on all platforms and showcases good performance. Finally, we explain the challenges and limitations encountered while implementing a modern map renderer with these technologies
-

% Geodata computation is important.

% Geodata computation is difficult.

% geo-web-vpls could help, but have seen little research

% This study: design, implement, analyze a new prototype geo-web-vpl. 

% Design-> utilize existing, native libraries written in C++ \& Rust on the web, in the format of a vpl.

% results -> it works.

% -> study shows that interplay between textual and visual programming is possible


% \begin{note}
%  - Performance intensive: (Big data, O(n^2) problems) 
%  - Heterogenous data (type, quality, scale, criteria, crs) 
%  - Complex (geometric) operations (linalg, bundle adjustment, procedural modelling) 
% \end{note}

% All of this makes the process of geocomputation difficult. 

% two ways of improving the process are actively being researched: Geocomputation by visual programming, and Geocomputation in web browsers.
% This study seeks to combine the advantages of both these approaches by 



% This thesis presents GeoFront, a web-based visual programming language, similar to Grasshopper \& FME. 
% With GeoFront, geoprocessing \& computational design flowcharts can be viewed, run, and shared using the web.  
% The full flowchart runs client-side in a browser, and both end results and intermediate products can be inspected in a 3D viewer.

% GeoFront offers several functionalities such as the parametric creation of 2D and 3D primitives, triangulation, isocurve extraction, and more. 
% These functionalities can be expanded upon though a plugin system which utilizes the existing "Node Package Manager" infrastructure.
% Together with WebAssembly, this enables the utilization of industry standard geoprocessing libraries such as `CGAL`, `GDAL` and `PROJ`, and data parsing libraries such as `IFC.js` and `laz-rs`.

% Geofront has been created as an experiment to explore if visual, 
% browser-based geo-computation can make geo-computation in general more accessible. 
% The advantages and disadvantages of browser based geo-computation, compared to native or server-side geo-computation, are examined in several scenario's. 
% Both quantitative indicators, like loading and runtime performance, as well as qualitative indicators, like the fitness for an intended use-case, are measured in each of these cases.
% This study concludes that based on these measurements, browser-based geo-computation is not only possible, but fast, and an enabler of many promising use-cases, such as on-demand geodata processing apps, educational demo apps, and code sharing. However, extensive user-group testing is required before any definitive statements on accessibility and fitness for geo-computation can be made.  

% Konstantinos' abstract
% x are continuously becoming more popular among practitioners due to [...]
% However, there is no [...] to allow [...]

% Problem: Web GIS and GEO-VPL have not found each other yet. 
% A general geo-web-vpl does not exist yet.  

% This thesis attempts to address this gap by proposing and implementing a general Web-based visual programming language for geo-computation. 
% This 'geo-web-vpl' is designed to fulfill what [...]
% Those requirements were identified after [...]

% [what]

% Following the implementation, the project was tested by simulating use-case scenarios. 
% The tests demonstrate the feasibility of [...]
% At the same time some key parameters of [...] identified which if tuned properly they can optimize the performance, behavior and robustness of the geo-web-vpl.
% With the project being a prototype solution, the vpl is far from operational and there is certainly a lot of space for improvement regarding both components. 

% Utilizing the workflow in practice would be the ideal way for collecting
% useful feedback. 
% Besides that, This is already a useful environment for demo's, education, and small-scale geometry processing. 



% webassembly interface proposal also poses  



% This thesis introduces GeoFront, a first step towards a web based, open source, geodata processing and visualization environment in an era of cloud-native geodata. 
% In this thesis, the motivation, requirements, technical aspects, and achieved functionality of geofront are covered. 

% GeoFront has been developed for two reasons. 

% Firstly, OGC Cloud native geospatial types, and specifically the Cloud Optimized Point Cloud, will create a need for equally accessible means to access, analyze, edit, and visualize this data. 
% Secondly, 

% % FUNCTIONALITY

% \emph{Accessibility} is achieved by compiling the geoprocessing libraries to webassembly, after which the tools can be run in a browser, client-side. 
% This mitigates the need for installment; 
% % As soon as the tool is found on the web, it can be used. 
% \emph{Interactivity} is achieved by making geofront a visual programming environment. 
% Functions are represented by blocks, and these can be chained together to form an interactively configurable geoprocessing pipeline. 
% In between products can be inspected and debugged using a 3D viewer. 
% % Special blocks such as sliders can be used to quickly try different values and parameters, and to 'sweetspot' certain settings.
% % INPUTS: OGC STANDARDS & FILE SYSTEM | OUTPUTS: DOWNLOAD

% % RESULT
% This study has successfully made a limited set of CGAL operations usable like this, as well as various rust-based geoprocessing algorithms such as Startin.
% % The performance is ..., \todo{"Do performance comparrisons"}

% % USE CASE
% GeoFront has multiple actual and possible use cases. 

% 1. Tryout (ACTUAL)
%    - A-la wapm WebAssembly Package Manager allows packages to be run from within the package-page itself. 
%   - Just meant to quickly try out some features.

% 2. Educational (ACTUAL)
%    - interactive educational tool
%    - (What does a delaunay triangulation look like? how does it behave? What happens if you lower the radius of inverse distance weighting ? )

% 3. Rapid-Prototyping (POSSIBLE)
%    - Setting up pipelines which can be consumed by cloud-based geoprocessing services. 
%    - Future work: export flowchart to a process which can be run natively or server side.

% 4. Publishing (POSSIBLE)
%    - Geotiff.io / ModelLab
%    - Web FME 
%    - Publish full web apps in and off themselves, making use of zero, one or multiple wasm-compiled libraries.  
%    - Future work: export to web-app (without flowchart)

% % FEATURES

% % CONTENT 

% % CONCLUSION 



% % Safesoft's FME, but web based \& open source 

% % CONCLUSION
% By creating geofront, this thesis was able to discover .............

% - The web is able to facilitate a visual programming language.

% - The web is able to be used for geoprocessing, albeit with some caveats
%   - TypedArrays,
%   - Geometric predicates 
%   - Rounding
%   - ETC.

% - Many of these things can be fixed with webassembly, but webassembly itself has other shortcomings
%   - Differences between Rust \& C++

% - reasonable performance 
%   (- great considering the platform)

% - would not be possible without these modern web features
%   - Web Assembly 
%   - Typed Array's 
%   - Web Workers
%   - Web Components,
%   - 2D Canvas API
%   - Web GL

% \todo{[JF]: I need to add more critical notes on promises of accessibility. Is a webapp truly accessible? Is a flowchart interactive, or does it hinder interactiveness? }

% We believe that such a web application can make geoprocessing more accessible to practitioners.

%%%%%%%%%%%%%%%%%%%%%%%%%%%%%%%%%%%%%%%%%%%%%%%%%%%%%%%%%%%%%%%%%%%%%%


% We introduce the \"Geofront\" application, which achieves this accessibility in two ways: 
% Firstly, geoprocessing libraries, written in either Rust or C++, are compiled to WebAssembly: binaries which can be run in any modern browser, client-side. This mitigates the need for any installation, and allows the software to be directly used as soon as a website is fully loaded. WebAssembly offers a performance comparable to native usage. Geofront accepts these binaries as plugins.

% Secondly, Geofront itself is set up as a web-based visual programming environment, complete with a 3D viewer and WMS, WFS \& WMTS support. Using these tools, users can interactively run these geoprocessing libraries with different datasets and parameters. Using visual programming, the user can chain and alter geoprocessing steps, visualize in-between products, and save \& load these workflows.



% 
% This empowers users to create small geoprocessing demo's, and share these 

% With geofront, geoprocessing libraries can be loaded and used interactively. Users are also able to create and share flowcharts.


%%%%%
