\chapter{Introduction}

\subsection*{Geodata \& Geodata computation}

The field of geo-informatics concerns itself with the collection, processing, storage, and visualization of geodata. 
By doing do, we offer the world priceless information about the land we build on, the seas we traverse, the air we breath, and the climates we inhabit. 
This information is foundational for many applications, including environmental modelling, infrastructure, urban planning, navigation, the military, and agriculture.   

Central to this effort is the process of converting raw geodata into meaningful geo-information. 
Additionally, the effort often involves executing transformations or calculations on existing datasets. 
The term \ac{geocomputation}, or geodata processing, is used to represent all types of computations performed on geographical datasets. Anything from the calculation of the area of a region, to a \ac{crs} transformations, or converting a raster dataset into a vectorized dataset, can be regarded as geocomputation.
All applications of geo-informatics require some level of geocomputation, as the raw data gathered from surveyance seldom corresponds to the precise information we wish to discover about the earth.       
This makes geocomputation a cornerstone of the entire field of geo-informatics, and vital to all applications of geo-information.

However, geocomputation is no trivial exercise. The sizable nature of geodata, together with a great variety in geodata formats and quality make geocomputation both computationally intensive and difficult to operate. 
In an ideal world, the activity of geocomputation should be ergonomic: Operations must perform as fast as possible, and the effects of those operations must be both clear to the user, and reliable.
While researchers and developers have made considerable improvement over the last decades, the improvement of both geocomputation, and the activity of geocomputation, remains as relevant as ever before. 

In recent years, two promising developments regarding geocomputation are emerging:
Visual programming, and browser-based geocomputation. 
these two developments require further explanation, after which the subject and goal this thesis can be made clear.

\subsection*{Visual programming}

% \emph{What is a VPL and how does it benefit geocomputation?}

% 2) Establish the VPL, and how it is used within geomatics 
A \ac{vpl}, or visual programming environment, is a type of programming language represented in a graphical, non-textual manner.
A VPL often refers to both the language and the Integrated Development Environment (IDE) which presents this language.

The two big advantages visual programming has over regular programming, are \textbf{experimentation}, and \textbf{debugability}.
A Visual programming language allows calculations to be changed on the fly, often with immediate feedback. By also allowing calculation results to be transformed, inspected and visualized intuitively, the process of programming is often experienced as more user-friendly than regular programming (SOURCE).
[quote a source to exemplify this].
VPLs like these offers users a chance to interactively automate workflows \& processing pipelines, while requiring little to no programming knowledge. 
A VPL done right can make automation available to a very large audience.

This advantage of interactive, low-code automation is why the VPL continues to be a popular interface within the field of GIS, as well as in neighboring fields like BIM, CAD, Shader Programming and Procedural Geometry. 
All these fields benefit from the combination of both low-code automation and visual debugging.
% The field of geo-informatics also appreciates the 'dataflow modelling' aspects (TODO FIX THIS SENTENCE)

Within the field of geo informatics, \ac{vpl}s are not a new phenomenon. VPLs have been used for decades to specify geodata transformations and performing spatial analyses.  
SaveSoft's FME (SOURCE) is a good example of this. This Extract Load Transform (ETL) platform automates data integration, and is widely used by GIS experts.

% \emph{what makes a vpl a 'new' development?}

Despite this history, the \ac{vpl} is still considered an ongoing development within the field of geocomputation. 
Especially in recent years, a push towards a more 'granular' \ac{vpl}s can be recognized: 
\ac{vpl}s offering geocomputation on a smaller, more detailed level. 
As an example, McNeels's Grasshopper (SOURCE) is increasingly being used for spatial analyses of buildings and neighborhoods, like solar irradiation or heating demands (source). 
Meanwhile, The GeoFlow \ac{vpl} (source) was used to model the 3D envelope of a building based on a pointcloud, which was subsequently scaled up in the creation of the 3D BAG dataset (source).

These ways of using a \ac{geo-vpl} are still relatively novel, and contain many open ended questions. That geocomputation in certain scenarios benefits from the interface offered by a \ac{vpl} is probably true, indicated by the number of prior studies on the topic (source, source, source), and the popularity of \ac{vpl}s used for geodata and geometry computation. The \emph{extend} to which \ac{vpl}s benefit geocomputation, is still relatively unknown. 

% Famous examples of VPLs are:
% \begin{itemize}

%   \item Scratch (src), a educational programming language
%   \item 
% \end{itemize}

% ... (name something about Dataflow programming, a programming paradigm often used in conjunction with a vpl. ) ... 

% -> testing & reproducability.
% RANSAC -> many 'magic' parameter. They need to be discovered by 'play'
% Jonathan blow -> using interactive applications, an intrinsic understanding can be gained without explicit communication.
% Game Of Life -> impossibility of 'proving' behaviour systems. 

\subsection*{Browser-based geocomputation}

\ac{geocomputation} within a web browser, henceforth dubbed as \ac{bbg}, is slowly gaining traction during the last decade \cite{kulawiak_analysis_2019, panidi_hybrid_2015, hamilton_client-side_2014}. 
Interactive geospatial data manipulation and online geospatial data processing techniques have been described as "current highly valuable trends in evolution of the Web mapping and Web GIS" \cite{panidi_hybrid_2015}. 
The central idea is to add browser-based geocomputation to web-mapping applications, allowing users not only to view data, but to edit and analyze the geodata to their own custom needs.

Additionally, browser-based geocomputation, compared to native GUI or CLI geocomputation, allows geocomputation to be more \textbf{accessible} and \textbf{distributable}. 
Accessible, since geocomputation on the web requires no installment or configuration, 
and distributable, since the web is cross-platform by default, and poses many advantages for updating, sharing, and licensing applications. 
By performing these calculations in the browser rather than on a server, server resources can be spared, and customly computed geodata does not have to be resend to the user upon every computation request.

However, \ac{bbg} also has many open-ended questions and challenges. 
The big catch is that browsers \& javascript are not ideal hosts for geocomputation. 
Javascript is a slow and imprecise language compared to C++, has limited support regarding reading and writing files, and does not possess of a rich ecosystem of geocomputation libraries.  
A new browser feature called WebAssembly might fix some of these issues, but this has not seen substantial research due to it novelty. 
More on this in \refsec{sec:background-wasm}.

%%%%%%%%%%%%%%%%%%%%%%%%%%%%%%%%%%%%%%%%%%%%%%%%%%%%%%%%%%%%%%%%%%%%%%%%%%%%%%%
%%%%%%%%%%%%%%%%%%%%%%%%%%%%%%%%%%%%%%%%%%%%%%%%%%%%%%%%%%%%%%%%%%%%%%%%%%%%%%%
%%%%%%%%%%%%%%%%%%%%%%%%%%%%%%%%%%%%%%%%%%%%%%%%%%%%%%%%%%%%%%%%%%%%%%%%%%%%%%%

% Note: this are all fragments and motivations. I'm leaving them here, don't
% be bothered by them, keep things concise, but if we need extra baggage, we 
% can get it from here. 

%%%%%%%%%%%%%%%%%%%%%%%%%%%%%%%%%%%%%%%%%%%%%%%%%%%%%%%%%%%%%%%%%%%%%%%%%%%%%%%
%%%%%%%%%%%%%%%%%%%%%%%%%%%%%%%%%%%%%%%%%%%%%%%%%%%%%%%%%%%%%%%%%%%%%%%%%%%%%%%
%%%%%%%%%%%%%%%%%%%%%%%%%%%%%%%%%%%%%%%%%%%%%%%%%%%%%%%%%%%%%%%%%%%%%%%%%%%%%%%

% This design of geofront contains several challenges. Browser-based geocomputation (BBGC) might not be as performant as back-end / native geocomputation \cite{panidi_hybrid_2015, hamilton_client-side_2014}, even with the usage of WebAssembly \cite{jangda_not_2019}. 
% Additionally, since current visual programming environments all fall short on at least one of the above features, the visual programming environment will have to be created from scratch.

% The cloud-native geospatial movement represents a significant opportunity and challenge to these VPL interfaces. The \textbf{opportunity} lies in the fact that a VPL interface is highly suitable as a 'configurator' or 'IDE' for cloud-computation. The promise of interactive, low-code automation matches the desire of most cloud native geocomputation providers to support users of different backgrounds, both programmers and non-programmers, both full GIS experts as well as non-experts (SOURCE: Modellab, SOURCE: Chris Holmes). A good example of this is the ModelLab VPL, found in Raster Foundry (SOURCE). This is also evident in the fact that existing VPL's like FME and Grasshopper have added proprietary cloud-computation features like FME Cloud (SOURCE) and ShapeDiver (SOURCE), respectively.

% <!-- HOWEVERRRRRRRRR -->
% However, the major \textbf{challenge} is that as of right now and despite these developments, popular VPL's fall short on a number of priorities and features required for cloud-native geospatial computation. 
% These shortcomings include their closed and proprietary nature, their distance from regular programming features and conventions (git version control, continuous integration), and the non-containerized, one to one relationship between the IDE application \& the cloud hosting platform. All of this hinders their suitability as general, cloud-computation configurers. 

% Sub component: FAIR: 
% - FINDABLE:      Hard to find the right tools for the job
% - ACCESSIBLE:    Hard to access these tools (install, setup environment, look at what you are doing)
% - INTEROPERABLE: Hard to use two tools from different ecosystems (bindings, plugins, etc). 
% - REUSABLE:      Hard to re-use a specific scripts written for one use case in another use case

% For a geocomputation method to match the 'cloud native geospatial movement'

% Currently, WebAssembly is the only way one can take a geocomputation function from an arbitrary source, and use it in both the browser and on a server. 
% We wish to discover if this has any practical use cases. 

% - "As massive datasets become more and move available, the availability of the \emph{means} to process and analyse these data becomes more and more relevant". 

% Despite these advantages, open, accessible vpl's are often unavailable. 
% - Expensive, proprietary environments 
% - Require specific plugins in order to utilize open, industry standard geocomputation libraries
% - Not FAIR geocomputation

%%%%%%%%%%%%%%%%%%%%%%%%%%%%%%%%%%%%%%%%%%%%%%%%%%%%%%%%%%%%%%%%%%%%%%%%%%%%%%%

% This thesis explores if these in-between processing steps could also be performed within a web application.  

% This way, geodata processing applications could profit from the ease of accessibility and maintainability granted by the web as a platform.  

% <!-- More Why of web GIS and web-geocomputation: 
% - making the geoweb more feature-rich
% - allowing quick demonstration (wapm)
% - allowing easy access (overleaf)

% ## Motivation
% Under normal circumstances, Web applications within the field of geo-informatics are mostly used for the first and final stages of a common geodata process: Retrieval and visualization. 
% The processing stages in-between are almost always performed on the desktop using environments like QGIS or ArcGIS, or by writing and using CLI tools. 

% <!-- At the same time, a need arrises -->

%%%%%%%%%%%%%%%%%%%%%%%%%%%%%%%%%%%%%%%%%%%%%%%%%%%%%%%%%%%%%%%%%%%%%%%%%%%%%%%

% Under normal circumstances, Web applications within the field of geo-informatics are mostly used for the first and final stages of a common geodata process.  
% (
% If one wishes to retrieve geodata, web portals are used to discover the required datasets. After this, the OGC Web Services are often utilized to download and truly access this geodata. 

% This geodata is then processed locally, using QGIS, ArcGIS, command line tools, or any other 

% and at the end Tools like Leaflet and Celcium have been created to visualize the earth in both 2D and 3D , and tools like d3.js can produce interactive graphs to supplement these web pages. 

% There is, however, more to the web than just visualization. Due to 
% )
% This thesis asks the question if the web could do more than just visualization. 

% By creating the use-case application GeoFront, we ask the question if a modern web browser, and the current state of the client-side web technologies are capable of facilitating  

%%%%%%%%%%%%%%%%%%%%%%%%%%%%%%%%%%%%%%%%%%%%%%%%%%%%%%%%%%%%%%%%%%%%%%%%%%%%%%%

\section{Problem Statement}
Taken together, the proven usefulness of a vpl for geodata computation, together with the theorized potential of browser-based geocomputation, raise the question why a \ac{geo-web-vpl} does not exist yet.

A \ac{geo-web-vpl} could be the next step in the pursuit of making the activity of geocomputation more fast, clear and reliable.
The activity of geocomputation could benefit from the experimentation and debugability advantages of a \ac{vpl}, combined with the accessibility and distribution advantages of a web application. 
This format might even yield completely unique benefits.
% , by means of a "1 + 1 = 3" line of reasoning.
However, since no geo-web-vpl exist, there is no way of definitively knowing or testing these aspects. 

Prior studies give an indication to where the disconnect between the field of geo-VPLs and web-GIS may be. This is further explained in \refchap{chap:methodology}.
We encounter either existing geo-VPLs which were not able to be properly used in a browser, or web-based VPLs unable to support existing geocomputation functionalities. From this, it can be concluded that the support of existing geocomputation functionalities is a central aspect of making a geo-web-vpl succeed. 

%%%%%%%%%%%%%%%%%%%%%%%%%%%%%%%%%%%%%%%%%%%%%%%%%%%%%%%%%%%%%%%%%%%%%%%%%%%%%%%

\section{Research Objective}

\begin{note}
TODO: make both of these presuppositions clear
1. Combining a geo-vpl and bbg is useful. It is valuable to figure out if it can be done
2. Proper utilization of existing, native geocomputation libraries within a VP environment is key to making a geo-web-vpl succeed. 
\end{note}

This study seeks to close the gap between geo-VPLs and browser-based geocomputation by designing, implementing, and evaluating a \ac{geo-web-vpl}.
The study starts from the presupposition that proper utilization of existing, native geocomputation libraries within a VP environment is key to making a geo-web-vpl succeed. 
As such, overcoming this technical challenge is the focal point of this study. 

\section{Research Questions}
Based on this objective, the research question is formulated as follows: 
\begin{itemize}[ ]
  \item \myMainRQ
\end{itemize}

\subsection*{Supporting Questions}
The following supporting questions are defined to aid in answering the main question.
These are based upon various components of the main problem statements, further explained in \refchap{chap:methodology}
\begin{itemize}[-]
  \item \mySubRQOne
  \item \mySubRQTwo
  \item \mySubRQThree
  \item \mySubRQFour
\end{itemize}
In order to obtain answers to these questions, a literature review is performed,
a prototypical software application is developed, 
and this application is tested and compared to existing methods using real-world datasets.


% \subsection*{Use Case: Geofront}
% Geofront is the name of a prototype web application, developed as part of this study.
% Geofront is created to both contextualize this study, and to answer the question.

% lowering the delta between "it works for me" and "it works for someone else"


% WHY DOES SOMEONE WANT TO TAKE EXISTING LIBRARIES AND TURN THEM INTO A WEB-BASED VPL FORMAT? 
% - interactive, visual debugging 
%   - explaining the behavior of algorithms to yourself and others
%   - why web: collaborative debugging
% - reproducability of results
%   - lowering the delta between "it works for me" and "it works for someone else"
% - improving the 'online presence' of research.
%   - making research results 'usable' via a link
%   - interdisciplinary exchange of ideas


% SO: Geofront is intended as a Computational Geometry Sandbox environment.
% Meant for: 
%  - publication
%  - sharing ideas 
%  - trying things out
%  - debugging code
%  - learning

% NOT for making programming as a whole 'easier'


\newpage
\section{Scope}
The scope of this thesis is cornered in the following six ways: 

\subsection*{Only frontend geocomputation}
This study excludes any \emph {backend} based geocomputation.
A web application \textit{could} be used to orchestrate geocomputation web-services, which could also deliver a form of browser-based geocomputation to end-users. 
However, for the scope of this thesis, we limit ourselves to purely client-side solutions, with calculations literally happening within the clients browser. 
This is also why this study excludes the OGC standard of the \ac{wps} \cite{ogc_web_2015}.

Adding backend-based geocomputation to a geo-web-VPL would be an excellent follow-up investigation to this study. 
Future work could research the possibility of utilizing a hybrid strategy of both client-side and server-side geocomputation, following in the footsteps of \cite{panidi_hybrid_2015}. 

\subsection*{No Usability Analysis} % 
While accessibility / usability is a motivation of the development of a \ac{vpl}, no claims will be made that this method of geocomputation is more usable as opposed to existing geocomputation methods. This research attempts to solve practical inhibitions in order to discover whether or not browser-based, vpl-based geocomputation is \emph{a} viable option. If it turns out that this method is viable technically, future research will be needed to definitively proof \emph{how} usable it is compared to all other existing methods.  

% This paper seeks to first close this gap, limiting itself to overcoming the technical and design boundaries in the pursuit of practical client-side geocomputation.

Similarly, a survey analyzing how users experience browser-based geocomputation in comparison to native geocomputation must also be left to subsequent research. While this would gain us tremendous insight, client-side geocomputation is too new to make a balanced comparison. Native environments like QGIS, FME or ArcGIS simply have a twenty year lead in research and development. 

% It is my goal to introduce this as a new geocomputation option, and to name the advantages and disadvantages we can be sure of. An actual comparison of client-side vs native geocomputation is something different. 

\subsection*{Only WebAssembly-based containerization}
This thesis examines a WebAssembly-based approach to containerization and distribution of geocomputation functionalities. 
Containerization using Docker is also possible for server-side applications, but is not (easily) usable within a browser. 
For this reason, Docker-based containerization is left out of this studies' examination. 
And to clarify: Docker and WebAssembly are not mutually exclusive models, and could be used in conjunction on servers or native environments. 

\subsection*{Mostly Point Cloud/ DTM focussed geocomputation}
We are also required to concentrate on the scope of 'geocomputation', which is a sizable phenomenon.
The term is generally used to cover all operations on geodata, from rasters, tabular datasets, 'object-oriented datasets' such as the CityJSON or IndoorGML, and point clouds. 
Due to time limitations, we are forced to focus on particular type of geocomputation.
3D-based geocomputation is chosen, with a particular focus on pointclouds and and DTMs. 
The hypothesis is that these types of data may benefit the most from 'geo-web-vpl' based geocomputation.

\subsection*{Assumption: a '\ac{geo-web-vpl}' is a normal 3D VPL capable of common geo-computation processes}
For the scope of this thesis, we assume a \ac{geo-web-vpl} is very similar to any 3D VPl, like Blender Geometry Nodes, Houdini, GeoFlow, and Grasshopper. 
It differs only in the fact that it supports geodata types, and offers common geocomputation functionalities. 
In reality, a lot of differences and nuances exist between the field of geocomputation and the field of procedural modelling. 
However, due to the limited scope of this study, these concerns will only be picked up at the final discussion, found at \refsec{sec:discussion}.
 
\subsection*{Only core browser features}
Lastly, the implementation of the geo-web-vpl will limit itself to core browser features, keeping dependencies at a minimum, in an attempt to generalize the results of the study.
If the study would use very specific frameworks and technologies to solve key issues, questions might arise if the results of the study counts in general for browser-based geocomputation or geo-VPLs, or if they only count in this very specific scenario. 
This study defines "Core browser features" in \refsec{sec:methodone}

\newpage
\section{Reading Guide}

The remainder of this study is structured as follows:

(work in progress)

\begin{itemize}[ ]
  \item \refchap{chap:related}, Related Work, provides an overview of the theoretical background that is used in the rest of this study.
  
  \begin{note}
    EXPLAIN WHY -> The study starts from the presupposition that proper utilization of existing, native geocomputation libraries within a VP environment is key to making a geo-web-vpl succeed. 
  \end{note}

  \item \refchap{chap:methodology}, Methodology, explains precisely in what way the research-questions will be answered. In addition, the main design decisions are described and justified in this part of the study.

  \item \refchap{chap:implementation}, Implementation, presents the implementation of the methodology. Additionally, it Provides an answer for the \textit{\mySubRQOneTitle, \mySubRQTwoTitle , and \mySubRQThreeTitle} sub research questions. 
  \subitem Answer of "\mySubRQOneTitle": Show the full application implementation. Conclude with what about this implementation is relevant for geocomputation, and in what ways the browser is limiting, or beneficial. 
  \subitem Answer of "\mySubRQTwoTitle": Show all wasm-considerations: C++ vs Rust vs rewriting in Js. Emscripten vs wasm-bindgen. Conclude with the preference for rust, and explain why.
  \subitem Answer of "\mySubRQThreeTitle": The implementation details of the plugin system schema. 
  
  \item \refchap{chap:experiments}: Experiments, describes the results obtained after the implementation of the methodology. With these results, an answer to the fourth research question of \mySubRQFourTitle can be given. 

  \item \refchap{chap:conclusion}: Conclusion, 
  \subitem conclude to which extend the solution was able to satisfy the main research question. 
  \subitem discuss to which extend the solution was able to satisfy main research question.
  \subitem discuss unaddressed aspects of the thesis, such as the massive nature of geodata, and the utility of a vpl for geocomputation.
  \subitem future work

  \item \refchap{chap:postscript}: Post Scriptum, argues the wider implications of this study. 

\end{itemize}
