\chapter{Introduction}

% Establish OGC in two sentences, mentioning their name and Vision
The Open Geospatial Consortium (OGC)...
Mission: FAIR Geodata 

% Establish Cloud Native movement.
A prominent development within the OGC is the recent effort towards a \textbf{"Cloud Native Geospatial"} future. 
This initiative aims to radically simplify geodata storehouses to static servers serving large, singular binary geodata files. All processing and analysis of this geodata can then be performed by separate cloud-based web services. 
This architecture has many advantages over current geodata storage and analysis methods:
\begin{itemize}
  \item These new Cloud Native geodata formats are much cheap to access by front-end and back-end services, compared to active services.
  \item Substituting active SQL or noSQL databases by static binary files is easier and cheaper for data providers, leading to more and more readily available geodata.
  \item By using supercomputers (Microsoft Planetary Computer) and cloud-storage (AWS), Geodata processes could make use of near-infinite computational and storage resources. 
  \item By having all data centralized in one location or type of location, new, large scale patterns within our geodata could be discovered.  
  \item For web GIS, this would offer direct data streaming options, similar to services like "Netflix" or "Spotify".  
\end{itemize}

These features may have a far reaching impact on society. Chris Holmes, forerunner of the cloud-native geospatial movement, envisions what the movement could mean for even non-GIS users: 
\emph{
  With the introduction of accessible, centralized data, and the dramatically different workflows that follow, Cloud Native Geospatial has the potential to introduce new, non-specialized users to the power of geospatial information that GIS practitioners have enjoyed for decades. [...]. The ecosystem of geospatial experts will collaborate to create analyses and insight, but any non-expert user will be able to select and apply those to the geographic area they care about. \~ Chris Holmes
}
% This is also reflected by cloud-native based tools like (Google Earth Engine or RasterFoundry) may achieve such a feed, by being web based and stuff...
All these reasons explain why the OGC and many other parties are now actively pursuing this vision.

But while this vision is in active development, many large-scale challenges are still in its way. 
One of the most important challenges is the required paradigm shift within geo-computation / geoprocessing workflows. 
The current, common geo-computation workflow of retrieving online data, only to run it through a local process and send the resulting data back into servers, will have to be reversed: In a cloud-native future, we will not retrieve data for our local process, but we will upload our process to the data.  
This introduces a sizable challenge: \textbf{Portable, Containerized Geo-computation}.

% \textbf{and the algorithms powering the processing can be shared online and customized collaboratively}. -> Chris again

% \begin{itemize}
%   \item Up to this point, the world of GIS has done a considerable effort to make geodata more Findable, Accessible, Interoperable, and Reusable. The challenge of Portable geo-computation now forces us to extend the effort of FAIR geodata to FAIR geodata computation as well.  
%   \item If we want our geodata processes to be just as portable as the geodata it takes as input, then perhaps the FAIR paradigm should extend from FAIR geodata to FAIR geodata processing . FAIR geo-computation.
%   \item Furthermore, it remains a mystery how these containerized containerized processes will be configured and accessed by frontend computation environments. 
%   \item Holmes: one of the vital ingredients: \emph{"and the algorithms powering the processing can be shared online and customized collaboratively"}.
% \end{itemize}

The challenge of sharing and chaining together containerized fragments of geoprocesses to a variety of environments will require more than open source collaboration. 
This study interprets the challenge of portable geo-computation by means of the FAIR paradigm. 
If geodata processes need to be just as portable as the geodata forming the input and output, then perhaps the FAIR paradigm should extend from FAIR geodata to FAIR geodata processing.
The challenge facing the cloud-native vision then becomes: \textbf{How to make geo-computation Findable, Accessible, Interoperable, and Reusable?} 

% auxiliary advantage : geoprocessing more accessible. 

% state of the art regarding this issue, make a path towards the particular thesis, and why it is an application
The current state of the art is far removed from either portable or FAIR geo-computation. 
\todo{Improve this intro}
\begin{itemize}
  \item current methods: Docker, and some geo-computation platforms.
  \item Not many implementations using WebAssembly, while this is a prime candidate: Even the guy who made Docker said so. 
  \item ignore the cloud: focus on the act of containerizing geoprocesses using webassembly an sich
\end{itemize}

FAIR geoprocessing may have many auxiliary advantages. 

% \begin{itemize}
%   \item Findable? -> web browser / Package Managers
%   \item Accessible? -> web tools
%   \item Interoperable -> WebAssembly / containerized runtimes and processes, using common types
%   \item Reusable -> compile to regular web service / function
% \end{itemize}

% X : wasm-based geo-computation applications

\section{Research Objectives}

% This must be better 
Due to the need for containerized geo-computation, The goal of this thesis is to develop both a new method and application for geo-computation in a browser-based front-end using WebAssembly.

This thesis presents the design, creation and evaluation of GeoFront, a web-based, visual geo-computation tool. 
With GeoFront, geoprocessing flowcharts can be created, shared and run from within a web browser.  
The full application runs front-end in a browser, and both end results and intermediate products can be inspected in a 3D viewer.

The tool offers functionalities such as point cloud loading, triangulation, and isocurve extraction.
These functionalities can be expanded upon though a plugin system which utilizes the existing "Node Package Manager" infrastructure and WebAssembly.
By using both, industry standard geoprocessing libraries such as `CGAL`, `GDAL` and `PROJ`, and data parsing libraries such as `IFC.js` and `laz-rs`, can be utilized.

In addition to the goal of examining wasm-based geo-computation, the auxiliary goal of this thesis is to make geoprocessing more accessible. 
By being free and open source, usable in a browser, and by focussing on the integration of existing geoprocessing libraries, GeoFront attempts to be a more accessible alternative to existing visual geo-computation environments, like FME or Grasshopper. 
This is done to be in line with the aforementioned cloud native vision of eventually allowing non-expert usage of GIS.

This design of geofront contains several challenges. Browser-based geo-computation (BBGC) might not be as performant as back-end / native geo-computation \cite{panidi_hybrid_2015, hamilton_client-side_2014}, even with the usage of WebAssembly \cite{jangda_not_2019}. 
Additionally, since current visual programming environments all fall short on at least one of the above features, the visual programming environment will have to be created from scratch.

\newpage
\section{Research Questions}
Based on these objectives, the research question is formulated as follows: 

\textit{"How to design and implement a front-end visual programming environment for containerized geo-computation?"}

\subsection*{Sub Questions}
The following sub-research questions are needed in order to answer the main question. 

\begin{enumerate}[a]
  \item \m{how to create a visual programming language?}: 
  \item \m{How to distribute existing geoprocessing libraries in a containerized manner?}: 
  \item \m{How to perform geo-computation with these containers in the browser?}: 
  \item \m{How to share types between libraries?}: 
  \item \m{How could these flowcharts be run in a cloud-native context?}:   
  \item \m{How well does THIS application serve certain use-cases?}:  
\end{enumerate}

\newpage
\section{Scope}
The scope of this thesis is defined in the following ways: 


\subsection*{ No back-end WebAssembly } 
This study will limit itself to the \emph{front-end} usage of WebAssembly. This looks contradictory to the goal of cloud-native geo-computation, and thus requires explanation.

This study investigates containerized, FAIR geoprocessing. While this is a prerequisite for cloud-native geospatial processing, is not necessarily needed to run code in the cloud to ideas and methods of containerization. 

This study uses the browser front-end as a proxy for containerization.
By making a traditionally native process run in a front-end web environment, it is safe to assume can also be run on a server. Additionally, by developing not just a method but also a full application, the method can be tested not only based on theoretical properties, but on actual applicability to use-cases as well. \todo{Improve this intro} 

-> goal cloud native: backend wasm

-> study: not possible due to time constraints.

Server-side or native wasm is beyond the scope of this paper

, but would be an excellent starting point for future work. Note that this also means that research into WebAssembly is important for more than just client-side geoprocessing. All geoprocessing could benefit from it.



\subsection*{ No Web Processing Services } 
Similarly, this study will exclude the OGC standard of the \ac{wps} \cite{ogc_web_2015}, since these services do not offer \emph{front-end} geoprocessing, but instead offer \emph{back-end} side geoprocessing. A client-side application \textit{could} create an interface to use such a service, to essentially offer geoprocessing to clients, but this study regards a solution like that as a workaround, not a true solution to the problem of client-side geoprocessing. 

This is not to say that client-side geoprocessing replaces the need for \ac{wps}. 
future work could research the possibility of utilizing a hybrid strategy of both client-side and server-side geoprocessing, following in the footsteps of \cite{panidi_hybrid_2015}. 



\subsection*{ No Usability Analysis } % 
\todo[]{Is this still true in this new thesis?}
While accessibility / usability is a motivation of this research, no claims will be made that the developed use-case is more usable to native GIS applications or geoprocessing methods. This research attempts to solve practical inhibitions in order to discover whether or not client-side is \emph{a} usable option. If it turns out that this method is viable technically, future research will be needed to definitively proof \emph{how} usable it is compared to all other existing methods.  

% This paper seeks to first close this gap, limiting itself to overcoming the technical and design boundaries in the pursuit of practical client-side geoprocessing.

Similarly, a survey analyzing how users experience client-side geoprocessing in comparison to native geoprocessing must also be left to subsequent research. While this would gain us a tremendous amount of insight, client-side geoprocessing is too new to make a balanced comparison. Native environments like QGIS, FME or ArcGIS simply have a twenty year lead in research and development. 

% It is my goal to introduce this as a new geoprocessing option, and to name the advantages and disadvantages we can be sure of. An actual comparison of client-side vs native geoprocessing is something different. 


\subsection {Only WebAssembly, no Docker}
Two technologies may pose a solution to FAIR geo-computation: Docker and WebAssembly.
This thesis examines a WebAssembly based approach to containerization. Approaches using Docker are not covered, and are left to subsequent research.

\subsection*{ 3D Vector / Point Cloud geoprocessing}
Geoprocessing, or geo-computation, is a sizable phenomenon. 
It covers all operations on geodata, from 2D rasters, 2D \& 3D vector data, tabular datasets, and point clouds. 
Due to time limitations, we are forced to provide one focus on particular type of geoprocessing
- raster
- table / 2D features

\subsection*{ Only core browser features }
GeoFront will be designed and constructed using only core browser features. 
- Only using basic HTML5 features.
- The research objective is to explore core web features, not to examine the front-end web ecosystem. 
- name all html things


\section{Guide}

As such, the study first describes the design and implementation of GeoFront, and then evaluates the tool to various geo-computation use cases.
The advantages and disadvantages of browser based geo-computation, compared to native or server-side geo-computation, are examined in several scenario's. 
Both quantitative indicators, like loading and runtime performance, as well as qualitative indicators, like the fitness for an intended use-case, are measured in each of these cases.