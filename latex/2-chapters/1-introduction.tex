\chapter{Introduction}

\m{->} Establish Cloud Native movement.

\begin{itemize}
  \item Chris Holmes, forerunner of the cloud native movement \& vision
  \item Radical simplification
  \item Massive online datasets, easely digestible by cloud-based computations
  \item Portable Geo-computation. sending not the data to a process, but the process to the data. This is a complete reversal of the current paradigm
\end{itemize}


\begin{enumerate}
  \item     All data of interest to a user is in the cloud; and these datasets will be far bigger than one could possibly fit on a desktop computer — worldwide imagery archives, historical gps data for fleets of assets, multi-dimensional weather data, global basemap data, etc.
  \item     Infinite computation capabilities are available to process massive amounts of data, \textbf{and the algorithms powering the processing can be shared online and customized collaboratively}.
  \item     Queuing and notification systems are in place, so newly acquired data (from satellites, from ground surveying, etc) can automatically kick off additional data processing, run updates, or send messages out to users.
  \item     Web tiled online maps are available to visualize any data (both source data and derived data in interim or final processing steps).
\end{enumerate}


\m{->} quickly mention the 'wider audience' narrative \& vision / denichifying argument

\emph{
  With the introduction of accessible, centralized data, and the dramatically different workflows that follow, Cloud Native Geospatial has the potential to introduce new, non-specialized users to the power of geospatial information that GIS practitioners have enjoyed for decades. [...]. The ecosystem of geospatial experts will collaborate to create analyses and insight, but any non-expert user will be able to select and apply those to the geographic area they care about. ~ Chris Holmes
}

\m{->} However, in order to make this vision a reality, there is much work to be done. Data formats, data conversion, tools, etc. 

One important open question regarding the vision is how to provide containerized geo-computation processes. Furthermore, it remains a mystery how these containerized containerized processes will be configured and accessed by frontend computation environments. 

\begin{itemize}
  \item This study focusses on the aspects of geo-computation \& tooling.
  \item Holmes: one of the vital ingredients: \emph{"and the algorithms powering the processing can be shared online and customized collaboratively"}.
  \item Not just open source: process sharing using fully containerized instances. Think Docker. Or WebAssembly.
\end{itemize}

\m{->} This study focus: cloud-native ready geo-computation tools. 

\m{->} current direction: containerized, sharable processes, together with web-based, front end visual programming environments ( RasterFoundry). Docker is usually named as a vision for these sharable processes.

\m{->} We do have examples of cloud-native geodata formats, and some examples of cloud-based geo-computation (RasterFoundry , Google Earth Engine, more). However, these approaches have not yet tried to use truly sharable, containerized geoprocesses using Docker or WebAssembly. 

\m{->} WebAssembly as a whole is underresearched. WebAssembly is not a fully virtualized container image, but just a binary set of instructions, meant to be executed on a virtual machine. Think of safe, cross-platform dll's. 
WebAssembly is in this regard more simple than docker, but this gives it more opportunities. 
WebAssembly runs in the browser for instance. 

\m{->} This opportunity to run in the browser would enable these cloud-native frontend environments to execute these processes from within the browser, completely detached from the server, as a means to experiment with processes on a small scale before applying them to a cloud native environment. 

\m{->} However, no implementations exist yet which combines containerized processes with these frontend computation environments. 

X : wasm-based geo-computation applications

\section{Research Objectives}
% we research front-end, as a proxy for containerization. if a process can be run in a front-end web envirionment, it can be run on the cloud. 

Due to the need for a first step towards wasm-based geo-computation applications, the goal of this thesis is to develop a method for geo-computation in a browser-based front-end. 

This thesis presents the design, creation and evaluation of GeoFront, a web-based, visual geo-computation tool. 
With GeoFront, geoprocessing flowcharts can be created, run, and shared from within a web browser.  
The full application runs front-end in a browser, and both end results and intermediate products can be inspected in a 3D viewer.

The tool offers functionalities such as point cloud loading, triangulation, and isocurve extraction.
These functionalities can be expanded upon though a plugin system which utilizes the existing "Node Package Manager" infrastructure and WebAssembly.
By using both, industry standard geoprocessing libraries such as `CGAL`, `GDAL` and `PROJ`, and data parsing libraries such as `IFC.js` and `laz-rs`, can be utilized.

In addition to the goal of examining wasm-based geo-computation, the auxiliary goal of this thesis is to make geoprocessing more accessible. 
By being free and open source, usable in a browser, and by focussing on the integration of existing geoprocessing libraries, GeoFront attempts to be a more accessible alternative to existing visual geo-computation environments, like FME or Grasshopper. 
This is done to be in line with the aforementioned cloud native vision of eventually allowing non-expert usage of GIS.

This design of geofront contains several challenges. Browser-based geo-computation (BBG) might not be as performant as back-end geo-computation \cite{panidi_hybrid_2015, hamilton_client-side_2014}, and WebAssembly 


\newpage
\subsection{Research Question}
Based on this objective, the research question is formulated as follows: 

\textit{"How to design, implement and evaluate a front-end visual programming environment for geo-computation?"}

\subsection*{Sub Questions}
The following sub-research questions are needed in order to answer the main question. 

\begin{enumerate}[a]
  \item \m{a thing}: 
  \item \m{drawing the canvas? drawing the }:   
  \item \m{loading modules? Sharing types in between modules?}: 
  \item \m{How to evaluate? How does it evalulate? }:  
\end{enumerate}

% - How to build a web app which fully runs in the browser
% - How to perform geo-computation with existing libraries in the browser?
% - How to make geo-computation accessible?
% - How could this use cloud-native geo types (COPC).
% - Who benefits from THIS web app?


- how to create a visual programming language?

\begin{enumerate}
  \item 
\end{enumerate}

\newpage
\section{Scope}
The scope of this thesis is limited in the following matters: 

\subsection{ 3D Vector / Point Cloud geoprocessing}
Due to time limitations, we are forced to provide one particular type of geoprocessing
- raster
- table / 2D features
- 


\subsection{ Only core browser features }
GeoFront will be designed and constructed using only core browser features. 
- Only using basic HTML5 features.
- The research objective is to explore core web features, not to examine the front-end web ecosystem. 
- name all html things



\subsection*{ No front-end WebAssembly } 
\todo[]{Still relevant?}
This study will limit itself to the \emph{front-end} usage of WebAssembly. 

-> goal cloud native: backend wasm

-> study: not possible due to time constraints.

Server-side or native wasm is beyond the scope of this paper

, but would be an excellent starting point for future work. Note that this also means that research into WebAssembly is important for more than just client-side geoprocessing. All geoprocessing could benefit from it.



\subsection*{ No Web Processing Services } 
Similarly, this study will exclude the OGC standard of the \ac{wps} \cite{ogc_web_2015}, since these services do not offer \emph{front-end} geoprocessing, but instead offer \emph{back-end} side geoprocessing. A client-side application \textit{could} create an interface to use such a service, to essentially offer geoprocessing to clients, but this study regards a solution like that as a workaround, not a true solution to the problem of client-side geoprocessing. 

This is not to say that client-side geoprocessing replaces the need for \ac{wps}. 
future work could research the possibility of utilizing a hybrid strategy of both client-side and server-side geoprocessing, following in the footsteps of \cite{panidi_hybrid_2015}. 



\subsection*{ No Usability Analysis } % 
\todo[]{Is this still true in this new thesis?}
While accessibility / usability is a motivation of this research, no claims will be made that the developed use-case is more usable to native GIS applications or geoprocessing methods. This research attempts to solve practical inhibitions in order to discover whether or not client-side is \emph{a} usable option. If it turns out that this method is viable technically, future research will be needed to definitively proof \emph{how} usable it is compared to all other existing methods.  

% This paper seeks to first close this gap, limiting itself to overcoming the technical and design boundaries in the pursuit of practical client-side geoprocessing.

Similarly, a survey analyzing how users experience client-side geoprocessing in comparison to native geoprocessing must also be left to subsequent research. While this would gain us a tremendous amount of insight, client-side geoprocessing is too new to make a balanced comparison. Native environments like QGIS, FME or ArcGIS simply have a twenty year lead in research and development. 

% It is my goal to introduce this as a new geoprocessing option, and to name the advantages and disadvantages we can be sure of. An actual comparison of client-side vs native geoprocessing is something different. 


\section{Guide}

As such, the study first describes the design and implementation of GeoFront, and then evaluates the tool to various geo-computation use cases.
The advantages and disadvantages of browser based geo-computation, compared to native or server-side geo-computation, are examined in several scenario's. 
Both quantitative indicators, like loading and runtime performance, as well as qualitative indicators, like the fitness for an intended use-case, are measured in each of these cases.