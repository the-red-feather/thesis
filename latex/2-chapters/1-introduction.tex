%---------------------%
% An introduction in which the relevance of the project and its place in the 
% context of geomatics is described, along with a clearly-defined problem statement.
%---------------------%
\chapter{Introduction}

\section{Motivation}
This thesis explores client-side geo-computation. 
- its relatively unknown 
- its needed  

The challenges of this second option are that browser-based geo-computation (BBG) might not be as performant as server side geo-computation \cite{panidi_hybrid_2015, hamilton_client-side_2014}, and that WebAssembly will have to be used to make use of industry-standard geoprocessing libraries such as Proj, Geos, GDAL, or CGAL.

with the general movement moving to the cloud, it becomes even more important to look at alternatives. 
- If study fails: we know that the move to the cloud is a wise one
- If study succeeds: we learn that in certain scenarios, client-based geo-computation is advantageous, and not ALL geo-computation should be done in the cloud.

this is why we study the second option. 

\newpage
\section{Research Questions}

\textit{"How to design and implement an accessible geo-computation web application by only using browser-based technologies?"}

\subsection*{Sub Questions}

The following sub-research questions are needed in order to answer the main question. The methodology chapter will explain the choices of these sub-questions. 

\newpage
\subsection*{Assessment}

%%%%%%%%%%%%%%%%%%%%%%%%%%%%%%%%%%%%%%%%%%%%%%%%%%%%%%%%%%%%%%%%%%%%%%%%%%%%%%%
\newpage
\section{Scope}
\subsection*{Included}
The scope of this thesis is represented by the Methodology chapter. 

\subsection*{Excluded}

% TIM: maybe move this to obstacle 3 ? 
\subsubsection*{ Server-side or Native WebAssembly } % **Client-side WebAssembly Only**

This study will limit itself to the \emph{client-side} usage of WebAssembly. 
A powerful case can be made for \emph{server-side}, or native level usage of WebAssembly, especially in conjunction with a programming language such as Rust. 
Rust compiled to WebAssembly could, compared to using python, java or C++, make geoprocessing more maintainable and reliable, while at the same time ensuring memory safety, security, and performance \cite{clack_standardizing_2019}. 

Server-side or native wasm is beyond the scope of this paper, but would be an excellent starting point for future work. Note that this also means that research into WebAssembly is important for more than just client-side geoprocessing. All geoprocessing could benefit from it.



\subsubsection*{ Web Processing Service } % Will not be dealing with WPS 

% offered as server-side geoprocessing services.  
Similarly, this study will exclude the OGC standard of the \ac{wps} \cite{ogc_web_2015}, since these services do not offer \emph{client} side geoprocessing, but instead offer \emph{server} side geoprocessing. A client-side application \textit{could} create an interface to use such a service, to essentially offer geoprocessing to clients, but this study regards a solution like that as a workaround, not a true solution to the problem of client-side geoprocessing. 

This is not to say that client-side geoprocessing replaces the need for \ac{wps}. 
future work could research the possibility of utilizing a hybrid strategy of both client-side and server-side geoprocessing, following in the footsteps of \cite{panidi_hybrid_2015}. 



\subsubsection*{ Usability Analysis } % 

While usability is a motivation of this research, no claims will be made that the developed use-case is more usable to native GIS applications or geoprocessing methods. This research attempts to solve practical inhibitions in order to discover whether or not client-side is \emph{a} usable option. If it turns out that this method is viable technically, future research will be needed to definitively proof \emph{how} usable it is compared to all other existing methods.  

% This paper seeks to first close this gap, limiting itself to overcoming the technical and design boundaries in the pursuit of practical client-side geoprocessing.

Similarly, a survey analyzing how users experience client-side geoprocessing in comparison to native geoprocessing must also be left to subsequent research. While this would gain us a tremendous amount of insight, client-side geoprocessing is too new to make a balanced comparison. Native environments like GRASSGIS, QGIS, FME or ArcGIS simply have a twenty year lead in research and development. 

% It is my goal to introduce this as a new geoprocessing option, and to name the advantages and disadvantages we can be sure of. An actual comparison of client-side vs native geoprocessing is something different. 








% \chapter{Justification}%%%%%%%%%%%%%%%%%%%%%%%%%%%%%%%%%%%%%%%%%%%%%%%%%%%%%%%%%
