\chapter{Introduction}

... (general introduction) ...
% 1) General story on geo-information. Go from data to application
% - geo-information applications are extremely valuable
% - bring the side of the gis application & geoprocessing to the reader's attention. 

In order to specify the contribution of this thesis, I wish to bring two phenomena to the reader's attention: The GIS visual programming language, and Web GIS applications. 
After this, the subject and goal this thesis can be made clear.

% The way maps are made is ever changing. Perhaps one of the most impactful ways this will change in the near future, is in the way described by the "Cloud Native Geospatial" vision. Proposed by the OGC, this movement envisions a future in which geodata storage and computation methods are re-imagined from the point of view of the cloud. 
% It aims to radically simplify geodata storehouses to static servers, serving large, singular binary geodata files. All processing and analysis of this geodata can then be performed by separate cloud-based web services, which could then run containerized processes on an unprecedented scale, with unprecedented speed. 

% <!-- Why is it different, why does it ask for change? what type of change? -->
% In order to make use of these advantages, both geodata storage and computation methods will need to be re-examined. 
% Cloud computation and cloud-based data access ask for different priorities and features over native, desktop based alternatives. 
% These features will either have to be added to existing formats and methods, or new substitutes formats and methods will need to be developed and tested.

% <!-- from cloud-native to vpl -->
% This study regards the requirements for cloud-native \textbf{computation} in , and specifically from the point of view of geospatial visual programming environments. 

% 2) Name relevant developments within the field of geo-informatics: more cloud-based, more massive datasets, a focus on open datasets, etc. 

\subsection*{Visual programming language}

\emph{What is a VPL and how is it used within the field of geo-information?}

% 2) Establish the VPL, and how it is used within geomatics 
A \ac{vpl}, or visual programming environment, is a type of programming language represented in a graphical, non-textual manner.
A VPL often refers to both the language and the Integrated Development Environment (IDE) which presents this language.
% Famous examples of VPLs are:
% \begin{itemize}
%   \item Ladder Diagram (src), the industry-standard method of programming Programmable Logic Controllers (PLCs)
%   \item Scratch (src), a educational programming language
%   \item 
% \end{itemize}

% ... (name something about Dataflow programming, a programming paradigm often used in conjunction with a vpl. ) ... 

Within the field of geo informatics, VPL's are most often used for specifying geodata transformations and performing spatial analyses.  
SaveSoft's FME (SOURCE) is a good example of this. This Extract Load Transform (ETL) automates data integration, and is widely used by GIS experts. 

% while McNeels's Grasshopper (SOURCE) is often used spatial analysis of buildings and cities, like solar irradiation. 

% Name Rasterfoundry ModelLab

% Name GeoFlow maybe

VPL's like these offers users a chance to interactively automate workflows \& processing pipelines, while requiring little to no programming knowledge. 
In between results can be inspected quickly, and the processes can be changed on the fly, often with immediate feedback.
This advantage of interactive, low-code automation is why the VPL continues to be a popular interface within the field of GIS, as well as all many other use-cases in need of both low-code automation and visual debugging, like BIM, CAD, Shader Programming and Procedural Geometry. 
A VPL done right can make automation available to a very large audience. 

% - "As massive datasets become more and move available, the availability of the \emph{means} to process and analyse these data becomes more and more relevant". 

% Despite these advantages, open, accessible vpl's are often unavailable. 
% - Expensive, proprietary environments 
% - Require specific plugins in order to utilize open, industry standard geo-computation libraries
% - Not FAIR geoprocessing



\subsection*{Web GIS}
Web \ac{gis} forms an indispensable component of the wider geospatial software landscape. 
For the average person, an interactive \ac{gis} web application is often their first and only exposure to a \acs{gis}, be it a web mapping service, a navigation system, or a pandemic outbreak dashboard. 
A web application is cross-platform by nature, and offers ease of accessibility, since no installment or app-store interaction is required to run or update the app. (src: vpl 2019, src: hybrid)
As soon as it can be found, it can be used.
The ability to share a full application with a link, or to embed it within the larger context of a webpage is also not trivial. 
Together, these aspects have made the browser a popular host for many important geographical applications, especially when accessibility is vital.

% "making functionalities available not only to developers, but to the masses."

% The cloud-native geospatial movement represents a significant opportunity and challenge to these VPL interfaces. The \textbf{opportunity} lies in the fact that a VPL interface is highly suitable as a 'configurator' or 'IDE' for cloud-computation. The promise of interactive, low-code automation matches the desire of most cloud native geoprocessing providers to support users of different backgrounds, both programmers and non-programmers, both full GIS experts as well as non-experts (SOURCE: Modellab, SOURCE: Chris Holmes). A good example of this is the ModelLab VPL, found in Raster Foundry (SOURCE). This is also evident in the fact that existing VPL's like FME and Grasshopper have added proprietary cloud-computation features like FME Cloud (SOURCE) and ShapeDiver (SOURCE), respectively.

% <!-- HOWEVERRRRRRRRR -->
% However, the major \textbf{challenge} is that as of right now and despite these developments, popular VPL's fall short on a number of priorities and features required for cloud-native geospatial computation. 
% These shortcomings include their closed and proprietary nature, their distance from regular programming features and conventions (git version control, continuous integration), and the non-containerized, one to one relationship between the IDE application \& the cloud hosting platform. All of this hinders their suitability as general, cloud-computation configurers. 

% FOCUS: CONTAINERIZATION & INTEROPERABILITY

% Sub component: FAIR: 
% - FINDABLE:      Hard to find the right tools for the job
% - ACCESSIBLE:    Hard to access these tools (install, setup environment, look at what you are doing)
% - INTEROPERABLE: Hard to use two tools from different ecosystems (bindings, plugins, etc). 
% - REUSABLE:      Hard to re-use a specific scripts written for one use case in another use case

% For a geoprocessing method to match the 'cloud native geospatial movement'

% Currently, WebAssembly is the only way one can take a geoprocessing function from an arbitrary source, and use it in both the browser and on a server. 
% We wish to discover if this has any practical use cases. 


\section{Research Objectives}
the goal of this thesis is to develop a novel method to perform geo-computation in a web browser using a \ac{vpl}. 
By doing so, this thesis seeks to connect the advantages of VPL-based geoprocessing (interactivity) to the advantages of web-based applications (distribution and accessibility). 
The method differentiates itself by allowing client-side, browser-based calculations. 
To the best of the authors knowledge, no prior research on the topic of web-based geo-VPLs exists.


% Due to the limitations of geo-computation using proprietary vpl's, the goal of this thesis is to develop an alternative, open vpl for geo-computation. This approach is specifically aimed to improve the accessibility, interactivity and reproducibility of existing geo-computation libraries. It builds on the preliminary developments by (Ravi Peters), who suggested the usage of vpl's to aid the development process of geo-computation pipelines.  

% -> testing & reproducability.
% RANSAC -> many 'magic' parameter. They need to be discovered by 'play'
% Jonathan blow -> using interactive applications, an intrinsic understanding can be gained without explicit communication.
% Game Of Life -> impossibility of 'proving' behaviour systems. 

% This design of geofront contains several challenges. Browser-based geo-computation (BBGC) might not be as performant as back-end / native geo-computation \cite{panidi_hybrid_2015, hamilton_client-side_2014}, even with the usage of WebAssembly \cite{jangda_not_2019}. 
% Additionally, since current visual programming environments all fall short on at least one of the above features, the visual programming environment will have to be created from scratch.

\section{Research Questions}
Based on this objective, the research question is formulated as follows: 

\textit{" How to design and implement a web-based geo-vpl? "} % (according to x)

% To what extent can 'a web-based vpl' make '3d geo-computations' more [ Accessible / Available / Interactive / ~~FAIR~~ / ... ] than proprietary vpl's?

\subsection*{Sub Questions}
The following sub-research questions are needed in order to answer the main question. 

\begin{enumerate}[a]
  \item Design: Which features are required for a 'usable' VPL intended for geo-computation? 
  \item Implementation: To what extend can a web browser facilitate these features?
  \item Usage: Who benefits from a web-geo-vpl? 
  \item Evaluate: To what extend does this solution contribute to making geoprocessing more Findable, Accessible, Interoperable, and Reusable, compared to ... ?  
\end{enumerate}

In order to obtain answers to these questions 
1) an literature review is done, 
2) a prototype software application is developed as a proof of concept and 
3) this application is tested and compared to existing methods using real-world datasets.

% FROM MEETING 27:
% \item 

% If I have to boil everything down to one thing and one thing only, I would choose: Connection to regular software development. 
% -> machine readability
% -> 

% ELLIE 
% - To what extent can machine learning provide a better estimate of the number of floors than a purely geometric approach?

% a. Which features are related to the number of floors? Is there any overlap between these features
% and which subset yields the best results?
% b. Which machine learning algorithm provides the best results? How are the results affected by
% feature subsets that reflect different levels of data availability?
% c. What level of performance can be achieved compared to a purely geometric approach? What types
% of gross errors are present?
% d. Since floor count is generally an integer value, is this a regression or classification problem? If
% considered regression, how does rounding the predictions affect the results?

% RAVI

% The main research question of this thesis is defined as follows:
% • Is the Voronoi- and surface-based approach a viable option for
% the automatic generation of depth-contours for hydrographic
% charts?
% Additionally the following set of sub-research questions are defined:
% 1. What characterizes surfaces that lead to good depth contours
% for hydrographic charts and what is needed in terms of interpo-
% lation and generalization to achieve such a surface?
% 2. Are those characterizations respected in the Voronoi- and surface-
% based approach?
% 3. Does the Voronoi- and surface-based approach perform well for
% heterogeneously distributed input data?
% 8 introduction
% 4. To what extent can the Voronoi- and surface-based approach be
% automated?
% 5. Is the Voronoi- and surface-based approach well scalable to big
% datasets?
% In order to obtain answers to all these questions 1) an extensive liter-
% ature review is done, 2) a prototype software application is developed
% as a proof of concept and 3) this application is tested and compared
% to existing methods using a number of real-world datasets

% FAIR APPROACH:

% \begin{enumerate}[a]
%   \item To what extend does the format of web application contribute to making point-cloud computations more \emph{Findable}?  
%   \item To what extend does the format of a visual programming language make point-cloud computations more \emph{Accessible}?  
%   \item Do point-cloud geoprocessing libraries become more \emph{Interoperable} when compiled to WebAssembly?  
%   \item Do point-cloud geoprocessing libraries become more \emph{Reusable and Reproducable} when compiled to WebAssembly?  
% \end{enumerate}

% How to design and implement a browser-based visual programming environment for containerized geo-computation?”

% \todo[]{Maybe translate this to 'how' questions: "How can X make Y more Findable?"}



% Leave this empty -> 

% SQ: one on ~libraries

% Q Can you use .... in webassembly?
% Q: how can we re-use code 
% A: yes, but with 

% SQ: one on VPL

% SQ: HOW DOES THIS SOLUTION FIT THE FAIR PRINCIPLES

% people need to follow from introduction
% ideally they should split from the main question
% all the main work


% FAIR RESEARCH QUESTIONS -> conclusion
% \begin{enumerate}[a]
%   \item Does the format of web application make point-cloud computations more \emph{Findable}?  
%   \item Does the format of a visual programming language make point-cloud computations more \emph{Accessible}?  
%   \item Do point-cloud geoprocessing libraries become more \emph{Interoperable} when compiled to WebAssembly?  
%   \item Does the ability to compile a VPL script to javascript make those VPL scripts more \emph{Reusable}?  
% \end{enumerate}


% PRACTICAL RESEARCH QUESTIONS
% Stelios: waaay too specific
% a how to create a visual programming language?:
% b How to distribute existing geoprocessing libraries in a containerized manner?:
% c How to perform geo-computation with these containers in the browser?:
% d How to share types between libraries?:
% e How could these flowcharts be run in a cloud-native context?:
% f How well does this application serve certain use-cases ?:

\newpage
\section{Scope}
The scope of this thesis is defined in the following ways: 


\subsection*{ No back-end WebAssembly } 
This study will limit itself to the \emph{front-end} usage of WebAssembly. This looks contradictory to the goal of cloud-native geo-computation, and thus requires explanation.

While this is a prerequisite for cloud-native geospatial processing, is not necessarily needed to run code in the cloud in order to test ideas and methods of containerization. 
Instead, this study investigates containerized, FAIR geoprocessing as an exercise alone.  


This study choice the browser front-end, because it serves as a good proxy for containerization:
By making a traditionally native process run in a front-end web environment, it is safe to assume can also be run on a server. 
Additionally, by developing not just a method but also a full application, the method can be tested not only based on theoretical properties, but on actual applicability to use-cases as well. 

Still, back-end containerized geoprocessing would be an excellent follow-up investigation to this study. 

\subsection*{ No Web Processing Services } 
Similarly, this study will exclude the OGC standard of the \ac{wps} \cite{ogc_web_2015}, since these services do not offer \emph{front-end} geoprocessing, but instead offer \emph{back-end} side geoprocessing. A client-side application \textit{could} create an interface to use such a service, to essentially offer geoprocessing to clients, but this study regards a solution like that as a workaround, not a true solution to the problem of client-side geoprocessing. 

This is not to say that client-side geoprocessing replaces the need for \ac{wps}. 
future work could research the possibility of utilizing a hybrid strategy of both client-side and server-side geoprocessing, following in the footsteps of \cite{panidi_hybrid_2015}. 

\subsection*{ No Usability Analysis } % 
\todo[]{Is this still true in this new thesis?}
While accessibility / usability is a motivation of this research, no claims will be made that the developed use-case is more usable to native GIS applications or geoprocessing methods. This research attempts to solve practical inhibitions in order to discover whether or not client-side is \emph{a} usable option. If it turns out that this method is viable technically, future research will be needed to definitively proof \emph{how} usable it is compared to all other existing methods.  

% This paper seeks to first close this gap, limiting itself to overcoming the technical and design boundaries in the pursuit of practical client-side geoprocessing.

Similarly, a survey analyzing how users experience client-side geoprocessing in comparison to native geoprocessing must also be left to subsequent research. While this would gain us tremendous insight, client-side geoprocessing is too new to make a balanced comparison. Native environments like QGIS, FME or ArcGIS simply have a twenty year lead in research and development. 

% It is my goal to introduce this as a new geoprocessing option, and to name the advantages and disadvantages we can be sure of. An actual comparison of client-side vs native geoprocessing is something different. 


\subsection {Only WebAssembly, no Docker}
This thesis examines a WebAssembly based approach to containerization. Approaches using Docker are not covered, and are left to subsequent research.

\subsection*{ Only 3D Vector / Point Cloud geoprocessing}
Geoprocessing, or geo-computation, is a sizable phenomenon. 
It covers all operations on geodata, from 2D rasters, 2D \& 3D vector data, tabular datasets, and point clouds. 
Due to time limitations, we are forced to provide one focus on particular type of geoprocessing
- raster
- table / 2D features

\subsection*{ Only core browser features }
GeoFront will be designed and constructed using only core browser features. 
- Only using basic HTML5 features.
- The research objective is to explore core web features, not to examine the front-end web ecosystem. 
- name all html things

\subsection*{ Visual Programming }
The format chosen for the demo application GeoFront is a visual programming language. Visual programming has the advantage 

\section{Guide}

...
\begin{itemize}[\m{->}]
  \item chapter 2: Thorough explanation \& on web VPL and web GIS. 
  \subitem Developments and challenges within these fields.
  \item chapter 3: Specify 'the solution', Convert the literature of chapter 2 into requirements for the solution. 
  \subitem Thesis is successful in how far the solution can meet these requirements. 
  \item chapter 4: Describe implementation details, and especially in relationship to these requirements
  \item chapter 5: Show the results of usage of GeoFront \& the results of experiments.
  \subitem These are used to evaluate the solution.
  \item chapter 6: discuss to which extend the solution was able to satisfy the requirements.
  \item chapter 7: conclude to which extend the solution was able to satisfy the requirements.
\end{itemize}

...

% As such, the study first describes the design and implementation of GeoFront, and then evaluates the tool to various geo-computation use cases.
% The advantages and disadvantages of browser based geo-computation, compared to native or server-side geo-computation, are examined in several scenario's. 
% Both quantitative indicators, like loading and runtime performance, as well as qualitative indicators, like the fitness for an intended use-case, are measured in each of these cases.


\section{Self-reflection} 
