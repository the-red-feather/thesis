\chapter{Introduction}
Web \ac{gis} forms an indispensable component of the wider geospatial software landscape. 
For the average person, an interactive \ac{gis} web application is often their first and only exposure to a \acs{gis}, be it a web mapping service, a navigation system, or a pandemic outbreak dashboard. 
A web application is cross-platform by nature, and offers ease of accessibility, since no installment or app-store interaction is required to run or update the app: 
As soon as it can be found, it can be used.
The ability to share a full application with a link, or to embed it within the larger context of a webpage is also not trivial. 
Together, these aspects have made the browser a popular host for many important geographical applications, especially when accessibility is vital.

The field of web GIS is, just like the wider context of the web, subject to many changes \& developments. 
These come in the shape of a new feature reaching a consensus in the major browser vendors, or a change in commonly used standards like the \ac{http}.
These changes may lead to a 'trend': a new way of reasoning about or organizing a web service or a web application.

- Recent improvements in the javascript runtime are a vital component of the current shift from backend-heavy to frontend-heavy web applications. 
- front-end Libraries like Celsium, Leaflet and D3 would be hard if not impossible to realize without HTML5 features like WebGL and the Canvas API.
- The cloud native geospatial movement requires the http range request feature to truly be effective.

- interplay between features and trends 
- trends lead to interesting new applications which may serve new use-cases
-> the influence and importance of the above web features and trends are why new developments within web GIS are very important to examine.



Geofront has been created to study how recent web features and trends can have a positive effect on the accessibility of geo-computation. 
The hypothesis is that the combination of cloud-native geodata together with front-end heavy applications with industry standard geoprocessing libraries using webassembly, may lead to new, powerful types of GIS applications and use-cases. 

an experiment to capitalize on these developments within the field of web GIS. 


...
In the field of Web GIS, a pattern can be recognised between web \& browser features,  
- javascript runtime improvements -> shift from backend-heavy to frontend-heavy web applications
- HTML5 with WebGL and Canvas Api -> Celsium, Leaflet, D3
- http range request -> cloud native geospatial movement 
- WebAssembly -> ...

Now we see a new technology 

This study has been conducted for two reasons.
Firstly, this study wishes explore the possibilities \& limitations of front-end geoprocessing.
  - Web technologies have become a vital, core component of the modern geospatial software landscape.
  - However, the web is changing : cloud-native geospatial movement, move to front-end heavy applications, WebAssembly. 
  - By creating an application using these trends and technologies, we can get a better understanding.

Secondly, to experiment with making small-scale geoprocessing more accessible
accessible alternative to existing visual geo-programming environments.

\section{Research Objectives}
Due to this need for a reorientation and re-examination of web features in relation to geo-computation, 
the goal of this thesis is to develop an alternative method for geo-computation on the web.

This thesis presents the design, creation and evaluation of GeoFront, a web-based, visual geo-computation tool. 
With GeoFront, geoprocessing flowcharts can be viewed, run, and shared from within a web browser.  
The full application runs front-end in a browser, and both end results and intermediate products can be inspected in a 3D viewer.

The tool offers functionalities such as point cloud loading, triangulation, and isocurve extraction.
These functionalities can be expanded upon though a plugin system which utilizes the existing "Node Package Manager" infrastructure.
Together with WebAssembly, this enables the utilization of industry standard geoprocessing libraries such as `CGAL`, `GDAL` and `PROJ`, and data parsing libraries such as `IFC.js` and `laz-rs`.

By being free, usable in a browser, and by focussing on the integration of existing geoprocessing libraries, GeoFront aims to be a more accessible alternative to existing visual geo-computation environments.

Geofront has been created as an experiment to explore if visual, browser-based geo-computation can make geo-computation more accessible. 

% As such, the study first describes the design and implementation of GeoFront, and then evaluates the tool to various geo-computation use cases.
% The advantages and disadvantages of browser based geo-computation, compared to native or server-side geo-computation, are examined in several scenario's. 
% Both quantitative indicators, like loading and runtime performance, as well as qualitative indicators, like the fitness for an intended use-case, are measured in each of these cases.

% The design of geofront contains several challenges. 
% As such ...
% The challenges of this second option are that browser-based geo-computation (BBG) might not be as performant as server side geo-computation \cite{panidi_hybrid_2015, hamilton_client-side_2014}, and that WebAssembly will have to be used to make use of industry-standard geoprocessing libraries such as Proj, Geos, GDAL, or CGAL.




\newpage
\subsection{Research Question}
Based on this objective, the research question is formulated as follows: 

\textit{"How to design, implement and evaluate a front-end visual programming environment for geo-computation?"}

\subsection*{Sub Questions}
The following sub-research questions are needed in order to answer the main question. 

\begin{enumerate}[a]
  \item \m{a thing}: 
  \item \m{drawing the canvas? drawing the }:   
  \item \m{loading modules? Sharing types in between modules?}: 
  \item \m{How to evaluate? How does it evalulate? }:  
\end{enumerate}

% - How to build a web app which fully runs in the browser
% - How to perform geo-computation with existing libraries in the browser?
% - How to make geo-computation accessible?
% - How could this use cloud-native geo types (COPC).
% - Who benefits from THIS web app?


- how to create a visual programming language?

\begin{enumerate}
  \item 
\end{enumerate}

\newpage
\section{Scope}
The scope of this thesis is limited in the following matters: 

\subsection{ 3D Vector / Point Cloud geoprocessing}
Due to time limitations, we are forced to provide one particular type of geoprocessing
- raster
- table / 2D features
- 


\subsection{ Only core browser features }
GeoFront will be designed and constructed using only core browser features. 
- Only using basic HTML5 features.
- The research objective is to explore core web features, not to examine the front-end web ecosystem. 
- name all html things



\subsection*{ No Server-side or Native WebAssembly } 
\todo[]{Still relevant?}
This study will limit itself to the \emph{client-side} usage of WebAssembly. 
A powerful case can be made for \emph{server-side}, or native level usage of WebAssembly, especially in conjunction with a programming language such as Rust. 
Rust compiled to WebAssembly could, compared to using python, java or C++, make geoprocessing more maintainable and reliable, while at the same time ensuring memory safety, security, and performance \cite{clack_standardizing_2019}. 

Server-side or native wasm is beyond the scope of this paper, but would be an excellent starting point for future work. Note that this also means that research into WebAssembly is important for more than just client-side geoprocessing. All geoprocessing could benefit from it.



\subsection*{ No Web Processing Services } 
Similarly, this study will exclude the OGC standard of the \ac{wps} \cite{ogc_web_2015}, since these services do not offer \emph{client} side geoprocessing, but instead offer \emph{server} side geoprocessing. A client-side application \textit{could} create an interface to use such a service, to essentially offer geoprocessing to clients, but this study regards a solution like that as a workaround, not a true solution to the problem of client-side geoprocessing. 

This is not to say that client-side geoprocessing replaces the need for \ac{wps}. 
future work could research the possibility of utilizing a hybrid strategy of both client-side and server-side geoprocessing, following in the footsteps of \cite{panidi_hybrid_2015}. 



\subsection*{ No Usability Analysis } % 
\todo[]{Is this still true in this new thesis?}
While accessibility / usability is a motivation of this research, no claims will be made that the developed use-case is more usable to native GIS applications or geoprocessing methods. This research attempts to solve practical inhibitions in order to discover whether or not client-side is \emph{a} usable option. If it turns out that this method is viable technically, future research will be needed to definitively proof \emph{how} usable it is compared to all other existing methods.  

% This paper seeks to first close this gap, limiting itself to overcoming the technical and design boundaries in the pursuit of practical client-side geoprocessing.

Similarly, a survey analyzing how users experience client-side geoprocessing in comparison to native geoprocessing must also be left to subsequent research. While this would gain us a tremendous amount of insight, client-side geoprocessing is too new to make a balanced comparison. Native environments like QGIS, FME or ArcGIS simply have a twenty year lead in research and development. 

% It is my goal to introduce this as a new geoprocessing option, and to name the advantages and disadvantages we can be sure of. An actual comparison of client-side vs native geoprocessing is something different. 








% \chapter{Justification}%%%%%%%%%%%%%%%%%%%%%%%%%%%%%%%%%%%%%%%%%%%%%%%%%%%%%%%%%
