% Define the scope, extend, and how of the study
\chapter{Methodology}
This chapter outlines the methodology of this study. 
The method can be subdivided into four chapters:

\begin{enumerate}[(1)]
  \item Define the requirements a web-based geo-vpl needs to meet up to.
  \item Develop the vpl according to these requirements.
  \item Create multiple demo GIS applications using this vpl.
  \item Based on these demo's, access to which extend the requirements are met.  
\end{enumerate}

This process is cyclical.
If the vpl scores significantly low on some criteria, the requirements or Functionality of the vpl might need to be altered, after which the applications require reassessment. 


\section{Requirements}

\emph{Answer this sub research question: Which features are required for a 'usable' VPL intended for geo-computation?}


The success of this study is measured by accessing to what extend GeoFront adheres to its requirements. 
These requirements are in turn based upon related works within VPL research, as well as web GIS studies. 
This section outlines these requirements, and why they are used. 
Additionally, certain use-cases are envisioned as proxies of these environments, to test certain requirements in a realistic scenario. 



\subsection*{A: Base features}
\begin{lstlisting}
a list of standard VPL features & application features required as a base-line
- inspect data
- read / write data
- ....
- ....
\end{lstlisting}

% NOTE: these are the '1 or 2 major contributions, besides making a web-geo-vpl'
\subsection*{B: Extendability}
\begin{lstlisting}
- ease of creating and using plugins 
- ease of using existing geo-libraries 
  - written in non-js languages
\end{lstlisting}

\subsection*{C: Application Publication}
\begin{lstlisting}
- ease of sharing apps as a visual program
- ease of compiling and running the app headless
\end{lstlisting}


% \begin{lstlisting}

% Methodology: define the scope, extend, and how of the study 

% why these sub questions? How will the sub-questions be answered?
% - Findable: THE WEB-APP INTRODUCTION / CLIENT-SIDE GEOPROCESSING
% - Accessible: THE FIRST PART OF THE VPL INTRODUCTION -> VPL advantages
% - Interoperable: THE WEBASSEMBLY INTRODUCTION / CONTAINERIZATION 
% - Re-usable: THE SECOND PART OF THE VPL INTRODUCTION -> VPL's have shortcomings


% \end{lstlisting}



This second assessment will be a judgement based on four distinct use-cases for the environment:


\section{ Evaluation }


% The entire application runs client-side in a browser, and uses a visual programming language as its primary \ac{gui}.
% The main goal and feature of geofront is to take existing low level geo-computation libraries, and to make these interactively usable on the web. 
% These libraries include a limited set of CGAL operations, complied from C++, and various geo-computation algorithms such as Startin, written in Rust. 
% Being a visual programming language, GeoFront can be used to interactively alter the geodata pipeline. 
% In between products can quickly be inspected using a 3D viewer.

% We test how well contemporary web technologies support such an application, as well as judge aspects such as accessibility \& performance of said application. We also judge if this type of application is indeed beneficial and usable as a scripting / demo environment.  

% These features could all be implemented by normal means ( buttons, panels, sliders ) -->

% CHOICE: do something in-between python bindings, and a full fletched end-user application. 
% Ergo: Visual programming

% For input, the environment offers \ac{wms} and \ac{wfs} support, as well as ways for users to load locally stored geodata. Parameters can be specified using various ui components, such as sliders. 
% For output, the environment can be used to either save data to the user's local machine, or to visualize the results within the geofront application using a WebGL based viewport.

% Where ModelLab is build on top of recent improvements to the accessability of satellite imagery, GeoFront is build in anticipation to a similar development for point cloud datasets with the introduction of COPC. The focus of Geofront is therefore on point cloud processing, and point-cloud based modelling, such as Digital Terrain Models (DTM). 




\section{use cases}
We envision four distinct use-cases which might benefit from browser-based geoprocessing. 

% Assessment

1. Tryout (ACTUAL)
   - A-la wapm WebAssembly Package Manager allows packages to be run from within the package-page itself. 
  - Just meant to quickly try out some features. To access, and reproduce

2. Educational (ACTUAL)
   - interactive educational tool
   - (What does a delaunay triangulation look like? how does it behave? What happens if you lower the radius of inverse distance weighting ? )

3. Rapid-Prototyping (POSSIBLE)
   - Web geoflow
   - Future work: export flowchart to a process which can be run natively or server side.

4. Publishing (POSSIBLE)
   - Geotiff.io
   - Web FME 
   - Publish full web apps in and off themselves, making use of zero, one or multiple wasm-compiled libraries.  
   - Future work: export to web-app (without flowchart)








% IMMIGRANT FROM JUSTIFICATIONS CHAPTER. MUST BE MOVED 
% OR MAKE THIS A FULL CHAPTER
\section{Use Cases}

% # A: 1. Geofront as a geoprocessing / analysis demo tool.
% - Frame Geofront as an expanded version of https://validator.cityjson.org/ this. 
%   - [this](https://validator.cityjson.org/) (a wasm web demo) + jsfiddle 
%   - Use rust, web, and c++ tools side by side, hand in hand

% # A: 2. Geofront as an integrator

% # A: 3. Geofront as a lightweight QGIS replacement
% - The web is already used for geodata retrieval and visualization. If geodata processing & analysis is also possible, 
% there is nothing preventing geofront for becoming a full GIS.

% # A: 4. Geofront as a client-side, web based grasshopper replacement
% Advantages over grasshopper: 
% - No install 
% - Could compile to client-side application.  


\subsubsection{Geofront as Web Demo application}




% _name the cityjson web tool_

% _name web based demo environments, such as jupyter notebook_

% _web demo's like these promote accessibility & science communication._

% _name the importance of interactivity, make a case for visual programming_

% _make the case for a new web based demo environments, to house applications such as the cityjson web tool_

% <br><br>

% ## Problems with publication

% Within the field of geo-informatics, we want to share our end-results. 
% - Usually on git, but this has limitations:
%     - Not everybody can immediately use it ( unfamiliar language / build system),
%     - Even people who can understand, often wont go through the trouble.  
%     - "Python bindings" -> half-solves the problem, but still hard to publish to a general audience. 

% This was the exact reason for developing https://validator.cityjson.org/. This solved the issue of publication. Why? 
% - Extremely findable, usable, accessible
%   - Cross-platform
%   - No install 
%   - first point of contact is precisely where you can use it
%   - You can send a link not to a download page, but to the application itself
%   - Great for communication: blogs with embedded applications.
%   - Code sharing: you exactly know what to expect.

% <br><br>

% ## Web Demo & Scripting environments

% we are not the first to recognize the suitability of the web for publishing demo's

% We see a lot of interactive web-demo's nowadays, and many of them are embedded within a type of "Demo Hosts":

% - Scripting environments in (Science) Communication:
%   - Jupyter Notebook 
%   - Observable
%   - JsFiddle
%   - Shadertoy
%   - Wapm

% - Scripting environments in Education: 
%   - TU Delft C++ course
%   - Udemy

% - Scripting environments in Tutorials: 
%   - Rust
%   - Lit

%   <!-- - (game-jam games)
%       - more save (no virus) -->

% <!-- We also see 

% - As accessible alternative to native
%   - Overleaf -> does not use webassembly, but a classic client-server architecture
%   - Google Earth -->

% All these applications lie on a crossroad between being an interactive demonstration of a certain result or phenomenon, 
% and an open invitation for the user to edit and use this result or phenomenon. 
% (CITE A STUDY PROVING HOW INTERACTION BENEFITS LEARNING), 
% so toying around is important.

% <!-- So it is save to say the web is suitable for these types of things. 
% But is the web also suitable for more? Can we use a web-based sandbox environment to -->


% we want to examine and edit the geodata flow, see for example, where these errors occur, try to get to that data, see if we can make hotfixes, etc. etc. 






\subsubsection{Geofront as Educational Sandbox}
- This use-case can be fully realized within the current state of geofront
- "Geoprocessing for kids"
- "What is a delaunay triangulation?" 
- "Let people play / experience / traverse a nef polyhedron"
- Using something helps with understanding

\subsubsection{Geofront as Web Demo Environment}
- Reproducibility toolkit:
- Workflow: 
  - Load your own code from a CDN
  - Build a demo setup around it
  - load a custom graph from a public json file
  - share a url pointing to this json (which contains the CDN address)
- You can now share a rust / C++ program as a fully usable web demo,   
  and analyze its performance using different datasets, test parameters, etc. 
- interdisciplinary exchange of ideas
- MISSING FEATURE: dependency list inside of the graph.json save file

\subsubsection{Geofront as End User Geoprocessing Environment}
- Lightweight QGIS.
- FME, but open source \& on the web.
- The tool in itself can be regarded as an end-user application:
  - Load file, do something with the file, download resulting file
  - REQUIRES WAY MORE SUPPORTING LIBRARIES AND TOOLS


% Under normal circumstances, Web applications within the field of geo-informatics are mostly used for the first and final stages of a common geodata process.  
% (
% If one wishes to retrieve geodata, web portals are used to discover the required datasets. After this, the OGC Web Services are often utilized to download and truly access this geodata. 

% This geodata is then processed locally, using QGIS, ArcGIS, command line tools, or any other 

% and at the end Tools like Leaflet and Celcium have been created to visualize the earth in both 2D and 3D , and tools like d3.js can produce interactive graphs to supplement these web pages. 

% There is, however, more to the web than just visualization. Due to 
% )
% This thesis asks the question if the web could do more than just visualization. 

% By creating the use-case application GeoFront, we ask the question if a modern web browser, and the current state of the client-side web technologies are capable of facilitating  


% ## Motivation
% Under normal circumstances, Web applications within the field of geo-informatics are mostly used for the first and final stages of a common geodata process: Retrieval and visualization. 
% The processing stages in-between are almost always performed on the desktop using environments like QGIS or ArcGIS, or by writing and using CLI tools. 

% <!-- At the same time, a need arrises -->

% ## This Study

% This thesis explores if these in-between processing steps could also be performed within a web application.  
% This way, geodata processing applications could profit from the ease of accessibility and maintainability granted by the web as a platform.  

% <!-- More Why's: 
% - making the geoweb more feature-rich
% - allowing quick demonstration (wapm)
% - allowing easy access (overleaf)


%  -->

% (similar to how overleaf is more accessible than desktop latex installation & usage)

% We ask ourselves if the current state of client-side web technologies are capable of facilitating multiple steps of geodata conversion, and what such an application would look like. 

% To concretize this question, this study covers the design and creation a use-case application titled "Geofront". 

% By designing and creating this environment, the study seeks to gain valuable insights in the current state of client-side core web technologies / javascript std, and how well this facilitates various geoprocessing operations.

\subsubsection{Geofront as **Rapid Prototyping Environment**} 
  - Ravi's GeoFlow, but on the web
    - Meant to visually debug a certain process, after which this process can be 'compiled' to a normal cli tool.
  - CURRENTLY MISSING FEATURE: compile to native cli tool (node.js script)
  - REQUIRES WAY MORE SUPPORTING LIBRARIES AND TOOLS
