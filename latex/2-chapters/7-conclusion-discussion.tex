% repeat results and answers in shortened form
\chapter{Conclusion \& Discussion}
\label{chap:conclusion}

This chapter consists of the answers to the research
questions that were given in the first chapter (\refsec{sec:conclusion}), 
a summary of the most significant contributions (\refsec{sec:contribution}), 
a discussion of the value and quality of this study (\refsec{sec:discussion}),
a number of theorized implications of this study (\refsec{sec:future-work}),
and lastly, a self reflection on \refsec{sec:reflection}.

\begin{note}
  Important: 
  - Show what I have learned
  - Show maturity
  - Recommendations 
  - Show balance
  - Show finality
\end{note}

%%%%%%%%%%%%%%%%%%%%%%%%%%%%%%%%%%%%%%%%%%%%%%%%%%%%%%%%%%%%%%%%%%%%%%%%%%%%%%%
%%%%%%%%%%%%%%%%%%%%%%%%%%%%%%%%%%%%%%%%%%%%%%%%%%%%%%%%%%%%%%%%%%%%%%%%%%%%%%%
%%%%%%%%%%%%%%%%%%%%%%%%%%%%%%%%%%%%%%%%%%%%%%%%%%%%%%%%%%%%%%%%%%%%%%%%%%%%%%%

\section{Conclusion}
\label{sec:conclusion}

This paragraph answers the research questions. 
It starts with answering the sub research questions and concludes with the answer to the main research question

\subsection*{Sub Questions}

\begin{itemize}[ ]
  \item \emph{\mySubRQOneTitle:} "\mySubRQOne"
\end{itemize}

[conclusion on implementation details]

Answer of "\mySubRQOneTitle": Show the full application implementation. Conclude with what about this implementation is relevant for geocomputation, and in what ways the browser is limiting, or beneficial.  

- web is more than capable for representation
  - canvas API \& webgl are more than powerful enough. 
  - though sometimes a little akward, javascript can be used to create a data represenation of a vpl. 
   
\begin{itemize}[ ]
  \item \emph{\mySubRQTwoTitle:} "\mySubRQTwo"
\end{itemize}

[conclude on compilation process]

Answer of "\mySubRQTwoTitle": Show all wasm-considerations: C++ vs Rust vs rewriting in Js. Emscripten vs wasm-bindgen. Conclude with the preference for rust, and explain why.

\begin{itemize}[ ]
  \item \emph{\mySubRQThreeTitle:}  "\mySubRQThree"
\end{itemize}

[conclusion on plugin model]

Answer of "\mySubRQThreeTitle": The implementation details of the plugin system schema. 

\begin{itemize}[ ]
  \item \emph{\mySubRQFourTitle:} "\mySubRQFour"
\end{itemize}

[conclusion on usability]

% X : wasm-based geo-computation applications

\subsection*{Main Question}

\begin{itemize}[ ]
    \item "\myMainRQ"
\end{itemize}

\begin{note}

  So, Does Geofront succeed in "converting existing geocomputation libraries to a sharable VPL format?" 
  
  Yes: 
   - Using the environment, you can take a rust-based geo-computation function or library, 
     and without very many steps, use it within a visual programming environment. 
     The environment can then be used to:
     - Visually debug, 
     - fine tune parameters, 
     - Compare performance to similar libraries,
     - And, unique to this environment, do this all online, in a 'published' format: the full configuration can be shared using a URL. 
  
     The combination of these aspects makes this environment unique. 
  
  - These libraries can be used with a minimum of configurations. Any Rust library with `wasm-bindgen` annotations, in other words, any rust library intended for javascript consumption, automatically works in 'geofront', albeit with some edge-case exceptions. 
  
  
  No: 
   - While it is indeed possible to use and run any rust library with `wasm-bindgen` annotations, in order to properly communicate, visualize, and make data interoperable, special 'config' functions and methods are needed. 
  
   - For now, only 'Rust' and 'JavaScript / TypeScript' can be properly used as libraries. However, most libraries relevant to geo-computation are C++ based. While C++ has excellent support for compiling full, self contained applications to WebAssembly using the 'emscripten' toolset, it lacks rust's level of support in compiling existing libraries. `embind` can be used for this, but compilation using embind is a more tedious, error-prone process.   
     - Additionally, many scientificly oriented C++ libraries like CGAL make extensive use of meta programming and template programming. These ideas do not translate well to an environment outside of C++. 
  
   - the environment uses browser-based calculations, so it cannot be used properly for big data, or other expensive processes.
     Future work: compile the flowchart, run it headless on a server for large datasets.
   
   - the flowchart can only represent linear processes. Many geoprocessing algorithms are iterative and make use of conditionals. These cannot easily be expressed on the canvas. As such, these processes must happen within the context of a function, within a 'computational node'
  
  \end{note}
  
In this study we described the design, creation and evaluation of GeoFront, a web-based point-cloud processing tool.
Overall, the study has succeeded in what it set out to do: designing and implementing a geo-web-vpl. 
Moreover, it has delivered a workflow which can be used to quickly configure existing, native geoprocessing libraries written in C++ or Rust to be consumed and used by said geo-web-vpl.  
However, its usefulness as a fully-featured geo-computation tool is limited.
[lacking support, C++ integration behind, Rust still new, not many geo-types implemented.]

This study concludes that based on these measurements, browser-based geo-computation  is fast enough that it can enable 
many promising use-cases, such as on-demand geodata processing apps, educational demo apps, and code sharing. 
However, extensive user-group testing is required before any definitive statements on accessibility and fitness for geo-computation can be made.  

The full design and implementation are represented by \refchap{chap:methodology} and \refchap{chap:implementation}. 
But to summarize:

[make a final conclusion based upon the 4 sub-answers above]

we split up this study into four sub-studies, each defining, analysing and overcoming a specific challenge of the implementation of a \ac{geo-web-vpl}.

In first sub-study, 

the study explained the challenges and solutions concerned with developing the prototype Web 3D VPL Geofront.
- the challenge of simulating a native interface and experience in a web browser. 
- Especially file system challenges 

In the second sub-study,

challenge: 
compilation to wasm

Observation: 
C++ is difficult to compile to wasm.
This is not due to wasm itself, but rather the focus of emscriptem in compilation of full C++ applications, and the emulation of a POSIX environment in a web browser. 
The library support lacks ergonomics.
The library support of rust is more advanced.

Conclusion: 
Rust is for the forseeable future a better choice for writing platform-independent libraries. 
Suggestion: Emscripten's 'embind' tool to expand to the level of 'wasm-bindgen'.

Third sub-study: 

Challenge:
Two layers of wrapper libraries are needed to take an existing rust / C++ library, and run it in a web-based visual programming language.

This is why this third component of the methodology is focused on mitigating the need for the second wrapper library. 

solutions: 
- automatically base visual components on typescript header-files
- add any 'config' information to the first type of wrapper library. 


- advantages: 
  - The web **is** able to facilitate a visual programming language.
    - does indeed make excellent use of accessibility \& interactivity aspect
  - reasonable performance 
    (- great considering the platform)

- disadvantages: 
  - all in-between data must be stored in memory if it is to be inspected.
    - Can't make use of 'writing files', so that something can be removed from memory 
    - BUT, even when using emscripten, you are still caching all sorts of things

  - The web is able to be used for geoprocessing, albeit with some caveats
    - Less control and precision
    - TypedArrays,
    - Geometric predicates 
    - Rounding
    - ETC.

  - Many of these things can be fixed with webassembly, but webassembly itself has other shortcomings
    - Differences between Rust \& C++

 - Notes:

   - would not be possible without these modern web features
    - Web Assembly 
    - Typed Array's 
    - Web Workers
    - Web Components,
    - 2D Canvas API
    - Web GL

- ability to share is a true enhancement






%%%%%%%%%%%%%%%%%%%%%%%%%%%%%%%%%%%%%%%%%%%%%%%%%%%%%%%%%%%%%%%%%%%%%%%%%%%%%%%
%%%%%%%%%%%%%%%%%%%%%%%%%%%%%%%%%%%%%%%%%%%%%%%%%%%%%%%%%%%%%%%%%%%%%%%%%%%%%%%
%%%%%%%%%%%%%%%%%%%%%%%%%%%%%%%%%%%%%%%%%%%%%%%%%%%%%%%%%%%%%%%%%%%%%%%%%%%%%%%

\section{Contributions}
\label{sec:contribution}

The method contributed:

% CITYJSON VALIDATOR ARGUMENT: THE WEB CAN BE USED TO IMMEDIATELY MAKE SOME TOOL / SOME RESEARCH PROJECT OPERATIONAL 'IN THE REAL WORLD'. THIS WAY, DATA CAN BE GATHERED, USER FEEDBACK CAN BE GATHERED, AND THE TOOL CAN BE EVALUATED IN TERMS OF REPRODUCABILITY. FINALLY, IT OFFERS THE POSSIBILITY OF THE TOOL BEING ACTUALLY USED, IN PRACTICE. 
% z

\begin{itemize}[-]
  \item \textbf{A new implementation of a web-based visual programming language for geocomputation}
    To improve the activity of geocomputation
  \item \textbf{A novel method of publishing an using native libraries on the web}
    To aid fair geocomputation.
  \item \textbf{}

  \item \textbf{}
\end{itemize}


\section{Limitations}
\label{sec:limitations}

The contributions mentioned above to have limitations, which can be described as follows:

\begin{itemize}[-]
  \item \textbf{The catch 21 between C++ and Rust}
  \item \textbf{...}
  \item \textbf{...}
\end{itemize}

\section{Discussion}
\label{sec:discussion}
In this section the value and quality of this research will be discussed. 

\textbf{Biases regarding C++ / Emscripten and Rust}

First of all, In the comparison between C++ and Rust, the studies conducted proved to be unfavorable towards C++. 
it could be that C++ was judged unfairly, due to the authors personal inexperience with the build tooling of the language. 
Many complications were encountered during compilation, leading to extensive editing of makefiles and attempts at recompiling forked subdependencies of CGAL using 'hacky fixes'.  
It is unknown how much of this was due to personal C++ inexperience, inflexibility of the libraries in question, or the shortcomings of the toolchain. 

Despite this, the study still did everything to make the judgement as non-bias as possible.
Preliminary studies were conducted with both languages, and additional C++ courses were followed. 

It could even be the case that this particular study is more fair than a study conducted by authors with more experience with C++, since before the assessment between Rust and C++, approximately the amount of time was spend with both languages. 
If an author was more familiar with one of the two languages, this might have lead to a bias result. 

\textbf{Too sizable in scope}

Additionally, the scope generated by combining geocomputation, web applications and vpls, might have been too extensive. 
This is evident in the number of 'supporting studies' conducted, and the sizable workload of the implementation.
It might have been better to focus the scope of the thesis down to only 'browser-based geocomputation', or 'visual programming and geo-computation', or 'geocomputation using rust', to allow for a more in-depth analysis.

Then again, the core of the contribution of this thesis lies precisely in the attempt to connect these subjects,
especially since prior studies remained by en large closely scoped to their respective domains.
The hypothesis was that synergies exist, and that each separate domain stand to gain much from the ideas and knowledge found in the other ones. 
In order to make this possible, the study had to acquire a scope to explore all in-between synergies and interactions, leading to geo-vpls, web-vpls, and browser-based geocomputation. 
Now that this study has made these connections explicit, future studies can focus on more precise aspects of these cornerstones again.

\textbf{Imbalance between Software implementation and study}

The third 'danger' which remained an ongoing balancing act during the conduction of this study, is the balance between 'performing a study' and 'developing an application'. 
Indeed, many of the aspects discussed throughout this study come down to implementation aspects of the geofront application. 

This is why the study has attempted to generalized its findings as much as possible.
Geofront is regarded as a proxy of geo-web-VPLS in general, in the sense that whatever was encountered during implementation of the application, must be the same for any attempt at creating a web-based vpl for geocomputation.

\textbf{Subjectivity in qualitative assessment}

Lastly, many of the assessments made by this study are qualitative assessment, and as such, might suffer from a high level of subjectivity. 
This is unavoidable in any assessment which does not come down to clear, quantifiable aspects, such as performance, memory usage or precision. 

Nevertheless, the study has attempted to scope this subjectivity by basing its assessments heavily on prior works in the field of vpl, and always showcasing clear examples. 
This has made this study at least as subjective as these prior works. 

\subsection*{Questions and answers}
Questions on the quality of the study, and the answer this study is able to provide.

\subsubsection*{Q: Geodata is big data. Will this web application be scalable to handle big datasets?}

One of the problems to address when considering the ergonomics of geodata geo computation, is the fact that geodata is almost always big data. 
A web application cannot be expected to process huge datasets. 
So how does geofront address this fundamental aspect of geoprocessing? 

First, lets give the devil it's due. 
- Even when processing "smaller" datasets of, lets say 4 GB, most of the 'flowchart niceties' of geofront will cease to be useful. Inspecting this data will take more time than its worth, and reconfiguring the flowchart will take a long time. This can be mitigated by using web workers, but it will still not be very ergonomic to work with. 
- This is why performance is everything within the field of geo-informatics.

\begin{note}
  BUT: 

  - A case can be made for on-demand, browser-based geoprocessing. 
  % - Cloud native -> streaming -> 
  % streaming just what you need, processing just what you need, what you are looking at
  
  - Even when we want to write a tool to deal with large datasets, we often test and develop this tool in a smaller context, with a smaller dataset first. The same thing is possible with geofront: 
  
  - Geofront is mostly meant as a sandboxing tool for experimentation: An environment try out different procedures, parameters, and different datasets. 
  
  - The flowcharts created with geofront are compilable to javascript. this allows any processing operation created with geofront to be executed from the command line using node.js. This is a way of how geofront can integrate with large-scale geodata pipelines. 
  
  The point is that even if we use cloud-based computation, we still want to be able to be able to ergonomically and correctly configure these geoprocesses. Geofront could still assist with that.
  

\end{note}

BUT MOREOVER:

The possibility of client-side geoprocessing also allows for an entirely new geoprocessing workflow, which could replace some use-cases that now require big-data processing and storage. Instead of storing big datasets of pre-processed results, by using client-based, on demand geoprocessing, an application could take a general big-data base layer, and process it on-demand, with a scope and settings determined by the end-users. 

This type of \emph{Process Streaming} is certainly not a drop-in replacement for all big-data use cases. But, in cases which can guarantee a 'local correctness', this should be possible. Examples of this are a delaunay triangulation, TIN interpolation or image filter-based operations. This could be a more cost-effective outcome, as server farms \& Terabytes of storage are time consuming, expensive phenomenon.

\subsubsection*{Q: Why not build everything as a local application, and publish the entire thing as wasm?}

\begin{note}

  TODO convert to prose 

  That would be:
  - more performant (probably)
  - Better native experience
  - Better compilation to standard executable
  
  BUT:
  - The current setup allows for javascript interoperability. 
    - This is useful for the purposes of UI, GUI, Web requests \& Responses, jsons, WebGL.
    - These are all aspects that would have needed to be part of the C++ application, that we now get 'for free', since the implementation of these features are present within the browsers of clients. 
      - webgl, dom as ui, web components, etc.
  - javascript can now also serve as its scripting language, making custom, scriptable components an easy possibility.
  
\end{note}

% - That would be very hard to script with.

\subsubsection*{Q: Where was this 'barrier' the methodology spoke of?}

\begin{note}

  TODO convert to prose

  combination: 
  1. hard to expose existing libraries in a way that is actually nice to use
  - "just compiling to wasm" was not enough.
  - Webassembly is a double edged sword. Interfacing with wasm binaries from javascript is slow: lots of duplication of data. 
  -> this study leaned heavely on Rust to solve this problem
  - catch 21 beyween rust and C++
  
  2. disconnect with regular programming libraries
  - to turn a function into a component usable in a visual programming graph, lots of meta-data is needed. 
    -> leading to config files 
    -> leading to a barrier between regular programming libraries, and vpl libraries. 
    -> this study opted for automated configs based on 'typescipt' headers to attempt to solve this problem. This worked, but had some new Limitations
       -> it did allow for rapid experimentation, and seemless interoperability
       -> however, other aspects like descriptions, library meta info, etc. 
       -> for this a 'config' of some sorts was still needed. 
  
  3. web interface
  - Having no file system really hurts the usefulness of the vpl as a data processing application.
  -> file system API is coming to fix this
  -> This study recommends cloud-native as a solution to this problem, and has added 
  
\end{note}

\subsubsection*{Q: Is a ac{geo-web-vpl} the same as a 3D vpl with existing geoprocessing functionalities attached? In other words, is geoprocessing nothing else than procedural modelling?}

No, but it is a good start. 
A geo-web-vpl is 'at the very least' a 3D vpl with geoprocessing functionalities. 
But, an actual geocomputation vpl might require more features, such as a global CRS, support of base-maps, more control on precision, etc. 
These important aspects must be left for future work.

\subsection{Q: Usage: Who benefits from a web-geo-vpl, and how? }
This study proposes 4 use-cases:
\begin{enumerate}[-]
  \item Educational Sandbox
  \item Web Demo Environment
  \item End-user geoprocessing environment 
  \item Rapid prototyping environment
\end{enumerate}

\subsection*{Q: Is this environment truly accessible?}

Based on the analysis given at \refsec{sec:analyses:utilization}, it is safe to say that based on its features, Geofront is about as accessible as comparable geo-vpls, like geoflow or grasshopper. 
However, this analysis is only based on the achieved functionality and features. 
Actual user-testing is required to assess The true accessibility of the tool.

\subsection*{Q: Is this environment truly a competitor to native / other methods of geoprocessing?}

In theory, yes.
Using the workflow as described, native geocomputation libraries could be written, and these same tools could be used on the web at near native performance. 
Additionally, the web offers enough functionality so that even sizable, local datasets could be processed this way.
In practice, the 'catch 21' problem between Rust and C++ means that in the sort term, this environment will not be used for professional geocomputation.
Additionally, the tool is still in a prototypical state, and will need to be more stable to be used professionally. 


%%%%%%%%%%%%%%%%%%%%%%%%%%%%%%%%%%%%%%%%%%%%%%%%%%%%%%%%%%%%%%%%%%%%%%%%%%%%%%%
%%%%%%%%%%%%%%%%%%%%%%%%%%%%%%%%%%%%%%%%%%%%%%%%%%%%%%%%%%%%%%%%%%%%%%%%%%%%%%%
%%%%%%%%%%%%%%%%%%%%%%%%%%%%%%%%%%%%%%%%%%%%%%%%%%%%%%%%%%%%%%%%%%%%%%%%%%%%%%%

\section{Future work}
\label{sec:future-work}

Here I list a number of interesting topics for follow-up research. 
I find the first two topics most interesting as these are both currently very active and interesting research areas.

\subsection{FAIR geocomputation \& User Testing}

An idea arose during development: if geodata needs to be fair, geocomputation needs to be fair as well. 
However, this aspect sees less study.

  
  
  % Sub component: FAIR: 
  % - FINDABLE:      Hard to find the right tools for the job
  % - ACCESSIBLE:    Hard to access these tools (install, setup environment, look at what you are doing)
  % - INTEROPERABLE: Hard to use two tools from different ecosystems (bindings, plugins, etc). 
  % - REUSABLE:      Hard to re-use a specific scripts written for one use case in another use case

% (NOTE: This is a nice point to make after the thesis: focus more on FAIR geoprocessing)

% The Geoweb, or Geospatial Web, covers a broad collection of topics located at intersection of the field of geo-information and the web. A noteworthy study on the Geoweb is Van den Brink's phd titled "Geospatial Data on the Web". \cite{brink_geospatial_2018}. She claims that even though geodata is vital to a diverse range of applications and people, the ability to properly retrieve geodata remains almost exclusive to experts in the field. This is to the determent of all these applications and people, jeopardizing value, opportunity, and decision making. She makes this argument by using the concept of FAIR geodata. Coined by \cite{mark_d_wilkinson_fair_2016}, The FAIR principles are a collection of four assessment criteria used to judge the usability of (scientific) data: Findable, Accessible, Interoperable, and Reusable. 

% We argue that if these concerns count for geodata \textit{retrieval}, they should just as well count for geodata \textit{processing}. After all, if a user is unable to convert the retrieved geodata to their particular use case, then the information they seek remains inaccessible. Therefore, this study introduces the concept of \emph{FAIR geoprocessing}. 

% Based on the arguments presented by \cite{brink_geospatial_2018}, we can also extrapolate that a \ac{gis} environment shouldn't exclusively be used by only experts. Van den Brink mentions a group called 'data users', presented as "web developers, data journalists etc. who use different kinds of data, including geospatial data, directly to create applications or visualizations that supply information to end users (citizens)". 

% We use both extrapolations to define the users and 'usability' for the context of this study. We will judge the proposed use case application as 'usable', if it is deemed Findable, Accessible, Interoperable, and Reusable. The user group intended to use this environment is defined as both experts in the field of geo-information and this more general group of data users.

\subsection*{Function Silo's \& denichification of GIS}

similarly: function silo's 

/ experiment to assess: 
- The fitness of the web in general for client-side geo-computation
- If new features of modern browsers mean anything for the field of geo-informatics at large 
- The topic of accessible geoprocessing.

now, answer this to the best of your ability

Many considerations

% Premature optimization is the root of all evil | Donald Knuth
% Delay decisions to the latest moments, to gain maximum context,
% Key insight into writing better compilers

While conducting this research, I came across various key insights from various studies, and there seemed to be a link between them 

Most important effort I saw is the "denichification" of the geospatial world.
- Hugo's keynote
- Linda van den Brink's PHD
- cloud-native geospatial 

observation: 
- a lot of attention for FAIR geodata, linking of geodata, using common standards, etc. 

BUT: We don't seem to put as much emphasis on FAIR geodata \emph{Processing}. 
- The means to filter a dataset, to process it or convert it to a required format or CRS, could become more FAIR

We focus on Accessibility
accessibility

\m{->} the fact of being able to be reached or obtained easily:
\textit{Two new roads are being built to increase accessibility to the town centre.}

\m{->} the quality of being easy to understand: 
\textit{The accessibility of her plays means that she is able to reach a wide audience.}

...

- The front-end browser technologies are a vital component of the modern geospatial software.
- Like how the entire cloud-native moment is only possible because of the HTTP range request feature. 


THE MAIN THING I WOULD LIKE YOU TO GET AWAY FROM ALL OF THIS:
- a vital component of the cloud-native geospatial moment is the "HTTP range request" web feature, and chris said as much.
  - this feature has been out for some time ( html1/1, )
- What I'm saying, is that we have a whole range of similar, 'game changing technologies' recently added to web browsers, and I have a feeling these features could be the birthing grounds for new, ground breaking ideas and movements of ideas. 

We have not fully envisioned these new trends, nor do we have a catchy, powerful name such as \emph{Cloud Native Geospatial}, But I have no doubt that something revolutionary will come of this. 

Nevertheless, I will attempt to name and envision a trend from these technologies. \emph{"FAIR Geodata Processing"}.

Vision: 
- Portable, cross-platform, binary geoprocessing libraries, which can be used on the cloud / on servers, natively, and in the browser, without any changes. 
\m{->} we can use that to build standards for geodata processing itself. Every \m{GP} library interoperable with every other library, at least on a language and package manager level.
- This also eliminates the need of platform specific plugins (QGIS plugins, ARCGIS plugins, Blender Plugins, 'web plugins').
- This could lead to a generalized geoprocessing library portal like NPM / cargo / WAPM with an attached content delivery network, Or these infrastructures can just be utilized, with just an UI sprayed on top.

- I am aware that these types of efforts have been attempted many times before, but WebAssembly might be a missing link 

\m{->} webassembly has a good balance between portability and performance.

\m{->}

If I were to attempt to name this trend, I would

\subsection{Using Geofront for CAD or  BIM}

\todo{Sorry for this rant, this will be more nuanced}
% AT THE END OF THE DAY, THERE IS NO REAL DIFFERENCE BETWEEN CAD, BIM and GIS.
% of course, there are many differences, like required precision and tolerances, which types of interfaces and operations are common, and the subject matter it represents. 
% But on a deeper, fundamental level, they are all the same: its just a bunch of 2D / 3D data, representing some real world thing. 

Today, we see the need for collaboration between CAD, BIM and GIS. Entire industries (Speckle, FME) have been introduced to bridge the gaps. 

All three of CAD, BIM and GIS want the ability to join solids together, desire to give certain spatial objects metadata, and want to run automated workflows in the cloud. These Individual fields are constantly reinventing features the other fields have already figured out. BIM is starting to open up to the idea of streaming only a part of the building instead of the whole thing, something which the GIS world has been doing for years. On the other hand, GIS is only now starting to make the transition to 3D, a transition not unlike to how BIM is replacing 2D CAD in the AEC industry. 

We will allow geofront to be fully customized by different plugins. By unloading all GIS plugins, and adding all BIM plugins, we turn GeoFront from a GIS to a BIM tool.


%%%%%%%%%%%%%%%%%%%%%%%%%%%%%%%%%%%%%%%%%%%%%%%%%%%%%%%%%%%%%%%%%%%%%%%%%%%%%%%
%%%%%%%%%%%%%%%%%%%%%%%%%%%%%%%%%%%%%%%%%%%%%%%%%%%%%%%%%%%%%%%%%%%%%%%%%%%%%%%
%%%%%%%%%%%%%%%%%%%%%%%%%%%%%%%%%%%%%%%%%%%%%%%%%%%%%%%%%%%%%%%%%%%%%%%%%%%%%%%

\section{Reflection}
\label{sec:reflection}

This section provides a discussion on the reproducibility of the developed methods and obtained results in this thesis, and includes an evaluation of the thesis process from a personal point of view.



\section{Personal Reflection}


One of the goals of this study was to investigate and explore browser-based geo-computation, and there are many ways of conducting exploitive studies. 
This study chose for a practical approach: investigation by means of creating an application.
The advantage of this approach is that it leads to tangible results. 

The disadvantage of this approach is that software development can lead a study astray, if the development needs of the application are put before the needs of the study itself.

This study was a battleground between "performing a study on if geocomputation benefits from the web \& a vpl", and "lets build an open-source tool to aid geocomputation"

It needed a bit of both.




\begin{note}
  
  Reflection
  
  "It is not the task of the University to offer what society asks for, but to give what society needs" ~ Edsger W. Dijkstra
  
  Despite the cliche of quoting Dijkstra, and despite the arrogant danger of pretending to know what people want better than they themselves know what they want, 
  
  
  'what the world wants is more react, angular, or vue developers'
  'what the world needs, in my opinion, is guidance on this front of web vs native application development.'
  
  right now, we are sucked in the rabbit hole of 'make everything web-based, then maybe try to execute that natively',
  And I think that we should start working towards a situation of: 'make everything natively, KNOWING that publishing it on the web is a piece of cake'.
  
  
  
  The thesis represents to me a bold "What if" scenario: 
  - What if geodata computation was an elegant, ergonomic process?
  - What if more people could more easily perform geospatial computations?
  - What if textual and visual programming languages worked complimentary?    
  - What if geocomputation libraries where written in Rust instead of C/C++?
  - 
  
  'we are not inventive enough with the tools at our disposal'
  'outdated web vs native, client vs server distinctions block our vision from seeing new types of applications'
  'we could be doing so much more with what we have.'
  
  
  A huge leap 
  
  
  \end{note}
  


  \subsection{Personal Motivation}
  During my internship I was tasked with creating a parametric 3D CAD model. 
  
  - local usage 
    -> quick, direct feedback
  
  - We needed to make this a product for end users. 
  
  - Industry-standard choice: cloud 
    -> smart-server dumb-client setup, cloud-native architecture 
  
  - Problems
    -> continuously downloading new resulting CAD files after every change created a lot of web traffic. 
    -> slow, not at all the same experience.
    -> cloud host was even more slow in cold-start scenario's   
    -> cloud host monetization scheme: pay for every time the script runs, 
       -> meant that consumers had to be discouraged to 'play around' with the tool too much. 
    
  This made me question the cloud-based paradigm, at least for the use case of calculating geometry by end users. 
  
  At the same time, many of our parametric designers could use grasshopper, and only grasshopper. 
  
  This led me to think about a vpl which can run client-side in a browser, and can produce client-side applications.
  
  if the data is geodata or CAD data, does not matter besides the fact that geodata is often big data.
  
  

  \textbf{Motivation 2: Accessible Geoprocessing Libraries}
  
  Most industry-standard geoprocessing libraries such as CGAL are difficult to use by anyone but experts in the field. A steep learning curve combined with relatively complex installation procedures hinders quick experimentation, demonstration, and widespread utilization of these powerful tools. It also limits the interdisciplinary exchange of knowledge, and compromises the return of investment the general public may expect of publicly funded research.
  
  Geofront could improve the accessibility of existing geodata processing and analysis libraries, without adding major changes to those tools, by loading webassembly-compiled versions of them, similar to [other web demo's](todo).

\subsection{Reproducibility}

Results themselves are insanely reproducable.
Software can be used, you can reproduce results easely by dumping versions of geofront in 
a folder.

\dots

Software can also be build without too many difficulties, but the procedure has some unconventional build steps: 

\dots

BUT, the code is not the cleanest, nor the most conventional. minus points on open-source accessibility.

%%%%%%%%%%%%%%%%%%%%%%%%%%%%%%%%%%%%%%%%%%%%%%%%%%%%%%%%%%%%%%%%%%%%%%%%%%%%%%%
%%%%%%%%%%%%%%%%%%%%%%%%%%%%%%%%%%%%%%%%%%%%%%%%%%%%%%%%%%%%%%%%%%%%%%%%%%%%%%%
%%%%%%%%%%%%%%%%%%%%%%%%%%%%%%%%%%%%%%%%%%%%%%%%%%%%%%%%%%%%%%%%%%%%%%%%%%%%%%%

\section{Self Assessment}

...

