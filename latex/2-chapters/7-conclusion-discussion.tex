% repeat results and answers in shortened form
\chapter{Conclusion \& Discussion}
\label{chap:conclusion}

% In this study we described the design, creation and evaluation of GeoFront, a web-based point-cloud processing tool.
% Overall, the study has succeeded in what it set out to do: designing and implementing a geo-web-vpl. 
% Moreover, it has delivered a workflow which can be used to quickly configure existing, native geoprocessing libraries written in C++ or Rust to be consumed and used by said geo-web-vpl.  

% This study concludes that based on these measurements, browser-based geo-computation is fast enough that it can enable 
% many promising use-cases, such as on-demand geodata processing apps, educational demo apps, and code sharing. 
% However, extensive user-group testing is required before any definitive statements on accessibility and fitness for geo-computation can be made.  

This chapter consists of the answers to the study.
questions that were given in the first chapter (\refsec{sec:conclusion}), 
a summary of the most significant contributions (\refsec{sec:contribution}) and the limitations of these contributions (\refsec{sec:limitations}), 
a discussion of the value and quality of this study (\refsec{sec:discussion}),
a number of theorized implications of this study (\refsec{sec:future-work}),
and lastly, a self reflection (\refsec{sec:reflection}).

\begin{note}
  Important: 
  - Show what I have learned
  - Show maturity
  - Recommendations 
  - Show balance
  - Show finality
\end{note}

%%%%%%%%%%%%%%%%%%%%%%%%%%%%%%%%%%%%%%%%%%%%%%%%%%%%%%%%%%%%%%%%%%%%%%%%%%%%%%%
%%%%%%%%%%%%%%%%%%%%%%%%%%%%%%%%%%%%%%%%%%%%%%%%%%%%%%%%%%%%%%%%%%%%%%%%%%%%%%%
%%%%%%%%%%%%%%%%%%%%%%%%%%%%%%%%%%%%%%%%%%%%%%%%%%%%%%%%%%%%%%%%%%%%%%%%%%%%%%%

\section{Conclusion}
\label{sec:conclusion}

This paragraph answers the research questions. 
It starts with answering the sub research questions and concludes with the answer to the main research question

\subsection*{Sub Questions}

\begin{itemize}[ ]
  \item 1. "\mySubRQOne"
\end{itemize}

The browser appears to be capable of representing a dataflow-type vpl graph to an acceptable degree, 
based on the implementation presented in \refsec{sec:implementation:representation}, and the analysis in \refsec{sec:analyses:representation}.
The browsers biggest advantage for an application like this is the sheer amount of features a javascript program can use by default, like the 2D canvas api, the DOM, and WebGL. 
These features do not need to be included within the source code of the application, leading to quick load times. 
All three of these features proved to be vital, and were performant enough to support an application like this. 
Only the 2D canvas Api can become slow when rendering a great number of components. 

JavaScript can also be used to represent the data structure and logic needed to make a VPL functional. 
Javascript's flexibility proved to be crucial to support features like dynamically loading and using libraries at runtime. 
The disadvantage was due to JavaScript's very limited type support, and limited precision in general. 
While the language does offers powerful options for reflection, this cannot truly be used if javascript itself makes no distinction between different number types (\m{int, float, double}) for example. 
% The prototype-based objects also led to type-check problems. 
Typescript has been used to mitigate some of these shortcomings.  

\begin{itemize}[ ]
  \item 2. "\mySubRQTwo"
\end{itemize}

% To what extend can geocomputation libraries written in system-level languages be compiled
% for web consumption?

Based on the experiments and analysis in \refsec{sec:analyses:compilation}, the study concludes that most contemporary, C++-based geocomputation libraries cannot be sufficiently compiled for web consumption, at least not for the purposes of loading the functionalities within a web-VPL.    
This is not due to wasm itself, but rather the focus of the emscripten compiler.
The tool can be used to compile full-scale C++ applications, and even includes an emulation of a POSIX environment.
However, it severely lacks support for compiling libraries themselves, compared to other wasm-library compilers.
libraries generated with emscripten's 'embind' tool severely lack ergonomics, and thus would not be directly loadable in a geo-web-vpl.
While web-implementations do exist like 'GDAL-js', these solutions are required to work though Web Workers, and use the emscripten virtual file systems, which again compromises their usage for the purpose of a dataflow-type vpl using pure functions.
Additionally, many scientifically oriented C++ libraries like CGAL make extensive use of meta programming and template programming, paradigms which do not translate well to an environment outside of C++. 
Finally, the study was able to recognize some discrepancies between the novelty of the WebAssembly format, juxtaposed to 50 year legacy of the C++ language.

Despite all of this, the study was able to provide a solution to these compilation shortcomings by expanding the range of 'system-level languages' beyond C++. 
The Rust programming language offers a performance and level of control similar to C++, and has better wasm-library support thanks to the \m{wasm-bindgen} toolkit. 
Using this toolkit, the study could successfully expose a native geocomputation library in a manner properly consumable by a web-vpl.
regrettably, not many rust-based geocomputation libraries are written in pure rust, and the general pool of existing geocomputation libraries is limited due to the novelty of the language. 

Thus, the conclusion is a 'catch 21' of some sorts.
Rust is for the forseeable future a better choice for writing easily consumable, platform-independent libraries, but does not have a 30+ year legacy of existing geocomputation libraries. 
On the other hand, C++ does have these libraries, but lacks proper wasm-library compilation options.

To overcome this, the study suggests that either the 'embind' tool must be expanded to the level of functionality of  'wasm-bindgen', or geocomputation libraries must be rewritten in Rust. 

\begin{itemize}[ ]
  \item 3. "\mySubRQThree"
\end{itemize}

Based on the method described in \refsec{sec:method-three} and \refsec{sec:implementation:loading}, and the analysis of \refsec{sec:analyses:loading}, it can be concluded that it is possible and even sufficiently usable to load a web-library into a VPL without explicit configuration. 
It also had the desired effect of breaking down the barrier between vpl libraries and regular text-based libraries: Using this method, only one type of library is needed to serve both. 
Moreover, it led to a workflow in which rapid experimentation was possible, since this method allows users to develop a library locally, and then quickly experiment and test its usage online. 

The drawback of allowing this seamless interoperability and rapid experimentation, is that many important properties like descriptions and library metadata do not need to be explicitly specified, and could not be automatically extracted. 
These properties still had to be added to the libraries in the shape of methods with a recognizable naming convention.

Additionally, the freedom of granted by not restricting input and output types can lead to a confusing user experience, since there is no way of restricting libraries to use particular type convention.
Even worse, the libraries could use references pointing to the same object, eliminating the 'immutable, no side effects' nature of a dataflow-type VPL.

\begin{itemize}[ ]
  \item 4. "\mySubRQFour"
\end{itemize}
% To what extend can a 'geo-web-vpl' be \textbf{used} to create geodata pipelines?

Based on the analysis of Geofront in \refsec{sec:analyses:utilization}, it can be concluded that a geo-web-VPL can be used for geocomputation to a sufficient extend.  
The analysis shows that many of Geofront's best aspects for the purpose of geocomputation are a consequence of the design decision to use a diagram-based, dataflow-type VPL.
Examples of these are how the Functional programming paradigm leads to pure functions and immutable variables, making the graph as a whole behave in a predicable manner, allowing for the inspection of in-between data at runtime. 
However, the openness of the plugin system inhibits the consistency of these functional aspects.
Imported libraries are not forced to exclusively use pure function. 
As a consequence, libraries can create functions with many side effects, or they can use inconsistent input and output datatypes, ultimately leading to confusion for the end-user.

\subsection*{Main Question}

\begin{itemize}[ ]
    \item "\myMainRQ"
\end{itemize}

% How can a VPL be used to support and execute existing geo-computation libraries in a browser?
% So, Does Geofront succeed in "converting existing geocomputation libraries to a sharable VPL format?" 

A VPL can support existing geocomputation libraries if and only if these libraries are able to be \emph{compiled}, \emph{loaded}, \emph{represented}, and \emph{utilized} in a VPL format.

Using a new javascript implementation of an acyclic, graph-based VPL, the study was able to demonstrate how the web platform can be used to \emph{represent} a VPL capable of constructing scripts from these libraries.
The dataflow-properties of a graph-based VPL like this also makes this libraries sufficiently \emph{usable}, albeit with some well-known caveats of dataflow-VPLS, like the representation of conditionals and iteration. 

The current methods of \emph{compiling} existing C++ geocomputation libraries to the web turned out to be insufficient for the purposes of this study.  
This is due to emscripten's focus on compiling full C++ applications instead of libraries.
Despite this, the study wás able to demonstrate how a novel method can be used to sufficiently \emph{compile} and \emph{load} a Rust-library for usage in the VPL.
While not many contemporary geocomputation libraries are written in Rust, the study offers this method to either offer emscripten contributors a blueprint of a desired workflow, or to offer geocomputation library contributors a powerful use-case for the Rust language. 

%%%%%%%%%%%%%%%%%%%%%%%%%%%%%%%%%%%%%%%%%%%%%%%%%%%%%%%%%%%%%%%%%%%%%%%%%%%%%%%
%%%%%%%%%%%%%%%%%%%%%%%%%%%%%%%%%%%%%%%%%%%%%%%%%%%%%%%%%%%%%%%%%%%%%%%%%%%%%%%
%%%%%%%%%%%%%%%%%%%%%%%%%%%%%%%%%%%%%%%%%%%%%%%%%%%%%%%%%%%%%%%%%%%%%%%%%%%%%%%
%%%%%%%%%%%%%%%%%%%%%%%%%%%%%%%%%%%%%%%%%%%%%%%%%%%%%%%%%%%%%%%%%%%%%%%%%%%%%%%
%%%%%%%%%%%%%%%%%%%%%%%%%%%%%%%%%%%%%%%%%%%%%%%%%%%%%%%%%%%%%%%%%%%%%%%%%%%%%%%
%%%%%%%%%%%%%%%%%%%%%%%%%%%%%%%%%%%%%%%%%%%%%%%%%%%%%%%%%%%%%%%%%%%%%%%%%%%%%%%
%%%%%%%%%%%%%%%%%%%%%%%%%%%%%%%%%%%%%%%%%%%%%%%%%%%%%%%%%%%%%%%%%%%%%%%%%%%%%%%
%%%%%%%%%%%%%%%%%%%%%%%%%%%%%%%%%%%%%%%%%%%%%%%%%%%%%%%%%%%%%%%%%%%%%%%%%%%%%%%
%%%%%%%%%%%%%%%%%%%%%%%%%%%%%%%%%%%%%%%%%%%%%%%%%%%%%%%%%%%%%%%%%%%%%%%%%%%%%%%

\section{Contributions}
\label{sec:contribution}
The full extend of the results of this study is represented by this section, as well as \refsec{sec:limitations}.

\begin{itemize}[-]
  \item \textbf{A new implementation of a web-based visual programming language for geocomputation}
    By providing the full source code of the application and all libraries used within the application, together will all implementation details given in \refchap{chap:implementation}, this study aims to provide guidance for all subsequent studies on the topic of VPLs, geocomputation, or geoweb applications. 
    Additionally, by having conducted this study on the intersection of all three of those fields, the study aims to show 
  
  \item \textbf{A novel method of publishing an using native libraries on the web}
    - Using the environment, you can take a rust-based geo-computation function or library, 
    and without very many steps, use it within a visual programming environment. 
    The environment can then be used to:
    - Visually debug, 
    - fine tune parameters, 
    - Compare performance to similar libraries,
    - These libraries can be used with a minimum of configurations. Any Rust library with `wasm-bindgen` annotations, in other words, any rust library intended for javascript consumption, automatically works in 'geofront', albeit with some edge-case exceptions. 
    - And, unique to this environment, do this all online, in a 'published' format: the full configuration can be shared using a URL.
    The combination of these aspects makes this environment unique. 
    
    % CITYJSON VALIDATOR ARGUMENT: THE WEB CAN BE USED TO IMMEDIATELY MAKE SOME TOOL / SOME RESEARCH PROJECT OPERATIONAL 'IN THE REAL WORLD'. THIS WAY, DATA CAN BE GATHERED, USER FEEDBACK CAN BE GATHERED, AND THE TOOL CAN BE EVALUATED IN TERMS OF REPRODUCABILITY. FINALLY, IT OFFERS THE POSSIBILITY OF THE TOOL BEING ACTUALLY USED, IN PRACTICE. 
% z
    
\end{itemize}

\section{Limitations}
\label{sec:limitations}

The contributions mentioned above to have limitations, which can be described as follows:

\begin{itemize}[-]
  \item \textbf{Only Rust, Js \& Ts library support}
For now, only 'Rust' and 'JavaScript / TypeScript' can be properly used as libraries. However, most libraries relevant to geo-computation are C++ based. While C++ has excellent support for compiling full, self contained applications to WebAssembly using the 'emscripten' toolset, it lacks rust's level of support in compiling existing libraries.

    In \refsec{sec:analyses:compilation}, The issues between compiling C++ and Rust libraries were given. 
    Since a stable method of using C++ can not be provided for at the current moment,
  \item \textbf{In practice, not all libraries can be used }
  - While it is indeed possible to use and run any rust library with `wasm-bindgen` annotations or any js + ts library, in order to properly communicate, visualize, and make data interoperable, special 'config' functions and methods are needed. 
  \item \textbf{Only small-scale geodata is possible}
    The 'near native performance' has some caveats, as explained in \refsec{sec:analyses:compilation}, but is not a problem at large. 
    What is a problem is the fact that geodata is in general large.  
    - the environment uses browser-based calculations, so it cannot be used properly for big data, or other expensive processes.
    Future work: compile the flowchart, run it headless on a server for large datasets.
  
  \item \textbf{Implementation shortcomings} 
    Its still a prototype, and has many usability shortcomings, explained in \refsec{sec:analyses:utilization}.
    In particular, many geocomputation-specific aspects are missing. 
\end{itemize}

\section{Discussion}
\label{sec:discussion}
Questions on the quality of the study, and the answer this study is able to provide.

\subsubsection*{Q: Geodata is big data. Will this web application be scalable to handle big datasets?}

% 3. web interface
% - Having no file system really hurts the usefulness of the vpl as a data processing application.
% -> file system API is coming to fix this
% -> This study recommends cloud-native as a solution to this problem, and has added 

First of all, the purpose of this study was only to get geocomputation libraries to the web, and inside of a vpl format. 
Scaling the application up to handle big data was not part of this study, and had to be left to future work.  

One of the problems to address when considering the ergonomics of geodata geo computation, is the fact that geodata is almost always big data. 
A web application cannot be expected to process huge datasets. 
So how does geofront address this fundamental aspect of geoprocessing? 

First, lets give the devil it's due. 
- Even when processing "smaller" datasets of, lets say 4 GB, most of the 'flowchart niceties' of geofront will cease to be useful. Inspecting this data will take more time than its worth, and reconfiguring the flowchart will take a long time. This can be mitigated by using web workers, but it will still not be very ergonomic to work with. 
- This is why performance is everything within the field of geo-informatics.


\begin{note}
  BUT: 

  - A case can be made for on-demand, browser-based geoprocessing. 
  % - Cloud native -> streaming -> 
  % streaming just what you need, processing just what you need, what you are looking at
  
  - Even when we want to write a tool to deal with large datasets, we often test and develop this tool in a smaller context, with a smaller dataset first. The same thing is possible with geofront: 
  
  - Geofront is mostly meant as a sandboxing tool for experimentation: An environment try out different procedures, parameters, and different datasets. 
  
  - The flowcharts created with geofront are compilable to javascript. this allows any processing operation created with geofront to be executed from the command line using node.js. This is a way of how geofront can integrate with large-scale geodata pipelines. 
  
  The point is that even if we use cloud-based computation, we still want to be able to be able to ergonomically and correctly configure these geoprocesses. Geofront could still assist with that.
  

\end{note}

\subsubsection*{Q: Why not build everything as a local application, and publish the entire thing as wasm?}

\begin{note}

  TODO convert to prose 

  That would be:
  - more performant (probably)
  - Better native experience
  - Better compilation to standard executable
  
  BUT:
  - The current setup allows for javascript interoperability. 
    - This is useful for the purposes of UI, GUI, Web requests \& Responses, jsons, WebGL.
    - These are all aspects that would have needed to be part of the C++ application, that we now get 'for free', since the implementation of these features are present within the browsers of clients. 
      - webgl, dom as ui, web components, etc.
  - javascript can now also serve as its scripting language, making custom, scriptable components an easy possibility.
  
\end{note}

% - That would be very hard to script with.

\subsubsection*{Q: Where was this 'barrier' the methodology spoke of?}

\begin{note}

  This is repetition of the introduction
  TODO convert to prose

  combination: 
  1. hard to expose existing libraries in a way that is actually nice to use
  - "just compiling to wasm" was not enough.
  - Webassembly is a double edged sword. Interfacing with wasm binaries from javascript is slow: lots of duplication of data. 
  -> this study leaned heavely on Rust to solve this problem
  - catch 21 beyween rust and C++
  
  2. disconnect with regular programming libraries
  - to turn a function into a component usable in a visual programming graph, lots of meta-data is needed. 
    -> leading to config files 
    -> leading to a barrier between regular programming libraries, and vpl libraries. 
    -> this study opted for automated configs based on 'typescipt' headers to attempt to solve this problem. This worked, but had some new Limitations
       -> it did allow for rapid experimentation, and seemless interoperability
       -> however, other aspects like descriptions, library meta info, etc. 
       -> for this a 'config' of some sorts was still needed. 
  
  3. web interface
  - Having no file system really hurts the usefulness of the vpl as a data processing application.
  -> file system API is coming to fix this
  -> This study recommends cloud-native as a solution to this problem, and has added 
  
\end{note}

\subsubsection*{Q: Is a ac{geo-web-vpl} the same as a 3D vpl with existing geoprocessing functionalities attached? In other words, is geoprocessing nothing else than procedural modelling?}

No, but it is a good start. 
A geo-web-vpl is 'at the very least' a 3D vpl with geoprocessing functionalities. 
But, an actual geocomputation vpl might require more features, such as a global CRS, support of base-maps, more control on precision, etc. 
These important aspects must be left for future work.

\subsection*{Q: Usage: Who benefits from a web-geo-vpl, and how? }
This study proposes 4 use-cases:
\begin{enumerate}[-]
  \item Educational Sandbox
  \item Web Demo Environment
  \item End-user geoprocessing environment 
  \item Rapid prototyping environment
\end{enumerate}

\subsection*{Q: Is this environment truly accessible?}

Based on the analysis given at \refsec{sec:analyses:utilization}, it is safe to say that based on its features, Geofront is about as accessible as comparable geo-vpls, like geoflow or grasshopper. 
However, this analysis is only based on the achieved functionality and features. 
Actual user-testing is required to assess The true accessibility of the tool.

\subsection*{Q: Is this environment truly a competitor to native / other methods of geoprocessing?}

In theory, yes.
Using the workflow as described, native geocomputation libraries could be written, and these same tools could be used on the web at near native performance. 
Additionally, the web offers enough functionality so that even sizable, local datasets could be processed this way.
In practice, the 'catch 21' problem between Rust and C++ means that in the sort term, this environment will not be used for professional geocomputation.
Additionally, the tool is still in a prototypical state, and will need to be more stable to be used professionally. 

\subsection*{Reproducibility}
This section provides a discussion on the reproducibility of the developed methods and obtained results in this study.


%%%%%%%%%%%%%%%%%%%%%%%%%%%%%%%%%%%%%%%%%%%%%%%%%%%%%%%%%%%%%%%%%%%%%%%%%%%%%%%
%%%%%%%%%%%%%%%%%%%%%%%%%%%%%%%%%%%%%%%%%%%%%%%%%%%%%%%%%%%%%%%%%%%%%%%%%%%%%%%
%%%%%%%%%%%%%%%%%%%%%%%%%%%%%%%%%%%%%%%%%%%%%%%%%%%%%%%%%%%%%%%%%%%%%%%%%%%%%%%

\section{Future work}
\label{sec:future-work}
The many fields this study draws from mean that a great variety of auxiliary aspects were discovered during the execution ot the study. 
Some of these aspects are listed here, and could lead to interesting topics for follow-up research. 

\subsection{Deployment \& Scalability}
An earlier, very simple version of the Geofront script had the ability to be compiled to normal javascript (see (\reffig{fig:early-geofront-compile-to-js})).  
All libraries were converted to normal import statement, all nodes were replaced by function calls, and the cables substituted by functions. 
This way, the application could be run headless (without the \ac{gui}) either the browser or on a server, using a "Node,js" (Source) or Deno interpreter (Source).

A future study could re-implement this feature, opening up the possibility for deployment and scalability: 
Scripts created with geofront could then deployed as web applications of themselves, or as a web processing services (WPS, SOURCE).
Also, by running this script on a server, and ideally a server containing the geodata required in the process, one could deploy and run a Geofront scripts on a massive scale. 

The overall purpose of this would be to create a free and open source alternative to similar tools like the Google Earth Engine, and FME cloud compute. 

\subsection{Streamed, on demand geocomputation}

The study showed that browser-based geocomputation using Rust is entirely possible. 
This possibility might allow for an entirely new type of geoprocessing workflow, which could replace some use-cases that now require big-data processing and storage.
Instead of storing the sizable, pre-processed results of some geocomputation, an application could take a raw dataset base layer, and process it on-demand in a browser.

This would have several advantages. 
First, end-users can specify the scope and parameters of this process, making the data immediately fine-tuned to the specific needs of this user. 
Secondly, this could be a more cost-effective method, as cloud computation \& Terabytes of storage are time consuming and expensive phenomena.

This type of \emph{On demand geocomputation} is certainly not a drop-in replacement for all use cases. 
But, in situations which can guarantee a 'local correctness', and if the scope asked by the used is not too large, this should be possible. 
Examples of this would be a streamed delaunay triangulation, TIN interpolation or color-band arithmetic. 

\subsection{Rust-based geocomputation \& Cloud Native}
An interesting aspect this study was able to touch on is using Rust for geocomputation.
The reason for this was the extensive support for webassembly, which was crucial for browser-based geocomputation. 
However, there are two additional reasons one might want to perform geocomputation within Rust.
One is that rust is widely considered as a more stable, less runtime error-prone language than C++, while offering similar performance and features. 
The second one is that Rust compiled to WebAssembly sees extensive use on both cloud and edge servers.  
This could be very interesting to the "cloud-native geospatial" movement. 
This web GIS movement aims to create the tools necessary to send geocomputation to servers, rather than sending geodata to the places where they are processed.
To do this, geocomputation must become much more interoperable, more exchangeable, and Rust compiled to WebAssembly forms an excellent candidate. 

Therefore, studying Rust-based geocomputation for cloud native and edge computing, would be an excellent topic for subsequent research. 

\subsection{FAIR geocomputation}
% The study has only scratched the surface on what is possible with combining geocomputation with the fields of Visual programming, and web applications. 
The introduction theorized on how both VPLS and web-apps could be used to make geocomputation less cumbersome.
The study chose to pursuit this on a practical, technical level. 

However, a more theoretical study could also be performed. 
It turns out that these ideas of 'less cumbersome geodata processing', have something in common with many of the geoweb studies on data accessibility and usability (\cite{brink_geospatial_2018}).
The ideas of 'data silo's', 'FAIR geodata', and 'denichifying of \ac{gis} data' (see \cite{brink_geospatial_2018}) map well to geocomputation:
Functionality Silo's, FAIR geocomputation, denichifying of \ac{gis} computation. 

Therefore, an interesting question for a subsequent study could be: "How could geocomputation become more Findable, Accessible, Interoperable, and Reusable?", or "How to integrate the function-silo's of GIS, BIM \& CAD?"
By focussing on data processing actions rather than the data itself, we could shed a new light on why data discrepancies and inaccessibility exist. 
After all, if a user is unable to convert retrieved geodata to their particular use case, then the information they seek remains inaccessible.

% \subsection{Hybrid geocomputation}

% Lastly, 


  
  % Sub component: FAIR: 
  % - FINDABLE:      Hard to find the right tools for the job
  % - ACCESSIBLE:    Hard to access these tools (install, setup environment, look at what you are doing)
  % - INTEROPERABLE: Hard to use two tools from different ecosystems (bindings, plugins, etc). 
  % - REUSABLE:      Hard to re-use a specific scripts written for one use case in another use case

% (NOTE: This is a nice point to make after the thesis: focus more on FAIR geoprocessing)

% The Geoweb, or Geospatial Web, covers a broad collection of topics located at intersection of the field of geo-information and the web. A noteworthy study on the Geoweb is Van den Brink's phd titled "Geospatial Data on the Web". \cite{brink_geospatial_2018}. She claims that even though geodata is vital to a diverse range of applications and people, the ability to properly retrieve geodata remains almost exclusive to experts in the field. This is to the determent of all these applications and people, jeopardizing value, opportunity, and decision making. She makes this argument by using the concept of FAIR geodata. Coined by \cite{mark_d_wilkinson_fair_2016}, The FAIR principles are a collection of four assessment criteria used to judge the usability of (scientific) data: Findable, Accessible, Interoperable, and Reusable. 

% We argue that if these concerns count for geodata \textit{retrieval}, they should just as well count for geodata \textit{processing}. After all, if a user is unable to convert the retrieved geodata to their particular use case, then the information they seek remains inaccessible. Therefore, this study introduces the concept of \emph{FAIR geoprocessing}. 

% Based on the arguments presented by \cite{brink_geospatial_2018}, we can also extrapolate that a \ac{gis} environment shouldn't exclusively be used by only experts. Van den Brink mentions a group called 'data users', presented as "web developers, data journalists etc. who use different kinds of data, including geospatial data, directly to create applications or visualizations that supply information to end users (citizens)". 

% We use both extrapolations to define the users and 'usability' for the context of this study. We will judge the proposed use case application as 'usable', if it is deemed Findable, Accessible, Interoperable, and Reusable. The user group intended to use this environment is defined as both experts in the field of geo-information and this more general group of data users.

% Function Silo's \& denichification of GIS

% similarly: function silo's 

% / experiment to assess: 
% - The fitness of the web in general for client-side geo-computation
% - If new features of modern browsers mean anything for the field of geo-informatics at large 
% - The topic of accessible geoprocessing.

% now, answer this to the best of your ability

% Many considerations

% Premature optimization is the root of all evil | Donald Knuth
% Delay decisions to the latest moments, to gain maximum context,
% Key insight into writing better compilers

% While conducting this research, I came across various key insights from various studies, and there seemed to be a link between them 

% Most important effort I saw is the "denichification" of the geospatial world.
% - Hugo's keynote
% - Linda van den Brink's PHD
% - cloud-native geospatial 

% We focus on Access

% \m{->} the fact of being able to be reached or obtained easily:
% \textit{Two new roads are being built to increase accessibility to the town centre.}

% \m{->} the quality of being easy to understand: 
% \textit{The accessibility of her plays means that she is able to reach a wide audience.}


%%%%%%%%%%%%%%%%%%%%%%%%%%%%%%%%%%%%%%%%%%%%%%%%%%%%%%%%%%%%%%%%%%%%%%%%%%%%%%%
%%%%%%%%%%%%%%%%%%%%%%%%%%%%%%%%%%%%%%%%%%%%%%%%%%%%%%%%%%%%%%%%%%%%%%%%%%%%%%%
%%%%%%%%%%%%%%%%%%%%%%%%%%%%%%%%%%%%%%%%%%%%%%%%%%%%%%%%%%%%%%%%%%%%%%%%%%%%%%%


% \subsection{Cloud Native}
% Lastly, the 

% - The front-end browser technologies are a vital component of the modern geospatial software.
% - Like how the entire cloud-native moment is only possible because of the HTTP range request feature. 

% - a vital component of the cloud-native geospatial moment is the "HTTP range request" web feature, and chris said as much.
%   - this feature has been out for some time ( html1/1, )
% - What I'm saying, is that we have a whole range of similar, 'game changing technologies' recently added to web browsers, and I have a feeling these features could be the birthing grounds for new, ground breaking ideas and movements of ideas. 

% We have not fully envisioned these new trends, nor do we have a catchy, powerful name such as \emph{Cloud Native Geospatial}, But I have no doubt that something revolutionary will come of this. 

% Nevertheless, I will attempt to name and envision a trend from these technologies. \emph{"FAIR Geodata Processing"}.

% Vision: 
% - Portable, cross-platform, binary geoprocessing libraries, which can be used on the cloud / on servers, natively, and in the browser, without any changes. 
% \m{->} we can use that to build standards for geodata processing itself. Every \m{GP} library interoperable with every other library, at least on a language and package manager level.
% - This also eliminates the need of platform specific plugins (QGIS plugins, ARCGIS plugins, Blender Plugins, 'web plugins').
% - This could lead to a generalized geoprocessing library portal like NPM / cargo / WAPM with an attached content delivery network, Or these infrastructures can just be utilized, with just an UI sprayed on top.

% - I am aware that these types of efforts have been attempted many times before, but WebAssembly might be a missing link 

% \m{->} webassembly has a good balance between portability and performance.

% \m{->}

% If I were to attempt to name this trend, I would


\section{Reflection}
\label{sec:reflection}

Here I reflect on possible shortcomings of the thesis in terms of value and quality, and how I have attempted to address these shortcomings. 

\textbf{Biases regarding C++ / Emscripten and Rust}

First of all, In the comparison between C++ and Rust, the studies conducted proved to be unfavorable towards C++. 
it could be that C++ was judged unfairly, due to the authors personal inexperience with the build tooling of the language. 
Many complications were encountered during compilation, leading to extensive editing of makefiles and attempts at recompiling forked subdependencies of CGAL using 'hacky fixes'.  
It is unknown how much of this was due to personal C++ inexperience, inflexibility of the libraries in question, or the shortcomings of the toolchain. 

Despite this, the study still did everything to make the judgement as non-bias as possible.
Preliminary studies were conducted with both languages, and additional C++ courses were followed. 

It could even be the case that this particular study is more fair than a study conducted by authors with more experience with C++, since before the assessment between Rust and C++, approximately the amount of time was spend with both languages. 
If an author was more familiar with one of the two languages, this might have lead to a bias result. 

\textbf{Scope too large}

Additionally, the scope generated by combining geocomputation, web applications and vpls, might have been too extensive. 
This is evident in the number of 'supporting studies' conducted, and the sizable workload of the implementation.
It might have been better to focus the scope of the thesis down to only 'browser-based geocomputation', or 'visual programming and geo-computation', or 'geocomputation using rust', to allow for a more in-depth analysis.

Then again, the core of the contribution of this thesis lies precisely in the attempt to connect these subjects,
especially since prior studies remained by en large closely scoped to their respective domains.
The hypothesis was that synergies exist, and that each separate domain stand to gain much from the ideas and knowledge found in the other ones. 
In order to make this possible, the study had to acquire a scope to explore all in-between synergies and interactions, leading to geo-vpls, web-vpls, and browser-based geocomputation. 
Now that this study has made these connections explicit, future studies can focus on more precise aspects of these cornerstones again.

\textbf{Imbalance between software implementation and study}

The third 'danger' which remained an ongoing balancing act during the execution of this study, is the balance between 'performing a study' and 'developing an application'. 
Indeed, many of the aspects discussed throughout this study come down to implementation aspects of the geofront application. 

This is why the study has attempted to generalized its findings as much as possible.
Geofront is regarded as a proxy of geo-web-VPLS in general, in the sense that whatever was encountered during implementation of the application, must be the same for any attempt at creating a web-based vpl for geocomputation.

\textbf{Subjectivity in qualitative assessment}

Lastly, many of the assessments made by this study are qualitative assessment, and as such, might suffer from a high level of subjectivity. 
This is unavoidable in any assessment which does not come down to clear, quantifiable aspects, such as performance, memory usage or precision. 

Nevertheless, the study has attempted to scope this subjectivity by basing its assessments heavily on prior works in the field of vpl, and always showcasing clear examples. 


% \section{Personal Reflection}

% One of the goals of this study was to investigate and explore browser-based geo-computation, and there are many ways of conducting exploitive studies. 
% This study chose for a practical approach: investigation by means of creating an application.
% The advantage of this approach is that it leads to tangible results. 

% The disadvantage of this approach is that software development can lead a study astray, if the development needs of the application are put before the needs of the study itself.

% This study was a battleground between "performing a study on if geocomputation benefits from the web \& a vpl", and "lets build an open-source tool to aid geocomputation"

% It needed a bit of both.


% \begin{note}
  
%   Reflection
  
%   "It is not the task of the University to offer what society asks for, but to give what society needs" ~ Edsger W. Dijkstra
  
%   Despite the cliche of quoting Dijkstra, and despite the arrogant danger of pretending to know what people want better than they themselves know what they want, 
  
  
%   'what the world wants is more react, angular, or vue developers'
%   'what the world needs, in my opinion, is guidance on this front of web vs native application development.'
  
%   right now, we are sucked in the rabbit hole of 'make everything web-based, then maybe try to execute that natively',
%   And I think that we should start working towards a situation of: 'make everything natively, KNOWING that publishing it on the web is a piece of cake'.
  
  
  
%   The thesis represents to me a bold "What if" scenario: 
%   - What if geodata computation was an elegant, ergonomic process?
%   - What if more people could more easily perform geospatial computations?
%   - What if textual and visual programming languages worked complimentary?    
%   - What if geocomputation libraries where written in Rust instead of C/C++?
%   - 
  
%   'we are not inventive enough with the tools at our disposal'
%   'outdated web vs native, client vs server distinctions block our vision from seeing new types of applications'
%   'we could be doing so much more with what we have.'
  
  
%   A huge leap 
  
  
% \end{note}
  
%   \subsection{Personal Motivation}
%   During my internship I was tasked with creating a parametric 3D CAD model. 
  
%   - local usage 
%     -> quick, direct feedback
  
%   - We needed to make this a product for end users. 
  
%   - Industry-standard choice: cloud 
%     -> smart-server dumb-client setup, cloud-native architecture 
  
%   - Problems
%     -> continuously downloading new resulting CAD files after every change created a lot of web traffic. 
%     -> slow, not at all the same experience.
%     -> cloud host was even more slow in cold-start scenario's   
%     -> cloud host monetization scheme: pay for every time the script runs, 
%        -> meant that consumers had to be discouraged to 'play around' with the tool too much. 
    
%   This made me question the cloud-based paradigm, at least for the use case of calculating geometry by end users. 
  
%   At the same time, many of our parametric designers could use grasshopper, and only grasshopper. 
  
%   This led me to think about a vpl which can run client-side in a browser, and can produce client-side applications.
  
%   if the data is geodata or CAD data, does not matter besides the fact that geodata is often big data.
  
  

%   \textbf{Motivation 2: Accessible Geoprocessing Libraries}
  
%   Most industry-standard geoprocessing libraries such as CGAL are difficult to use by anyone but experts in the field. A steep learning curve combined with relatively complex installation procedures hinders quick experimentation, demonstration, and widespread utilization of these powerful tools. It also limits the interdisciplinary exchange of knowledge, and compromises the return of investment the general public may expect of publicly funded research.
  
%   Geofront could improve the accessibility of existing geodata processing and analysis libraries, without adding major changes to those tools, by loading webassembly-compiled versions of them, similar to [other web demo's](todo).

% \subsection{Reproducibility}

% Results themselves are insanely reproducable.
% Software can be used, you can reproduce results easely by dumping versions of geofront in 
% a folder.

% \dots

% Software can also be build without too many difficulties, but the procedure has some unconventional build steps: 

% \dots

% BUT, the code is not the cleanest, nor the most conventional. minus points on open-source accessibility.

%%%%%%%%%%%%%%%%%%%%%%%%%%%%%%%%%%%%%%%%%%%%%%%%%%%%%%%%%%%%%%%%%%%%%%%%%%%%%%%
%%%%%%%%%%%%%%%%%%%%%%%%%%%%%%%%%%%%%%%%%%%%%%%%%%%%%%%%%%%%%%%%%%%%%%%%%%%%%%%
%%%%%%%%%%%%%%%%%%%%%%%%%%%%%%%%%%%%%%%%%%%%%%%%%%%%%%%%%%%%%%%%%%%%%%%%%%%%%%%

% \section{Self Assessment}

% \subsection{Using Geofront for CAD or BIM}

% \todo{make it more nuanced}
% % AT THE END OF THE DAY, THERE IS NO REAL DIFFERENCE BETWEEN CAD, BIM and GIS.
% % of course, there are many differences, like required precision and tolerances, which types of interfaces and operations are common, and the subject matter it represents. 
% % But on a deeper, fundamental level, they are all the same: its just a bunch of 2D / 3D data, representing some real world thing. 

% Today, we see the need for collaboration between CAD, BIM and GIS. Entire industries (Speckle, FME) have been introduced to bridge the gaps. 

% All three of CAD, BIM and GIS want the ability to join solids together, desire to give certain spatial objects metadata, and want to run automated workflows in the cloud. These Individual fields are constantly reinventing features the other fields have already figured out. BIM is starting to open up to the idea of streaming only a part of the building instead of the whole thing, something which the GIS world has been doing for years. On the other hand, GIS is only now starting to make the transition to 3D, a transition not unlike to how BIM is replacing 2D CAD in the AEC industry. 

% We will allow geofront to be fully customized by different plugins. By unloading all GIS plugins, and adding all BIM plugins, we turn GeoFront from a GIS to a BIM tool.
