% repeat results and answers in shortened form
\chapter{Conclusion}


In this article we described the design, creation and evaluation of GeoFront, a web-based point-cloud processing tool.


\begin{lstlisting}
Contributions: 
- a web based visual programming language for geometry computation.  
- a novel vpl plugin system which maximally re-uses existing webassembly / javascript.infrastructures. 
- a novel manner of vpl-app publication: both as a link and by making the vpl compile to javascript.

central to both these contributions
\end{lstlisting}





% With GeoFront, geoprocessing flowcharts can be created, shared and run from within a web browser.  
% The full application runs front-end in a browser, and both end results and intermediate products can be inspected in a 3D viewer.

% The tool offers functionalities such as point cloud loading, triangulation, and isocurve extraction.
% These functionalities can be expanded upon though a plugin system which utilizes the existing "Node Package Manager" infrastructure and WebAssembly.
% By using both, industry standard geoprocessing libraries such as `CGAL`, `GDAL` and `PROJ`, and data parsing libraries such as `IFC.js` and `laz-rs`, can be utilized.

% In addition to the goal of examining wasm-based geo-computation, the auxiliary goal of this thesis is to make geoprocessing more accessible. 
% By being free and open source, usable in a browser, and by focussing on the integration of existing geoprocessing libraries, GeoFront attempts to be more in line with the wider vision of cloud-native Geospatial than visual geo-computation environments, like FME or Grasshopper. 
% This is done to be in line with the aforementioned cloud native vision of eventually allowing non-expert usage of GIS.

...
This demonstrates that FAIR, containerized geoprocessing might serve more use-cases than the cloud-native perspective. 
... 


% (I must figure out how to frame this thesis more compactly. I must focus on a smaller aspect of geofront than the totality of it.)

% By creating geofront, this thesis was able to discover .............

% - advantages: 
%   - The web **is** able to facilitate a visual programming language.
%     - does indeed make excellent use of accessibility & interactivity aspect
%   - reasonable performance 
%     (- great considering the platform)

% - disadvantages: 
%   - all in-between data must be stored in memory if it is to be inspected.
%     - Can't make use of 'writing files', so that something can be removed from memory 
%     - BUT, even when using emscripten, you are still caching all sorts of things

%   - The web is able to be used for geoprocessing, albeit with some caveats
%     - Less control and precision
%     - TypedArrays,
%     - Geometric predicates 
%     - Rounding
%     - ETC.

%   - Many of these things can be fixed with webassembly, but webassembly itself has other shortcomings
%     - Differences between Rust & C++

%  - Notes:

%    - would not be possible without these modern web features
%     - Web Assembly 
%     - Typed Array's 
%     - Web Workers
%     - Web Components,
%     - 2D Canvas API
%     - Web GL

% - ability to share is a true enhancement

