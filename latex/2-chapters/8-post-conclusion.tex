\chapter{Post-Conclusion}%%%%%%%%%%%%%%%%%%%%%%%%%%%%%%%%%%%%%%%%%%%%%%%%%%%%%%%%%%% CHAPTER

\section{On the future}

/ experiment to assess: 
- The fitness of the web in general for client-side geo-computation
- If new features of modern browsers mean anything for the field of geo-informatics at large 
- The topic of accessible geoprocessing.

now, answer this to the best of your ability

Many considerations

% Premature optimization is the root of all evil | Donald Knuth
% Delay decisions to the latest moments, to gain maximum context,
% Key insight into writing better compilers

While conducting this research, I came across various key insights from various studies, and there seemed to be a link between them 

Most important effort I saw is the "denichification" of the geospatial world.
- Hugo's keynote
- Linda van den Brink's PHD
- cloud-native geospatial 


\subsection{Use Cases of Browser Based Geoprocessing}



- For which use-cases might such an application be beneficial? (~Old Phase 4)
- Name all reasons for browser based geoprocessing, and the 4 audiences.

Discuss 

\section{Reproducibility}
% This chapter explains how this thesis might be reproduced, as well as 

Reproducibility
Results themselves are insanely reproducable.
Software can be used, you can reproduce results easely by dumping versions of geofront in 
a folder.

\dots

Software can also be build without too many difficulties, but the procedure has some unconventional build steps: 

\dots

BUT, the code is not the cleanest, nor the most conventional. minus points on open-source accessibility.



\subsection{Environment}%%%%%%%%%%% SECTION
- github organization 
- repo for engine 
- repo for app 
- repo for each plugin
- build procedure

\subsection{Usage}%%%%%%%%%%% SECTION

\emph{basically, write a tutorial}

This is how one can use Geofront

\subsection{Creating \& Using your own code}
\emph{show the insane (rust + wasm + npm) workflow}

Locally: 
1. Write a geoprocessing / analysis library using a system-level language (rust, C++).
2. Expose certain functions as public, using 'wasm-pack' or 'emscripten'.
3. Compile to `.wasm` + `d.ts` + `.js`.
4. publish to npm (very easy to do with wasm-pack, can also be done with emscripten)

Alternatively: 
1. Write a library using typescript, 
3. Compile to `d.ts` + `.js`.
4. publish to npm 

In Geofront: 
4. Reference the CDN (content delivery network) address of this node package. 

Congrats, this is now a publicly accessible geofront plugin!
The library is loaded, A component is created for each function, with inputs for input parameters, and a singular output. If a 'typescript tuple type' is exported, the plugin loader will create multiple outputs according to each component of the tuple.

\section{Limitations}%%%%%%%%%%% SECTION

Limitations of usage right now



\section{Future Work}
Boulevard of broken dreams :) 

This thesis has only scratched the surface of all that

- Geofront as web-based version of grasshopper
- Geofront as lightweight QGIS replacement 





\subsection{App Features}

\subsubsection{ Constrain plugin support}
- Its becoming too hard to cater to both. The more constraint plugin support becomes, the more powerful and well-integrated we can make the plugins
- It would be really nice to make rust the only way of creating plugins. 
- Build a separate plugin loader for C / C++, or only support those through C++ bindings (dont know if wasm-pack can handle C bindings)


\subsubsection{Adding first-class type support to plugin types, by creating a Rust crate }
We couldn't do this, because we wanted to cater to .ts, .js and C++- based plugins.