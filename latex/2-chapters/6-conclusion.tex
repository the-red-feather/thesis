% repeat results and answers in shortened form
\chapter{Conclusion}
\label{chap:conclusion}
In this article we described the design, creation and evaluation of GeoFront, a web-based point-cloud processing tool.

Overall, the study has succeeded in what it set out to do: designing and implementing a geo-web-vpl. 

Moreover, it has delivered an incremental workflow which can be used to quickly configure existing, native geoprocessing libraries written in C++ or Rust to be consumed and used by said geo-web-vpl.  

However, the usefulness as a fully-featured geo-computation tool is limited. 

The usefulness and 

This study concludes that based on these measurements, browser-based geo-computation  is fast enough that it can enable 

many promising use-cases, such as on-demand geodata processing apps, educational demo apps, and code sharing. 

However, extensive user-group testing is required before any definitive statements on accessibility and fitness for geo-computation can be made.  


% With GeoFront, geoprocessing flowcharts can be created, shared and run from within a web browser.  
% The full application runs front-end in a browser, and both end results and intermediate products can be inspected in a 3D viewer.

% The tool offers functionalities such as point cloud loading, triangulation, and isocurve extraction.
% These functionalities can be expanded upon though a plugin system which utilizes the existing "Node Package Manager" infrastructure and WebAssembly.
% By using both, industry standard geoprocessing libraries such as `CGAL`, `GDAL` and `PROJ`, and data parsing libraries such as `IFC.js` and `laz-rs`, can be utilized.

% In addition to the goal of examining wasm-based geo-computation, the auxiliary goal of this thesis is to make geoprocessing more accessible. 
% By being free and open source, usable in a browser, and by focussing on the integration of existing geoprocessing libraries, GeoFront attempts to be more in line with the wider vision of cloud-native Geospatial than visual geo-computation environments, like FME or Grasshopper. 
% This is done to be in line with the aforementioned cloud native vision of eventually allowing non-expert usage of GIS.


% (I must figure out how to frame this thesis more compactly. I must focus on a smaller aspect of geofront than the totality of it.)

% By creating geofront, this thesis was able to discover .............

% - advantages: 
%   - The web **is** able to facilitate a visual programming language.
%     - does indeed make excellent use of accessibility & interactivity aspect
%   - reasonable performance 
%     (- great considering the platform)

% - disadvantages: 
%   - all in-between data must be stored in memory if it is to be inspected.
%     - Can't make use of 'writing files', so that something can be removed from memory 
%     - BUT, even when using emscripten, you are still caching all sorts of things

%   - The web is able to be used for geoprocessing, albeit with some caveats
%     - Less control and precision
%     - TypedArrays,
%     - Geometric predicates 
%     - Rounding
%     - ETC.

%   - Many of these things can be fixed with webassembly, but webassembly itself has other shortcomings
%     - Differences between Rust & C++

%  - Notes:

%    - would not be possible without these modern web features
%     - Web Assembly 
%     - Typed Array's 
%     - Web Workers
%     - Web Components,
%     - 2D Canvas API
%     - Web GL

% - ability to share is a true enhancement



\section{Answers}

\subsection*{Sub Questions}

\begin{itemize}[ ]
  \item "\mySubRQOne"
\end{itemize}

la la la

\begin{itemize}[ ]
  \item "\mySubRQTwo"
\end{itemize}

la la la

\begin{itemize}[ ]
  \item "\mySubRQThree"
\end{itemize}

la la la

\begin{itemize}[ ]
  \item "\mySubRQFour"
\end{itemize}

la la la



\subsection*{Main Question}

\begin{itemize}[ ]
    \item "\myMainRQ"
\end{itemize}

The full design is represented by \refchap{chap:methodology}, the full implementation is represented by \refchap{chap:implementation}. 
But to summarize:

[make a final conclusion based upon the 4 sub-answers above]



\section*{Limitations}

The qualitative and quantitative measurements have a degree of subjectivity
- research topic is sizable
- 

% ## Accessibility
% _"Is this environment truly accessible?"_


% ## Practical 
% _"Is this environment truly a competitor to native / other methods of geoprocessing?"_

% ## As Demo / Scripting Environment 
% _"Is this environment a good demo host?"_

