\chapter{Analyses}%%%%%%%%%%%%%%%%%%%%%%%%%%%%%%%%%%%%%%%%%%%%%% 
\label{chap:analyses}

In this chapter the various software implementations and design choices made in \refchap{chap:implementation} are analyzed. 
First, \refsec{sec:analyses:representation}.
Second, \refsec{sec:analyses:compilation}.
Then, \refsec{sec:analyses:loading}.
Finally, \refsec{sec:analyses:utilization}.

\section{\mySubRQOneTitle}
\label{sec:analyses:representation}
[Analyze the base implementation of geofront]

(keep this very surface level, we have enough other things to be done)

\section{\mySubRQTwoTitle}
\label{sec:analyses:compilation}
[Analyze the web-exposed geoprocessing libraries]

(I like rust-> wasm)

(I don't like C++ -> wasm)

\section{\mySubRQThreeTitle}
\label{sec:analyses:loading}

[Assess the plugin model]

\section{\mySubRQFourTitle}
\label{sec:analyses:utilization}

[Assess the utilization aspects]

% ## Case Studies

% ### Vector
% _Vector data retrieval, transformation, visualization_

% ### Raster 
% _Raster data retrieval, transformation, visualization_

% ### 6. Experiments 
% _Performance benchmark between rust-wasm / cpp-wasm / cgal-cpp-wasm / js / cli usage_

% ## Final 
% _Answer research questions ????_


% \section{Experiments}

% \subsection{ Web Mapping Service }
% -> could work, must be captured in component
% -> streaming question

% \subsection{ Open Street Map }
% -> could be hooked up to the geojson viewer



% \section{ Performance }
% \subsection{Vector 3D}

% ....

% \subsection{Raster}

% ....

% \subsection{Geo features}

% ....


%%%%%%%%%%%%%%%%%%%%%%%%%%%%%%%%%%%%%%%%%%%%%%%%%%%%%%%%%%%%%%%%%%%%%%%%%
% \section{Use Case: Educational Sandbox}
% \begin{lstlisting}
% WHAT: 
%  - show the behaviour of a simple ransac algorithm, fitting a plane through 
%    a point cloud
%  - show it beharivourly: show in-between steps
%  - make parameters ajustable (number of high scores, minimum high score, 
%    number of tries, etc.)
%  - add least squares adjustment, and compare.

% SIMILAR EXAMPLES: 
%  - the geometric predicates explanation 
%  (https://observablehq.com/@mourner/non-robust-arithmetic-as-art)
%  - 

% ASSESS ON: 
%  - educational value
%  - ease of usage (the promise of **Criterium A**)
 
% ASSESSMENT (hypothesis): 
%  + indeed very insightful for analying the behaviour 
%    and operation of certain 
%    algorithms & parameters. Not many applications can show 
%    this level of insight. 
%  + feature B can be used to strip the tool down 
%    to the bare minimum,   helping 
%    with not overwhelming the user with features

%  - This VPL is not easy to operate. 
%    It remains difficult to communicate what needs to 
%    be done, how things work. This is not an expert tool, 
%    but also not a beginners' tool.
%  - No built-in tutorialization
%  - hard to discover the code underneath, 
%    obfuscating the link between process and code.

% \end{lstlisting}

% %%%%%%%%%%%%%%%%%%%%%%%%%%%%%%%%%%%%%%%%%%%%%%%%%%%%%%%%%%%%%%%%%%%%%%%%%
% \section{Use Case: Web Demo Environment}
% \begin{lstlisting}
%     WHAT:
%      - Show the startin delaunay triangulator
%      - Accept user-submitted Laz files as input
%        - filter the ground
%      - Accept a randomly generated point cloud based on perlin noise.
%      - Visualize the generated mesh, and make it available for download

%     EXAMPLES: 
%      - hugo's demo
    
%      ASSESS ON:
%      - **Criterium B**: extendability: 
%        - how does foreign and native codebases interact? 
%      - clarity
%      - reproducibility
%      - performance
%      - scope (raster / vector / 2D / 3D)

%     JUDGEMENTS (hypothesis): 
%      + Clarity is fine
%      + Reproducability is 
%      + Performance is decent

%      ~ clarity is fine, vpl allows users to 'play around' and try
%        different configurations, even run their own data through the
%        demonstrated functions

%      - Application is less suited for 2D 
    
% \end{lstlisting}


% \section{Use Case: Geoprocessing Environment}%%%%%%%%%%% SECTION
% \begin{lstlisting}

%     WHAT:
%       - Query an area of a point cloud with a polygon
%       - turn that area of points into a triangulation
%       - turn triangulation into isocurves
%       - save this as a geojson
%       - turn this whole thing into a function, which takes a PC, 
%         polygon and isocurve range, and spits out the geojson
%       - share this using a link
%       - turn this into an app?
%       - turn this into a script?

%     EXAMPLES: 
%      - 

%      ASSESS ON:
%      - **Criterium C**: Publicability:
%        - the ability to operationalize the application 
%        - can it be used by end-users? clarity? too much clutter

%     JUDGEMENTS (hypothesis): 
%       ~ sharing by link is possible, but for end-user usage, 
%         its very cluttered 
%       ~ compiling to js only partially works
%       ~ compiling into an app is not possible, but 
%         since everything runs client-side, it could be implemented. 

% \end{lstlisting}
