

\newcommand{\myName}{Jos Feenstra}
\newcommand{\myTitle}{GEOFRONT: A browser based visual programming language for geo-computation}

\newcommand{\myGroup}{3D geoinformation group}
\def\myGroupLogo{figs/tud-3dgeoinfo-black.png}
\newcommand{\myUni}{Delft University of Technology}

\newcommand{\myGraduationYear}{2022}
\newcommand{\myGraduationMonth}{June}

\newcommand{\mySupervisorOne}{Ir.\ Stelios Vitalis}
\newcommand{\mySupervisorTwo}{Dr.\ Ken Arroyo Ohori}
\newcommand{\mySupervisorThree}{Dr.\ Giorgio Agugiaro}
\newcommand{\myCoreader}{Dr.\ Hugo Ledoux}

%-- for names for \autoref commands
\def\chapterautorefname{Chapter}
\def\sectionautorefname{Section}
\def\subsectionautorefname{Section}
\def\subsubsectionautorefname{Section}
\def\algorithmautorefname{Algorithm}

% a way to make images fit the paper
% https://tex.stackexchange.com/questions/86350/includegraphics-maximum-width
\def\maxwidth{460pt} 


%-- for pdf metadata
\hypersetup{pdfauthor={\myName}}
\hypersetup{pdfkeywords={thesis, geomatics, TU Delft}}
\hypersetup{pdfsubject={A thesis submitted to the Delft University of Technology in partial fulfillment of the requirements for the degree of Master of Science in Geomatics}}
\hypersetup{pdftitle={\myTitle}}

%-- handy shortcuts
\newcommand{\ie}{i.e.}
\newcommand{\eg}{e.g.}
\newcommand{\reffig}[1]{Figure~\ref{#1}}
\newcommand{\refsec}[1]{Section~\ref{#1}}
\newcommand{\refchap}[1]{Chapter~\ref{#1}}

% [JF]: added a command to prettify this dodgy syntax. m stands for monospace
\newcommand{\m}[1]{\texttt{#1}}

%-- colours for the hyperlinks
\definecolor{colorforlinks}{RGB}{27, 60, 131}

\lstdefinestyle{note}
{
    breaklines=true, 
    basicstyle=\footnotesize,
    keywordstyle=\color{blue},
    commentstyle=\color[rgb]{0.13,0.54,0.13},
    backgroundcolor=\color{cyan!10},
}
\lstnewenvironment{note}
{\lstset{style=note}}
{}

% I keep changing the styling of geofront, so lets put them here
\newcommand{\geofront}{Geofront}

% I keep changing research questions, so lets put them here
% \newcommand{\myMainRQ}{\textit{How to design and implement a web-based vpl for geo-computation which supports running existing geo-computation libraries in a web browser?}}
\newcommand{\myMainRQ}{\textit{To what extend can native geocomputation libraries be \emph{represented}, \emph{compiled}, \emph{loaded}, and \emph{utilized} within a browser-based dataflow-VPL?}}

\newcommand{\mySubRQOneTitle}{Representation} % previously: facilitation
% \newcommand{\mySubRQOne}{\textit{To what extent is a browser-based application able to facilitate a generic 3D geometry vpl?}}
\newcommand{\mySubRQOne}{\textit{To what extent is the browser capable of \textbf{representing} a generic dataflow-vpl for processing 3D geometry?}}

% \newcommand{\mySubRQOne}{\textit{How to implement a visual programming environment in a browser which able to facilitate geo-computation?}}
\newcommand{\mySubRQTwoTitle}{Compilation}
\newcommand{\mySubRQTwo}{\textit{To what extent can geocomputation libraries written in system-level languages be \textbf{compiled} for web consumption?}}
\newcommand{\mySubRQThreeTitle}{Loading}
\newcommand{\mySubRQThree}{\textit{To what extent can a web-consumable library be \textbf{loaded} into a web-vpl without explicit configuration?}}
\newcommand{\mySubRQFourTitle}{Utilization}
% \newcommand{\mySubRQFour}{\textit{What are the advantages and disadvantages of \textbf{using} an existing geoprocessing library through a geo-web-vpl, as opposed to native utilization of said library?}}
\newcommand{\mySubRQFour}{\textit{To what extent can a 'geo-web-vpl' be \textbf{used} to create geodata pipelines?}}


