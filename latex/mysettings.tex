

\newcommand{\myName}{Jos Feenstra}
\newcommand{\myTitle}{Geofront: More directly accessible GIS tools using a web-based visual programming language}

\newcommand{\myGroup}{3D geoinformation group}
\def\myGroupLogo{figs/tud-3dgeoinfo-black.png}
\newcommand{\myUni}{Delft University of Technology}

\newcommand{\myGraduationYear}{2022}
\newcommand{\myGraduationMonth}{June}

\newcommand{\mySupervisorOne}{Ir.\ Stelios Vitalis}
\newcommand{\mySupervisorTwo}{Dr.\ Ken Arroyo Ohori}
\newcommand{\mySupervisorThree}{Dr.\ Giorgio Agugiaro}
\newcommand{\myCoreader}{Dr.\ Hugo Ledoux}

%-- for names for \autoref commands
\def\chapterautorefname{Chapter}
\def\sectionautorefname{Section}
\def\subsectionautorefname{Section}
\def\subsubsectionautorefname{Section}
\def\algorithmautorefname{Algorithm}

% a way to make images fit the paper
% https://tex.stackexchange.com/questions/86350/includegraphics-maximum-width
\def\maxwidth{460pt} 


%-- for pdf metadata
\hypersetup{pdfauthor={\myName}}
\hypersetup{pdfkeywords={thesis, geomatics, TU Delft}}
\hypersetup{pdfsubject={A thesis submitted to the Delft University of Technology in partial fulfillment of the requirements for the degree of Master of Science in Geomatics}}
\hypersetup{pdftitle={\myTitle}}

%-- handy shortcuts
\newcommand{\ie}{i.e.}
\newcommand{\eg}{e.g.}
\newcommand{\reffig}[1]{Figure~\ref{#1}}
\newcommand{\refsec}[1]{Section~\ref{#1}}
\newcommand{\refchap}[1]{Chapter~\ref{#1}}

% [JF]: added a command to prettify this dodgy syntax. m stands for monospace
\newcommand{\m}[1]{\texttt{#1}}

%-- colours for the hyperlinks
\definecolor{colorforlinks}{RGB}{27, 60, 131}

\lstdefinestyle{note}
{
    breaklines=true, 
    basicstyle=\footnotesize,
    keywordstyle=\color{blue},
    commentstyle=\color[rgb]{0.13,0.54,0.13},
    backgroundcolor=\color{cyan!10},
}
\lstnewenvironment{note}
{\lstset{style=note}}
{}

% I keep changing research questions, so lets put them here
% \newcommand{\myMainRQ}{\textit{How to design and implement a web-based vpl for geo-computation which supports running existing geo-computation libraries in a web browser?}}
% \newcommand{\myMainRQ}{\textit{How can native geocomputation libraries be \emph{compiled}, \emph{loaded}, and \emph{utilized} within a browser-based dataflow-VPL?}}

% \newcommand{\mySubRQOne}{\textit{To what extent is a browser-based application able to facilitate a generic 3D geometry vpl?}}
% \newcommand{\mySubRQOne}{\textit{Representation: How to implement a browser-based dataflow-vpl for processing 3D geometry?}}
% \newcommand{\mySubRQTwo}{\textit{Compilation: How can geocomputation libraries written in system-level languages be \textbf{compiled} for web consumption?}}
% \newcommand{\mySubRQThree}{\textit{Loading: To what extent can a web-consumable library be \textbf{loaded} into a web-vpl without explicit configuration?}}
% \newcommand{\mySubRQFour}{\textit{Utilization: How can a 'geo-web-vpl' be \textbf{used} to create geodata pipelines?}}

\newcommand{\myNewMainRQ}{\textit{Is a web based VPL a viable method for directly accessing native GIS libraries with a composable, user-defined interface?}}
% \newcommand{\myNewMainRQ}{\textit{Does a web-based, GUI-rich VPL lead to more \textbf{accessible} and \textbf{composable} native GIS libraries, compared to alternative methods?}}
\newcommand{\myNewSubRQOne}{\textit{What GUI features are required to facilitate this method, and to what extend does the web platform aid or hurt these features?}}
\newcommand{\myNewSubRQTwo}{\textit{To what extend does this method intent to address the discrepancies between software applications and libraries, as described by Elliott (2007)? Does it succeed in doing so?}}
\newcommand{\myNewSubRQThree}{\textit{What are the differences between compiling and using a GIS library written in C++ in a web browser, compared to a GIS library written in Rust?}}
% how can .... be compiled, loaded, and utilized within a static, web-based VPL?
\newcommand{\myNewSubRQFour}{\textit{What measures are taken to make this VPL scalable to large geo-datasets, and how effective are these measures?}}
\newcommand{\myNewSubRQFive}{\textit{How does this method compare to existing, alternative VPLs and browser-based geocomputation methods, regarding the properties mentioned in the prevous questions?}}
%  (2. Are the functional properties of a dataflow-VPL uphold by this solution?)


% \item Which Input and Output features does the VPL need to effectively use GIS libraries?
% \item How does this solution compare to similar VPLs? 

% ********************************************************************
% listings from arsclassica
% ********************************************************************

\let\orgtheindex\theindex
\let\orgendtheindex\endtheindex
\def\theindex{%
	\def\twocolumn{\begin{multicols}{2}}%
	\def\onecolumn{}%
	\clearpage
	\orgtheindex
}
\def\endtheindex{%
	\end{multicols}%
	\orgendtheindex
}

\makeindex

% ********************************************************************
% listings
% ********************************************************************

\definecolor{lightergray}{gray}{0.99}

\lstset{language=[LaTeX]Tex,
    keywordstyle=\color{RoyalBlue},
    basicstyle=\normalfont\ttfamily,
    commentstyle=\color{Emerald}\ttfamily,
    stringstyle=\rmfamily,
    numbers=none,
    numberstyle=\scriptsize,
    stepnumber=5,
    numbersep=8pt,
    showstringspaces=false,
    breaklines=true,
    frameround=ftff,
    frame=lines,
    backgroundcolor=\color{lightergray}
} 

\lstset{morekeywords=%
        {function, f32, i32},
        commentstyle=\color{Emerald}\ttfamily,%
        frame=lines}

\lstset{basicstyle=\normalfont\ttfamily}
\lstset{flexiblecolumns=true}
\lstset{moredelim={[is][\normalfont\itshape]{/*}{*/}}}
\lstset{basicstyle=\normalfont\ttfamily}
\lstset{flexiblecolumns=false}
\lstset{moredelim={[is][\ttfamily]{!?}{?!}}} 
\lstset{escapeinside={£*}{*£}}
\lstset{firstnumber=last}
\lstset{moredelim={[is][\ttfamily]{!?}{?!}}}

\DeclareRobustCommand*{\pacchetto}[1]{{\normalfont\ttfamily#1}%
\index{Pacchetto!#1@\texttt{#1}}%
\index{#1@\texttt{#1}}}

\DeclareRobustCommand*{\bibtex}{\textsc{Bib}\TeX%
\index{bibtex@\textsc{Bib}\protect\TeX}%
}

\DeclareRobustCommand*{\amseuler}{\protect\AmS{} Euler%
\index{AmS Euler@\protect\AmS~Euler}%
\index{Font!AmS Euler@\protect\AmS~Euler}}

\lstnewenvironment{code}% 
{\setkeys{lst}{columns=fullflexible,keepspaces=true}%
\lstset{basicstyle=\small\ttfamily}%
}{}

\lstset{extendedchars} 
\lstnewenvironment{sidebyside}{% 
    \global\let\lst@intname\@empty 
    \setbox\z@=\hbox\bgroup 
    \setkeys{lst}{columns=fullflexible,% 
    linewidth=0.45\linewidth,keepspaces=true,%
    breaklines=true,% 
    breakindent=0pt,%
    boxpos=t,%
    basicstyle=\small\ttfamily
}% 
    \lst@BeginAlsoWriteFile{\jobname.tmp}% 
}{% 
    \lst@EndWriteFile\egroup 
        \begin{center}% 
            \begin{minipage}{0.45\linewidth}% 
                \hbox to\linewidth{\box\z@\hss} 
            \end{minipage}% 
            \qquad 
            \begin{minipage}{0.45\linewidth}%
            \setkeys{lst}{frame=none}% 
                \leavevmode \catcode`\^^M=5\relax 
                \small\input{\jobname.tmp}% 
            \end{minipage}% 
        \end{center}% 
} 

\lstset{numbers=left,
    numberstyle=\scriptsize,
    stepnumber=1,
    numbersep=8pt
}   