\section{Conclusion}

This paper analysed the current openness status of the Scottish Spatial Data Infrastructure, 
the barriers in the way of a more open SDI, and the strategies which can be taken for achieving this. This state was assessed using a custom SDI assessment framework, based on the works by \citet{assessing_openness_SDI_2018}, \citet{mulder}, and \citet{olausson}. 

This concluded that the SSDI in general can be considered halfway in between a second and third generation SDI, which still needs a couple of steps before it can be called an Open SDI. While The SSDI is easy to discover, uses a lenient licensing framework, and its data is properly machine readable, the lack of especially sufficient user involvement limits the openness status. 

The integration of the SDI and the parcel dataset is also limited. This dataset has a web viewer which is generous in its meta-data, including market evaluations. However, we were unable to find a machine-readable or downloadable version of this dataset, at least not for more than one parcel. The parcel dataset also makes use of an entirely separate technical solution, in the form of a custom data portal. 

Open participation in policy making \& standardization regarding the SSDI is, thanks to the efforts made on behalf of the Open Government Partnership, well on its way on a general level. The Government of Scotland appears very committed to open government plans, visions and strategies since 2016, and uses multiple action plans to also implement these ideas. This paper remains indecisive on wow well this applies to the SSDI itself, but the commitments to open government will make sure that eventually the SSDI will be influenced by these developments. 

During analysis, several major barriers were found that prevents the SSDI from becoming an Open SDI. These include legislative complications due to Brexit negotiations, A reluctance by the SSDI to improve user participation, and a lack of agreement of key stakeholders. Per barrier, we suggest a strategy towards mitigating it. We base these suggestions on the Government Instrument paradigm \citep{gov_instruments}and the Kindon policy model \citep{kingdon}, alongside 'best practice' concepts and examples. The main strategy is to solve the discrepancies between stakeholders in order to improve collective decision making \& strategic management. 

\newpage
\section{Limitations}

It is important to name the limitations of this study. Due to the scope of this analysis, only a low-resolution overview of the status of the Scottish SDI could be made. As an SDI is both complex and open to interpretation, the conclusions of this analysis should not be taken as absolute facts, but as conclusions based upon the data found and provided. Further research is required to ensure the validity of our findings. 

It was also found to be difficult to get a complete oversight on all forces applied to the SSDI. Due to scope limitations, not all stakeholders could be explored, and stakeholders themselves had to be briefly characterized by their main visions. Future research is needed for a more complete overview on stakeholders involved with the SDI, as well as making a more nuanced image of their visions and internal conflicts. 


\section{Recommendations} 

Both Scientific and practical actions can be taken to continue the work set out by this study. As the Brexit situation is still a relatively recent issue, the UK leaving the European Union at the start of this current year, we believe that more research into the actual effects of this geo-political decision on the Scottish SDI is needed. Secondly, the proposed strategies represents a starting point for a more nuanced report on how the key stakeholders should proceed in order to evolve the SSDI to a fully open one. 