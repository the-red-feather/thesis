% TODO 
% - do a final check by comparing our intro to other intro's (cyprus, belgium)
% SOURCES

\section{Introduction}

% ELEMENTS
% - Justification: 
%   - Explain GI and advantages
%   - Explain SDI (introduce it) and advantages
%   - Which knowledge will be added to the research already done? 
% - Research question:     
% - Cause- effect: how does it affect ... ? 
% - Aim and scope of the paper 
% - reading guide (LEAVE EMPTY FOR NOW, THERE IS NOTHING TO GUIDE TO YET)

% REQUIREMENT:    
% - 500 words 

% ELEMENTS OF A GENERAL INTRODUCTION (take it with a grain of salt / must have):
%
% 1.    Introducing the topic(from broad perspective to focus) 
%        --- Call attention to the specific subject or hypothesis that you are to discuss
% 2.    Why: Problem/challenge 
%       --- Provide background and justify your study relative to its importance and the results of other studies (your contribution to the Body Of Knowledge)
% 3.    What:research question/problem statement --- List the objective of your
%       research project/give information on what you plan to accomplish in your
%       paper (Objective of your work)
% 4.    How:research methodology --- Explain how you accomplished the research:–Quantitative:e.g.,surveys–Qualitative:e.g.,case studies,interviews–Desk/literature study
% 5.    Aim of paper 
% 6.    Reading guide


% grading 
% - The topic is smoothly introduced and (societal/ scientific) relevance is very clear %   and convincing, research method is presented, reading guide present




% 
The goal of a spatial data infrastructure, to improve interoperability of data \citep{emergance_open_data_2018}, has lead SDI developers to standardize data not only within the context of one SDI, but across several. On a European level, various efforts have been made to unify datasets across the borders of EU member-states \citep{inspire_directive_2007}, \citep{psi_directive_law_2011}, \citep{open_data_directive_2019}. These cross-continent initiatives have impacted the development of the National SDI of the member states. The emphasis on transparency, open data, High Value Datasets, and the re-use of public sector information necessitates that the NSDIs of member states evolve beyond their current level of openness. This is why the concept of 'openness' of an SDI has come into play in recent years. 


%
The subject of this paper is Scotland. While the people of the United Kingdom as a whole voted to leave the European Union in 2016, the people of Scotland decidedly voted to stay \citep{brexit}. This brings the country of Scotland in a complicated position regarding the commitment to European directives. At the same time, research on the current state of openness of the Scottish SDI (SSDI) does not exist. All of this complicates the development of the SSDI as a whole. 


% purpose of paper : propose a SDI strategy : Research question
The main purpose of this paper is to clarify these uncertainties. This will hopefully lead to further improvements upon this SSDI, all to the benefit of the people of Scotland. The research question that this paper aims to answer is: What is the current openness status of the National Spatial Data Infrastructure in Scotland, what are the challenges in the way of more openness, and what steps can be taken for achieving an open SDI. While these questions will be explored from the perspective of Scotland, this paper aims to be applicable to other countries as well. 


%research methodology -> meta-analysis
The primary method for answering this question will be to apply an assessment framework to Scotland's existing geo-data portals. This is supported by literature studies, analysis on the documentations provided by the Scottish Government and the European Union, as well as already-existing research papers on this topic created by independent researchers.


% READING GUIDE
This paper uses the following structure: Section two covers all relevant theory and background information. It states the definitions of (open) SDI's, their components and different generations. this is then used to state and explain the used assessment framework. Section three covers a qualitative assessment of the SSDI according to this framework. Section four states the main barriers which prevent the SSDI of becoming more open, followed up by section five, which grants a number of suggestions \& strategies which could mitigate these barriers. Lastly, section six grants a conclusion and recommendations.