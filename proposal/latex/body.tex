\section{Kaas}
lalalala 


\section{Henk}
lalalala 


\section{Peep}



% %%%%%%%%%%%%%%%%%%%%%%%%%%%%%%%%%%%%%%%%%%%%%%%%%%%%%%%%%%%%%%%%%%%%%%%%%%%%%%%
% \section{Introduction}
% An introduction in which the relevance of the project and its place in the context of geomatics is described, along with a clearly-defined problem statement.

% "The Mystery of life is not a problem to solve, but a reality to experience"\cite{ref:einstein}

% %%%%%%%%%%%%%%%%%%%%%%%%%%%%%%%%%%%%%%%%%%%%%%%%%%%%%%%%%%%%%%%%%%%%%%%%%%%%%%%
% \section{Related work}
% A related work section in which the relevant literature is presented and linked to the project.

% %%%%%%%%%%%%%%%%%%%%%%%%%%%%%%%%%%%%%%%%%%%%%%%%%%%%%%%%%%%%%%%%%%%%%%%%%%%%%%%
% \section{Research questions}
% The research questions are clearly defined, along with the scope (ie what you will not be doing).

% To help you define a \"good\" research question, 
% read \url{https://sites.duke.edu/urgws/files/2014/02/Research-Questions_WS-handout.pdf}.

% % \citet{Smith03} cites something as a noun, and it's also possible to put the references between parentheses at the end of at sentence~\citep{Smith03}


% %%%%
% \section{Methodology}
% Overview of the methodology to be used.

% %%%%
% \section{Time planning}
% Having a Gantt chart is probably a better idea then just a list.

% %%%%
% \section{Tools and datasets used}
% Since specific data and tools have to be used, it’s good to present these concretely, 
% so that the mentors know that you have a grasp of all aspects of the project.
