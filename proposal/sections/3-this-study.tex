%-------------------------------------------------------------------------------------------------%
% [HUGO]: The research questions are clearly defined, along with the scope (ie what you will not be doing).
% To help you define a "good" research question, 
% read \url{https://sites.duke.edu/urgws/files/2014/02/Research-Questions_WS-handout.pdf}.
% [ From Hugo's handout: ]
% 
% clear | focussed | unique | 
% start asking open-ended “How?” “What?” and Why?” questions. 
% Then evaluate possible responses to those questions
% While a good research question allows the writer to take an arguable position, 
% it DOES NOT leave room for ambiguity.
% 
% 1)Is the research question something I/others care about? Is it arguable?
% 2)Is the research question a new spin on an old idea, or does it solve a problem?
% 3)Is it too broad or too narrow?
% 4)Is the research question researchable within the given time frame and location?
% 5)What information is needed?
% 
% are you trying to accomplish one of these goals? 
% 1) Define or measure a specific fact or gather facts about a specific phenomenon. -> YES. IT TRIES TO GATHER FACTS ABOUT WEBASSEMBLY
% 4) Prove that a certain method is more effective than other methods. -> ONLY WASM compared to native
% Moreover, the research question should address what the variables of the experiment are, their relationship, 
% and state something about the testing ofthose relationships. 
% 
% [JF] : I really know what I want to do, and I am convinced improving the usability of geoprocessing tools is both valuable and Good. 
% 
\newpage
\section{This Study}

% This paper's main objective is to judge the fitness of WebAssembly for client-side geo-processing purposes. 
% This fitness will be judged quantitatively by means of performance benchmarks and compilation performance, as well as qualitatively by documenting the creation of a web-based geoprocessing application using WebAssembl
% The study aims to provide a guide for using WebAssembly for Geoprocessing

This study's main objective is to determine if and how WebAssembly can practically enable client-side geoprocessing. It aims to gather facts and develop tools around both wasm and client-side geoprocessing, so that this judgement can be fairly made. 
% 1 / 3
The required facts will be gathered by means of performance benchmarks, as well as a compilation method comparison, and a distribution method comparison.  
% 2
The tools take the shape of a client-side GIS application using WebAssembly.
To ensure that these facts \& tools are unambiguous, reproducible, and re-usable, the study will offer guides on exactly how certain C++ geoprocessing libraries are converted to wasm. For the same reasons, it also presents implementation details and considerations of the geoprocessing environment.  
% 4
Finally, by presenting and using this environment, the study aims to explore the advantages and disadvantages of a client-side-GIS application equipped with powerful geoprocessing tools. 

% This will require research into the technical effectiveness of WebAssembly. 
% C++ geoprocessing libraries such as CGAL \& GDAL will be tested on their ability to be compiled, loaded, and used from a browser. 
% This is compared against their compilation by other means, such as native binaries or the aforementioned `asm.js`. 

% This research is complemented by an extensive 'case study' to explore the design possibilities of a web-application equipped with client-side geoprocessing. 
% A thick-client web application will be created, and this will serve as platform for testing the aforementioned `wagl`'s. 
% The tool and can be additionally used for acquiring, visualizing, and saving geodata. 

% (WebAssembly geoprocessing libraries -> `wagl`'s)

%-------------------------------------------------------------------------------------------------%
\subsection{Research Questions}

"How well does WebAssembly support client-side geoprocessing environments \& applications?"

\subsubsection*{Sub Questions}

1 : What is the most effective methodology of compiling sizable C++ geoprocessing libraries to WebAssembly?

2 : What should a client-side geoprocessing environment look like?

3 : What is the optimal way of distributing, loading, and using wasm-compiled geoprocessing libraries **in** a client-side geoprocessing environment?

4 : What are the advantages and disadvantages of GIS applications created using a client-side geoprocessing environment?

% ----
\newpage
\subsection{Scope (\& Method?)}


\subsection*{Will Include}

\subsubsection*{Benchmarks of C++ geoprocessing libraries native versus wasm}

The performance benefit of WebAssembly is an important component of why WebAssembly might be beneficial for client-side geoprocessing. As such, this research is interested in confirming whenether this is the case for geoprocessing applications. Individual functions of geoprocessing libraries will be benchmarked using five different methods: 

\begin{itemize}
    \item Compiled and run as native binary (g++), 
    \item Compiled to wasm, run natively (WASI),
    \item Compiled to wasm, run in a browser,
    \item Compiled to asm.js, run natively (NODE.js),
    \item Compiled to wasm, run in a browser. 
\end{itemize}


\subsubsection*{Comparison and discussion of multiple wasm compilation methods for large libraries}
This research will try out multiple methods of WASM compilation. 
Since compiling and downloading full libraries might become very slow, We might need to look at incremental methods of compiling and distributing libraries. 
This is possible since Wasm libraries do accept dependencies. 
Additionally, 


\subsubsection*{Implementation details of creating a geo-web-vpl}

This study will include implementation details of the VPL, and the design considerations made during its development. These decisions will be informed by analysis of exsting geo-vpl's, and studies discussing the usability aspects of VPL's (SOURCE: VPL Usability paper,  Ravi Peter)


\subsubsection*{Conclusions and Discussion on using WebAssembly for geo-processing}

Additionally, this study will give an answer to if and how WebAssembly contributed to performant and sharable geoprocessing. 
This will be based on the aforementioned quantitative assessment of performance and compilation possibilities, but also on the experience gathered by using the built application.  

\subsubsection*{Conclusions and Discussion on using a web based VPL for geo-processing}

Additionally, this study will give an answer to if and how a VPL contributed to more usable geoprocessing. This will be a purely qualitative assessment, based on the findings and experience using the environment.  


%-----------------------------------------------------------------------------%
\subsection*{Will not include}

\subsubsection*{Server-side WebAssembly} % **Client-side WebAssembly Only**

This study will limit itself to the **client-side** usage of WebAssembly. 
A powerful case can be made for **server-side** usage of WebAssembly, especially in conjunction with a programming language such as Rust. 
Rust compiled to WebAssembly could, compared to using python, java or C++, make geoprocessing more maintainable and reliable, while at the same time ensuring memory safety, security, and performance [SOURCE: wasi, wasm-ai]. 
Server-side wasm is beyond the scope of this paper, but would be an excellent starting point for future work. 

Note that this also means that research into `wagl`'s is important for more than just client-side geoprocessing. All geoprocessing could benefit from it.



\subsubsection*{Web Processing Services} % Will not be dealing with WPS 

This research will exclude the OGC standard of web processing services (SOURCE), since these services are not about \emph{client} side geoprocessing, but instead cover \emph{server} side geoprocessing. 
Future work could, however, research the possibility of utilizing a vpl for WPS orchestration. 



\subsubsection*{Usability Analysis of geoprocessing VPL's} % 

While usability* is a primary motivation for conducting this research, no claims will be made that a client-side vpl geoprocessing environment is more usable to native GIS applications or geoprocessing methods. This research attempts to solve practical inhibitions in order to discover whether or not client-side is \textbf{an} option. If it turns out that this method is viable technically, future research will be needed to definitively proof \textbf{how} usable it is compared to all other existing methods.  

% This paper seeks to first close this gap, limiting itself to overcoming the technical and design boundaries in the pursuit of practical client-side geoprocessing.

* (Usability within this context is meant as a collection of aspects such as "performance", "ease of use", and FAIR principles. )

Similarly, a survey analyzing how users experience client-side geoprocessing in comparison to native geoprocessing must also be left to subsequent research. While this would gain us tremendous insight, client-side geoprocessing is too new to make a balanced comparison. Native environments like GRASSGIS, QGIS, FME or Esri simply have a 20+ year lead in research and development. 





% ### This Study

% The aim of this study is to provide an environment meant for client-side geoprocessing using WebAssembly. This environment will be used to demonstrate if and how wasm-based, client-side geoprocessing is possible. At the same time, by presenting this environment, the study aims to explore the design possibilities of a web-GIS application equipped with such tools. 

% This will require research into the technical effectiveness of WebAssembly. C++ geoprocessing libraries such as CGAL & GDAL will be tested on their ability to be compiled, loaded, and used from a browser. This is compared against their compilation by other means, such as native binaries or the aforementioned `asm.js`. This research is complemented by an extensive 'case study' to explore the design possibilities of a web-application equipped with client-side geoprocessing. A thick-client web application will be created, and this will serve as platform for testing the aforementioned `wagl`'s. The tool and can be additionally used for acquiring, visualizing, and saving geodata. 
