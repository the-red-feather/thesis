%-------------------------------------------------------------------------------------------------%
% [HUGO]: The research questions are clearly defined, along with the scope (ie what you will not be doing).
% To help you define a "good" research question, 
% read \url{https://sites.duke.edu/urgws/files/2014/02/Research-Questions_WS-handout.pdf}.
% [ From Hugo's handout: ]
% 
% clear | focussed | unique | 
% start asking open-ended “How?” “What?” and Why?” questions. 
% Then evaluate possible responses to those questions
% While a good research question allows the writer to take an arguable position, 
% it DOES NOT leave room for ambiguity.
% 
% 1)Is the research question something I/others care about? Is it arguable?
% 2)Is the research question a new spin on an old idea, or does it solve a problem?
% 3)Is it too broad or too narrow?
% 4)Is the research question researchable within the given time frame and location?
% 5)What information is needed?
% 
% are you trying to accomplish one of these goals? 
% 1) Define or measure a specific fact or gather facts about a specific phenomenon. -> YES. IT TRIES TO GATHER FACTS ABOUT WEBASSEMBLY
% 4) Prove that a certain method is more effective than other methods. -> ONLY WASM compared to native
% Moreover, the research question should address what the variables of the experiment are, their relationship, 
% and state something about the testing ofthose relationships. 
% 
% [JF] : I really know what I want to do, and I am convinced improving the usability of geoprocessing tools is both valuable and Good. 
% 
\newpage
\section{This Study}

% This paper's main objective is to judge the fitness of WebAssembly for client-side geo-processing purposes. 
% This fitness will be judged quantitatively by means of performance benchmarks and compilation performance, as well as qualitatively by documenting the creation of a web-based geoprocessing application using WebAssembl
% The study aims to provide a guide for using WebAssembly for Geoprocessing





% The studies final objective is to make client-side geoprocessing a viable alternative to native geoprocessing. It does this by researching if the current state of web technologies are ready for geoprocessing:
% "Can WebAssembly offer the performance and interoperablility with C++ needed for geoprocessing?"
% Can the "web ui", consisting of HTML, css and javascript, offer us enough to work with to create a UI needed for proper geoprocessing?% 

This study's main objective is to determine if and how WebAssembly can practically enable client-side geoprocessing. It aims to gather facts and develop tools around both wasm and client-side geoprocessing, so that this judgement can be fairly made. 
% 1 / 3
The required facts will be gathered by means of performance benchmarks, as well as a compilation method comparison, and a distribution method comparison.  
% 2
The tools take the shape of a client-side GIS application using WebAssembly.
To ensure that these facts \& tools are unambiguous, reproducible, and re-usable, the study will offer guides on exactly how certain C++ geoprocessing libraries are converted to wasm. For the same reasons, it also presents implementation details and considerations of the geoprocessing environment.  
% 4
Finally, by presenting and using this environment, the study aims to explore the advantages and disadvantages of a client-side-GIS application equipped with powerful geoprocessing tools. 

% This will require research into the technical effectiveness of WebAssembly. 
% C++ geoprocessing libraries such as CGAL \& GDAL will be tested on their ability to be compiled, loaded, and used from a browser. 
% This is compared against their compilation by other means, such as native binaries or the aforementioned `asm.js`. 

% This research is complemented by an extensive 'case study' to explore the design possibilities of a web-application equipped with client-side geoprocessing. 
% A thick-client web application will be created, and this will serve as platform for testing the aforementioned `wagl`'s. 
% The tool and can be additionally used for acquiring, visualizing, and saving geodata. 

% (WebAssembly geoprocessing libraries -> `wagl`'s)

%-------------------------------------------------------------------------------------------------%
\subsection{Research Questions}

"How well does the current state of web technologies support a client-side geoprocessing vpl?"

\subsubsection*{Sub Questions}

1 : What is the most effective methodology of compiling sizable C++ geoprocessing libraries to WebAssembly?

2 : How to design and create a client-side geoprocessing interface?

3 : How can wasm-compiled geoprocessing libraries be used with a client-side geoprocessing interface?

4 : What are the advantages and disadvantages of GIS applications created using a client-side geoprocessing environment powered by WebAssembly?

% ----
\newpage
\subsection{Scope}


\subsection*{Will Include}

The 'will include' scope is represented by the Methodology chapter. 

%-----------------------------------------------------------------------------%
\subsection*{Will not include}

\subsubsection*{Server-side WebAssembly} % **Client-side WebAssembly Only**

This study will limit itself to the **client-side** usage of WebAssembly. 
A powerful case can be made for **server-side** usage of WebAssembly, especially in conjunction with a programming language such as Rust. 
Rust compiled to WebAssembly could, compared to using python, java or C++, make geoprocessing more maintainable and reliable, while at the same time ensuring memory safety, security, and performance [SOURCE: wasi, wasm-ai]. 
Server-side wasm is beyond the scope of this paper, but would be an excellent starting point for future work. 

Note that this also means that research into `wagl`'s is important for more than just client-side geoprocessing. All geoprocessing could benefit from it.



\subsubsection*{Web Processing Services} % Will not be dealing with WPS 

This research will exclude the OGC standard of web processing services (SOURCE), since these services are not about \emph{client} side geoprocessing, but instead cover \emph{server} side geoprocessing. 
Future work could, however, research the possibility of utilizing a vpl for WPS orchestration. 



\subsubsection*{Usability Analysis of geoprocessing VPL's} % 

While usability* is a primary motivation for conducting this research, no claims will be made that a client-side vpl geoprocessing environment is more usable to native GIS applications or geoprocessing methods. This research attempts to solve practical inhibitions in order to discover whether or not client-side is \textbf{an} option. If it turns out that this method is viable technically, future research will be needed to definitively proof \textbf{how} usable it is compared to all other existing methods.  

% This paper seeks to first close this gap, limiting itself to overcoming the technical and design boundaries in the pursuit of practical client-side geoprocessing.

* (Usability within this context is meant as a collection of aspects such as "performance", "ease of use", and FAIR principles. )

Similarly, a survey analyzing how users experience client-side geoprocessing in comparison to native geoprocessing must also be left to subsequent research. While this would gain us tremendous insight, client-side geoprocessing is too new to make a balanced comparison. Native environments like GRASSGIS, QGIS, FME or Esri simply have a 20+ year lead in research and development. 





% ### This Study

% The aim of this study is to provide an environment meant for client-side geoprocessing using WebAssembly. This environment will be used to demonstrate if and how wasm-based, client-side geoprocessing is possible. At the same time, by presenting this environment, the study aims to explore the design possibilities of a web-GIS application equipped with such tools. 

% This will require research into the technical effectiveness of WebAssembly. C++ geoprocessing libraries such as CGAL & GDAL will be tested on their ability to be compiled, loaded, and used from a browser. This is compared against their compilation by other means, such as native binaries or the aforementioned `asm.js`. This research is complemented by an extensive 'case study' to explore the design possibilities of a web-application equipped with client-side geoprocessing. A thick-client web application will be created, and this will serve as platform for testing the aforementioned `wagl`'s. The tool and can be additionally used for acquiring, visualizing, and saving geodata. 
