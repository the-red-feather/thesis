%-------------------------------------------------------------------------------------------------%
% A related work section in which the relevant literature is presented and 
% linked to the project. 
% It should show that you clearly know the problem you plan to solve, 
% and that you master the related work. 
\newpage

\section{Related work}

This chapter offers background literature on the aforementioned topics, and will explain how this literature relates to the proposed study. three related topics are regarded: prior studies on WebAssembly, Prior studies on client-side geoprocessing, and prior studies on geoprocessing interfaces.


\subsection{WebAssembly}

On 5 December 2019, WebAssembly officially becomes the fourth language of the web 

\cite{haas_bringing_2017}.

\cite{w3c_world_2019}.


the World Wide Web Consortium (W3C) 


WebAssembly is an emergent 

I found a couple of interesting papers on WebAssembly, including the original wasm paper



\subsubsection*{Existing Geo Information applications utilizing WebAssembly}

\begin{itemize}
  \item Google earth
\end{itemize}



\subsection{On client-side geoprocessing}





\subsubsection*{On geo web processing services}
These are client-side interfaces for server side processes. 
Due to the massive nature of geodata, this is understandable.







\subsection{On geoprocessing interfaces}



To the best of the author's knowledge, no papers exist coupling VPL to geoprocessing.

Still, this is being done, evident by...



- Ravi?

Lots of research has been done on the topic of VPL's, and their advantages and disadvantages. 
(I explicitly want to name the cognitive dimentions paper, it is very good and appropriate, and contains many suggestions for future VPL's)



\subsection{Conclusion}

- Time is important 
- 'missing link'


