%-------------------------------------------------------------------------------------------------%
% Overview of the methodology to be used.
% \newpage

\section{Methodology}

The methodology, in board terms, involves Four major steps. 

The first step is the development of the aforementioned web based visual programming language. 

At the end of this step, This environment will be usable 

basic mathematical and geometry procedures. 

and will not contain any geodata processing capabilities, nor WebAssembly. 

This development will be done incrementally, where in between steps 



The second step is determining how CGAL \& GDAL can be compiled



Third step is joining step 1 and 2



Fourth step is using the entire environment to create demo's





% This research attempts to improve the Accessibility and Interoperability issues of VPL's by % developing a prototype VPL. 
% This prototype will be used as the stage to develop the tools and knowledge necessary to % utilize the Web \& WebAssembly for geodata processing.   
% 
% \par
% A logbook will be maintained to explain how this prototype was constructed, what design % choices were made and why, and exactly in which way the FAIR principles influenced its % design. 
% 
% \par
% When the VPL contains all tools necessary to be used to properly process geodata, a final % assessment is needed. 
% Three different applications will be created using regular means (jupyter notebook, python, % panda's, etc)
% and these same applications will be created using the prototype web-VPL. 
% These two methods will then be compared on usability aspects and performance.   
