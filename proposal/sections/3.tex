%-------------------------------------------------------------------------------------------------%
% [HUGO]: The research questions are clearly defined, along with the scope (ie what you will not be doing).
% To help you define a "good" research question, 
% read \url{https://sites.duke.edu/urgws/files/2014/02/Research-Questions_WS-handout.pdf}.
% [ From Hugo's handout: ]
% 
% clear | focussed | unique | 
% start asking open-ended “How?” “What?” and Why?” questions. 
% Then evaluate possible responses to those questions
% While a good research question allows the writer to take an arguable position, 
% it DOES NOT leave room for ambiguity.
% 
% 1)Is the research question something I/others care about? Is it arguable?
% 2)Is the research question a new spin on an old idea, or does it solve a problem?
% 3)Is it too broad or too narrow?
% 4)Is the research question researchable within the given time frame and location?
% 5)What information is needed?
% 
% are you trying to accomplish one of these goals? 
% 1) Define or measure a specific fact or gather facts about a specific phenomenon. -> NO
% 2) Match facts and theory. -> NO
% 3) Evaluate and compare two theories, models, or hypotheses. -> SORT OF
% 4) Prove that a certain method is more effective than other methods. -> YES
% Moreover, the research question should address what the variables of the experiment are, their relationship, 
% and state something about the testing ofthose relationships. 
% 
% [JF] : I really know what I want to do, and I am convinced improving the usability of geoprocessing tools 
% is both valuable and Good. 
% 
\newpage
\section{Research questions}

\subsection{problem}
Client side geoprocessing is underdeveloped \& vpl's are not FAIR 

\subsection{testable hypotheses}
\textbf{By utilizing WebAssembly, C++ geoprocessing libraries such as CGAL \& 3dfier will run within a browser without needing to be installed.}
\begin{itemize}
    \item right, full functionality
    \item right, but with limitations 
    \item wrong, but could be done
    \item wrong, Impossible 
\end{itemize}

\textbf{These WebAssembly compiled libraries will achieve similar performance as native, cli usage of the same libraries.}
\begin{itemize}
    \item Right, it is slower, but the difference is negligible 
    \item Half Right, Users can notice it is a bit slower.
    \item wrong, it is significantly slower, borderline unusable  
\end{itemize}

\textbf{A web-based geo-vpl equipped with these libraries will be more \emph{Usable} for the given case study than a conventional method using Python}

\begin{itemize}
    \item Right, the web-geo-vpl scores better on all FAIR usability criteria.
    \item Half right, it scores better on some of the FAIR criteria.
    \item wrong, this way of working scores worse on all FAIR criteria.
\end{itemize}



\subsection{report results}
??? dont know how to interprete this 



%-------------------------------------------------------------------------------------------------%
\subsection{Research Questions}





% ----
\subsection{Scope}


\subsubsection*{Will Include}

- design decisions informed by (Usability analysis...)
- 


\subsubsection*{Will Not Include}
This research will exclude web processing services, since these services are not about \emph{client} side geoprocessing, but instead cover \emph{server} side geoprocessing. 
Future work could, however, research the possibility of utilizing a vpl for server orchestration. 


