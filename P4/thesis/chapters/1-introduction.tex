%-------------------------------------------------------------------------------------------------%
% An introduction in which the relevance of the project and its place in the 
% context of geomatics is described, along with a clearly-defined problem statement.
\chapter{Introduction}

\section{Introduction}


% [JF]: I THINK THIS CAN BE MORE ELEGANT, MORE SPECIFIC, MORE QUICKLY, MORE TO THE POINT 
% Dissolving the discrepancy between visualization \& processing in web-apps.
Interactive, browser-based GISS form an indispensable component of the modern geospatial software landscape. For the common person, a web application is often their first and only exposure to a GIS, be it a web mapping service, a navigation system, or a pandemic outbreak dashboard. A web application is cross-platform by nature, and offers ease of maintainability and access, since no installment or app-store interaction is required to run or update the app. The ability to be shared with a link, or to be embedded within the larger context of a webpage is also not trivial. These aspects have made the browser a popular host for many geographical applications, especially when targeting end-users. 

% A TREND 


\section*{Motivation For Geofront}

\subsection{Motivation 1: Client-side Geoprocessing}

Despite the popularity of geographical web applications, the range of actual \ac{gis} abilities these applications are capable of is very limited. \ac{geoprocessing} abilities, like CRS translations, interpolation or boolean operators, are usually not present within the same software environment as the web app. Consequently, current geospatial web applications serve for the most part as not much more than viewers; visualizers of pre-processed data. 

This limited range of capabilities inhibits the number of users and use cases geographical web applications can serve, and with it the usefulness of web \ac{gis} as a whole. 

If web applications gain \ac{geoprocessing} capabilities, they could grow to be just as diverse and useful as desktop \ac{gis} applications, with the added benefits of being a web application. It would allow for a new range of highly accessible and sharable geoprocessing and analysis tools, which end-users could use to post-process and analyze geodata quickly, uniquely, and on demand.

This is why \ac{geoprocessing} within a web application, whereby mentionned as \ac{bbg}, is slowly gaining traction during the last decade \cite{kulawiak_analysis_2019, panidi_hybrid_2015, hamilton_client-side_2014}. Interactive geospatial data manipulation and online geospatial data processing techniques are described as "current highly valuable trends in evolution of the Web mapping and Web GIS" \cite{panidi_hybrid_2015}. But this also raises the question: \textit{Why is geoprocessing within a web application as of today still nowhere to be found?} 

se concerns represent the three main obstacles preventing a smooth, widespread adoption of \ac{bbg}. 

The study proposed by this paper seeks the advancement of web \ac{gis} \& client-side geoprocessing by attempting to overcome these obstacles. It will do this by researching possible solutions to key components of all three of them. However, we must first regard each obstacle more closely, so that the significance of these key components can be made clear. 

\subsection{Motivation 2: Accessible Geoprocessing Libraries}

Most industry-standard geoprocessing libraries such as CGAL are difficult to use by anyone but experts in the field. A steep learning curve combined with relatively complex installation procedures hinders quick experimentation, demonstration, and widespread utilization of these powerful tools. It also limits the interdisciplinary exchange of knowledge, and compromises the return of investment the general public may expect of publicly funded research.

Geofront could improve the accessibility of existing geodata processing and analysis libraries, without adding major changes to those tools, by loading webassembly-compiled versions of them, similar to [other web demo's](todo).

\subsection{Motivation 3: Visual Programming}

The choice for a Visual Programming Language(vpl) is made to further explore this idea of accessible geoprocessing. 

demonstrate the advantage of making a geoprocessing tool web based, and thus potentially accessible to a larger audience. 
Using visual programming, the geoprocessing sequence can be altered on the fly, and in-between products can be inspected quickly, as both data and in a 3D viewer. 
This way, a user can easily experiment with different methodologies and parameters which, hypothetically, improves the quality of the processed geodata.
Additionally, a vpl forms a balance between a programming language and a full gui, making the tool accessible to both programmers and non-programmers alike.

\section{This Study}


\section{Use Case}

% what 
The use case application for this study is the so called "GeoFront" environment.
% , nicknamed as a concatenation of 'geoprocessing' and 'frontend'. 
GeoFront will be developed as a web-based \ac{gis} environment, offering users the ability to load, process, and then visualize or save various types of geodata. 
The entire application will run client-side in a browser, and uses a visual programming language as its primary \ac{gui}.
For input, the environment offers \ac{wms} and \ac{wfs} support, as well as ways for users to load local geodata.
For processing, geofront offers geoprocessing functions provided by GDAL and CGAL. 
Being a visual programming language, GeoFront can be used to chain multiple processing steps together, while still being able to retrieve in-between products. 
For output, the environment can be used to either save data to the user's local machine, or to visualize the results within the geofront application using a WebGL based viewport.


%-------------------------------------------------------------------------------------------------%

% TODO this is good, but de-emph 'existing libraries', and emph 'visual programming language'

\newpage
\section{Research Questions}

This study intends to discover if contemporary web technologies can facilitate a full-scale client-side geoprocessing tool, by seeking an answer to the following question: 

\textit{How to \textbf{design and create} a browser-based GIS environment which can \textbf{effectively utilize} \textbf{existing geoprocessing libraries}, using only \textbf{contemporary}, \textbf{standard client-side web features}?}

\subsection*{Explanation}

% The research question is written purposefully written in the "how well does X support Y question" shape. To unpack its components: 

- \textbf{Design and create}: The wording 'design and create' is used to signal that this will consider the theoretical design , as well as the practicalities of creating this design. 

- \textbf{Standard client-side web features}: This phrase is meant to limit the scope to only the standard, core technologies of major browsers (Chrome, Edge, Safari, Firefox). This means the four languages \ac{wasm}, CSS, JavaScript and HTML. Additionally, HTML5 gives us features such as WebGl, the 2d Canvas API, SVG's, and Web Components to work with.

- \textbf{Contemporary}: The study will use contemporary, even bleeding edge features of the modern web, but its findings will nonetheless be bound to this time of writing, as web technologies in particular quickly change over time. 

- \textbf{Existing geoprocessing libraries}. This wording expresses this studies desire to explore the usage of existing geoprocessing libraries, rather than to recreate geoprocessing libraries from scratch.

- \textbf{Effectively utilize}: The study intends to not only find out how wagl's can be \textit{run} in a browser, but also how wagl's can be \textit{used}. 'Effective' is used in the wholistic sense, since a balance will have to be found between several aspects such as load-time, run-time, and less concrete usability aspects. 


\subsection*{Sub Questions}

The following sub-research questions are needed in order to answer the main question. The methodology chapter will explain the choices of these sub-questions. 


\textit{1 : What is the most fitting methodology of compiling C++ geoprocessing   libraries to Web-Assembly?}

\textit{2 : How to design and create a client-side geoprocessing environment for data-users?}

\textit{3 : How can wasm-compiled geoprocessing libraries be distributed and used in a client-side geoprocessing environment?}

\textit{4 : What are the advantages and disadvantages of GIS applications created using a client-side geoprocessing environment powered by WebAssembly?}

\newpage
\subsection*{Assessment}

At the final conclusion of the proposed thesis, The Thesis will answer if the designed and created GIS environment can indeed effectively utilize these geo-libraries.
This will be answered by quantitative and qualitative means:

Quality
\begin{itemize}
    \item Have all design goals been met?
    \item Can data users 'effectively' handle input, process and output?
    \item Can the load \& run times be regarded as acceptable to use? 
\end{itemize} 

Quantity
\begin{itemize}
    \item Have all required features been implemented?
    \item Which libraries can / could be used?
    \item What are the load \& run times of these libraries, compared to native execution?
\end{itemize} 

%%%%%%%%%%%%%%%%%%%%%%%%%%%%%%%%%%%%%%%%%%%%%%%%%%%%%%%%%%%%%%%%%%%%%%%%%%%%%%%
\newpage
\section{Scope}
\subsection*{Will Include}

The 'will include' scope is represented by the Methodology chapter. 


\subsection*{Will not include}

% TIM: maybe move this to obstacle 3 ? 
\subsubsection*{Server-side or Native WebAssembly} % **Client-side WebAssembly Only**

This study will limit itself to the \emph{client-side} usage of WebAssembly. 
A powerful case can be made for \emph{server-side}, or native level usage of WebAssembly, especially in conjunction with a programming language such as Rust. 
Rust compiled to WebAssembly could, compared to using python, java or C++, make geoprocessing more maintainable and reliable, while at the same time ensuring memory safety, security, and performance \cite{clack_standardizing_2019}. 

Server-side or native wasm is beyond the scope of this paper, but would be an excellent starting point for future work. Note that this also means that research into WebAssembly is important for more than just client-side geoprocessing. All geoprocessing could benefit from it.



\subsubsection*{Web Processing Service} % Will not be dealing with WPS 

% offered as server-side geoprocessing services.  
Similarly, this study will exclude the OGC standard of the \ac{wps} \cite{ogc_web_2015}, since these services do not offer \emph{client} side geoprocessing, but instead offer \emph{server} side geoprocessing. A client-side application \textit{could} create an interface to use such a service, to essentially offer geoprocessing to clients, but this study regards a solution like that as a workaround, not a true solution to the problem of client-side geoprocessing. 

This is not to say that client-side geoprocessing replaces the need for \ac{wps}. 
future work could research the possibility of utilizing a hybrid strategy of both client-side and server-side geoprocessing, following in the footsteps of \cite{panidi_hybrid_2015}. 



\subsubsection*{Usability Analysis} % 

While usability is a motivation of this research, no claims will be made that the developed use-case is more usable to native GIS applications or geoprocessing methods. This research attempts to solve practical inhibitions in order to discover whether or not client-side is \emph{a} usable option. If it turns out that this method is viable technically, future research will be needed to definitively proof \emph{how} usable it is compared to all other existing methods.  

% This paper seeks to first close this gap, limiting itself to overcoming the technical and design boundaries in the pursuit of practical client-side geoprocessing.

Similarly, a survey analyzing how users experience client-side geoprocessing in comparison to native geoprocessing must also be left to subsequent research. While this would gain us a tremendous amount of insight, client-side geoprocessing is too new to make a balanced comparison. Native environments like GRASSGIS, QGIS, FME or ArcGIS simply have a twenty year lead in research and development. 
