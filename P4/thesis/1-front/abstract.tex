
\chapter*{ (Long) Abstract }

This thesis introduces GeoFront, a web based, open source, geodata processing and visualization environment. 
In this thesis, the motivation, requirements, technical aspects, and achieved functionality of geofront is covered. 

% MOTIVATION & FUNCTIONALITY
GeoFront has been created in an effort to make low-level geoprocessing libraries more accessible and interactible.
\emph{Accessibility} is achieved by compiling the geoprocessing libraries to webassembly, after which the tools can be run in a browser, client-side. 
This mitigates the need for installment; 
% As soon as the tool is found on the web, it can be used. 
\emph{Interactivity} is achieved by making geofront a visual programming environment. 
Functions are represented by blocks, and these can be chained together to form an interactively configurable geoprocessing pipeline. 
In between products can be inspected and debugged using a 3D viewer. 
% Special blocks such as sliders can be used to quickly try different values and parameters, and to 'sweetspot' certain settings.
% INPUTS: OGC STANDARDS & FILE SYSTEM | OUTPUTS: DOWNLOAD

% RESULT
This study has successfully made a limited set of CGAL operations usable like this, as well as various rust-based geoprocessing algorithms such as Startin.
The performance is ..., \todo{"Do performance comparrisons"}

% USE CASE
GeoFront has multiple actual and possible use cases. 

1. Tryout (ACTUAL)
   - A-la wapm WebAssembly Package Manager allows packages to be run from within the package-page itself. 
  - Just meant to quickly try out some features.

2. Educational (ACTUAL)
   - interactive educational tool
   - (What does a delaunay triangulation look like? how does it behave? What happens if you lower the radius of inverse distance weighting ? )

3. Rapid-Prototyping (POSSIBLE)
   - Web geoflow
   - Future work: export flowchart to a process which can be run natively or server side.

4. Publishing (POSSIBLE)
   - Geotiff.io
   - Web FME 
   - Publish full web apps in and off themselves, making use of zero, one or multiple wasm-compiled libraries.  
   - Future work: export to web-app (without flowchart)

% FEATURES

% CONTENT 

% CONCLUSION 



% Safesoft's FME, but web based \& open source 

% CONCLUSION
By creating geofront, this thesis was able to discover .............

- The web is able to facilitate a visual programming language.

- The web is able to be used for geoprocessing, albeit with some caveats
  - TypedArrays,
  - Geometric predicates 
  - Rounding
  - ETC.

- Many of these things can be fixed with webassembly, but webassembly itself has other shortcomings
  - Differences between Rust \& C++

- reasonable performance 
  (- great considering the platform)

- would not be possible without these modern web features
  - Web Assembly 
  - Typed Array's 
  - Web Workers
  - Web Components,
  - 2D Canvas API
  - Web GL

\todo{[JF]: I need to add more critical notes on promises of accessibility and interactivity. Is a webapp truly accessible? Is a flowchart interactive, or does it hinder interactiveness? }

We believe that such a web application can make geoprocessing more accessible to practitioners.

%%%%%%%%%%%%%%%%%%%%%%%%%%%%%%%%%%%%%%%%%%%%%%%%%%%%%%%%%%%%%%%%%%%%%%


% We introduce the \"Geofront\" application, which achieves this accessibility in two ways: 
% Firstly, geoprocessing libraries, written in either Rust or C++, are compiled to WebAssembly: binaries which can be run in any modern browser, client-side. This mitigates the need for any installation, and allows the software to be directly used as soon as a website is fully loaded. WebAssembly offers a performance comparable to native usage. Geofront accepts these binaries as plugins.

% Secondly, Geofront itself is set up as a web-based visual programming environment, complete with a 3D viewer and WMS, WFS \& WMTS support. Using these tools, users can interactively run these geoprocessing libraries with different datasets and parameters. Using visual programming, the user can chain and alter geoprocessing steps, visualize in-between products, and save \& load these workflows.



% While the research restrains itself from empirically measuring an aspect as nebulous as 'accessibility', it does demonstrate ...

% This empowers users to create small geoprocessing demo's, and share these 

% With geofront, geoprocessing libraries can be loaded and used interactively. Users are also able to create and share flowcharts.


%%%%%





% In one sentence: It is my goal to make your grandma use CGAL.

% This requires making something extremely user-friendly compared to QGIS / python (the vpl)
% As well as making it extremely accessible (wasm: no install, direct usage)

