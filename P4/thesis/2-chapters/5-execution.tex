\chapter{Execution}%%%%%%%%%%%%%%%%%%%%%%%%%%%%%%%%%%%%%%%%%%%%%% CHAPTER

\section{Environment}%%%%%%%%%%% SECTION
- github organization 
- repo for engine 
- repo for app 
- repo for each plugin
- build procedure

\section{Usage}%%%%%%%%%%% SECTION

\emph{basically, write a tutorial}

This is how one can use Geofront

\section{Creating \& Using your own code}
\emph{show the insane (rust + wasm + npm) workflow}

Locally: 
1. Write a geoprocessing / analysis library using a system-level language (rust, C++).
2. Expose certain functions as public, using 'wasm-pack' or 'emscripten'.
3. Compile to `.wasm` + `d.ts` + `.js`.
4. publish to npm (very easy to do with wasm-pack, can also be done with emscripten)

Alternatively: 
1. Write a library using typescript, 
3. Compile to `d.ts` + `.js`.
4. publish to npm 

In Geofront: 
4. Reference the CDN (content delivery network) address of this node package. 

Congrats, this is now a publicly accessible geofront plugin!
The library is loaded, A component is created for each function, with inputs for input parameters, and a singular output. If a 'typescript tuple type' is exported, the plugin loader will create multiple outputs according to each component of the tuple.



\section{Performance}%%%%%%%%%%% SECTION

Benchmark time!

\section{Limitations}%%%%%%%%%%% SECTION

Limitations of usage right now
