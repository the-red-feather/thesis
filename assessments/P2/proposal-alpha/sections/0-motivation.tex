\section{Motivation}

Besides the problems stated during the introduction, this study carries my personal motivation of improving the usability of geo processing tooling. 

Why Visual Programming language?



% ## 4.1 'higher level' questions. 

% The research questions chosen for this research are part of a set of larger questions. While the research will not completely answer the following questions, I believe the questions are nonetheless important to address.

% > What should the field of geomatics do with WebAssembly?
% > - Why should the field of geomatics be interested? 
% >   <!-- A: Yes  -->
% > - Can we technically use it for geomatics? 
% >   <!-- A: Probably  -->
% > - Can we practically use if for geomatics? 
% >   <!-- A: Unsure -->

% This also further explains the need for the vpl application within this research. I believe it necessary to develop an application whom's existence serves as a starting point for answering the more complicated "why should we", and "practical" sub-questions.

% <br><br><br>

% ## 4.2 Additional problems the software tries to solve, and features it tries to present:

% additionally, 

% ### - Real-time geodata processing

% - A number of use-cases exist with a growing need for real-time geodata processing. (SOURCE: INCIDENT MAPPER)

% - Moving tools like CGAL closer to the final product (Web Application) can create more dynamic applications. 

% ### - Improved Geoprocessing Ergonomics

% - Insightful debugging: Client-side geoprocessing together with a VPL allows direct user feedback unlike server-side geoprocessing. Users can be on top of the calculations, look at in betweens steps, reconfigure the procedure without recompilation, see the immediate effects of parameter changes. 

% - Improved communications: Users will be able to share demo's and procedures with a link.

% - Improved accessibility: Users will not have to install anything except a web-browser.
%   This will make geoprocessing more accessible & operational to a larger audience. It allows more people to do more things with geodata, and reach more interesting conclusions quicker. 

% ### - Just In Time / Personal Geodata

% - JIT: Instead of having large, preprocessed datasets, geodata could be processed on demand from the source client-side. If a user is only interested in a small area of the source dataset, this could save vast amounts of time, storage space and computational resources. 

% - Personal: It also allows the end user to tailor geodata to their exact needs. 




