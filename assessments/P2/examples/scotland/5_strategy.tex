\section{Strategy towards an open SDI}
The third element of the research question goes "what steps can be taken for achieving an open SDI?". Per aforementioned barrier, we suggest a strategy towards mitigating this barrier. We base these suggestions on the Government Instrument paradigm \citep{gov_instruments} \& Kindon Theory \citep{kingdon}, alongside 'best practice' concepts and examples, as well as on our on conclusions based upon the assessment of the SSDI.

These strategy proposals are aimed at the recently appointed Scottish Geospatial Network Integrator (SGNI). This entity is a surrogate by the Scottish Enterprise, the Scottish Government, and the UK Geospatial Commission, which has been given the purpose of "enhancing the use of location-based data across Scotland". \citep[]{scottish_geospatial_network_integrator}. Components of this mission include: "Helping to build a dedicated network of geospatial organisations across Scotland", and "developing the supply chain and signpost to potential growth opportunities". While these statements seem more focused to economic stimulation, and networking, we do believe that the SGNI is the best actor to carry out the vision of the Open SDI. This is because the entity is already an integral component connected to three different actors within the geospatial field (enterprise, geospatial commission, government), in addition to being its own separate actor. This gives it a broad and complete overview.


\subsection{Instrument Utilization}

% 1. Collective Decision making
% 2. Strategic management 
% 3. Allocation of tasks and responsibilities 
% 4. Creation of markets
% 5. Inter organizational culture and knowledge management
% 6. Regulating and formalizing the infrastructure

% GOVENRMENT INSTRUMENTS ARE BETTER REFERENCED TO Crompvoets et al. (2018)

First, to correctly implement these strategies, the SGNI should maximize its capabilities. Acting as a surrogate of, among others, the Scottish Government, the SGNI should have several governmental instruments at its disposal. We distinguish six different governance instruments, as mentioned by \citep{gov_instruments}. This framework can be used to identify underutilized abilities of the SGNI, and thus the Scottish Government. Utilizing all six of these instruments ensures the full potential of the SGNI. These six instruments are: 1. Collective Decision making, 2. Strategic management, 3. Allocation of tasks and responsibilities, 4. Creation of markets, 5. Inter organizational culture and knowledge management, 6. Regulating and formalizing the infrastructure.

When comparing the mission statements of the SGNI with the instruments, it seems that only instrument four and five are utilized. The SGNI focuses on market stimulation and raising awareness of the possibilities of geospatial data. We suggest the SGNI to especially improve focus on instrument one and two: collective decision making, and strategic management. These are not only underutilized at the moment, but also vital requirement for the creation of an Open SDI. This is why in subsection 'strategy on key stakeholders', we will focus primarily on suggestions on these two aspects. 

Finally, when comparing the current strategies of the SGNI with the 'Carrot, Preach \& Stick' paradigm in \citep{gov_instruments}, we also see a unbalanced reliance on the 'preach' method. Proper utilization of 'carrot \& sticks', or in more proper terms, rewards \& punishment, will make the SGNI better at completing its mission. This might require additional action from the Scottish Government themselves to grant SGNI this authority. Also, we suggest that due to the sensitive nature of the roll-out of an Open SDI, i.e. the reluctance to adaptation by the Local Government, using reward is in this case preferable over punishment. 


\subsection{Strategy On Politics}

Discussing or analyzing solutions to the issues caused by Brexit would be far beyond the scope of this paper, as the matter at hand would require a deep analysis into the socio-political underpinnings of the United Kingdom. However, we do want to encourage that the SSDI remains in accordance to the Cross-European directives as much as possible. This because the issues that partially led to the creation of for example the INSPIRE directive, do not remain at the borders of counties \citep{inspire_background}. Brexit does not change this fact. Luckily, the Cross-European directives are still in place in the UK and Scotland, but this should remain so.


\subsection{Strategy on User Participation}

The second barrier, the barrier towards more user participation, will also not be mitigated easily. Nevertheless, we offer a strategy towards a policy enabling more user participation within the SSDI. This will be based upon the Windows of Opportunity model made by John Kingdon \citep{kingdon}.

According to this policy window model, three ‘streams’ must be aligned before a policy can be taken into action by a governmental body. These three streams are the problem stream (is the problem sufficiently recognised by the public as a problem?), the policy stream (do alternate policies exist?), and the political stream (are there politicians willing and able to make these changes?). With these three streams aligned, the window of opportunity opens, and action can be taken. 

Concerning a policy towards more user participation in the SSDI, the second stream is in place. The SDI theory mentioned within this paper offers multiple examples of SDI models which improve user participation, and with it the quality of the SDI as a whole. The third steam is also aligned. From the research done to create figure \ref{fig:actor-diagram}, We know that the Open Government Network (OGN) is working together with the Scottish Government, suggesting adequate political support by Scotland in favour of open data policies.

It is the first stream that actually misaligned, and preventing the mitigation of this barrier. From the stakeholder research, we know that only the OGN recognises low levels of user participation troubles open data policy making. Most other stakeholders see more open user participation as a compromise to data quality on the SDI \citep{IS_local_SSDI}. 

Thus, while we cannot offer a strategy or solution to this discrepancy, we can conclude that as soon as the divisions among stakeholders are overcome, the window of opportunity towards a new user participation policy opens up with it. This is why we regard the third barrier, the division among key stakeholders, as the most important one. 


\subsection{Strategy on Key Stakeholders}

The third and most major barrier towards more openness is the division among key stakeholders (Figure \ref{fig:actor-diagram}). This is confirmed by both the Open Data Institute \citep{odi_geodatainfrastructure} and the Scottish Geospatial Network Integrator(SGNI) \citep[SGNI]{scottish_geospatial_network_integrator}.

As mentioned before, the government instruments 'Collective Decision Making' and 'Strategic Management' are  underutilized by the SGNI. This is why this strategy is of a collaborative nature. Based on the visions and statements published by the stakeholders themselves, we suggest two major collaborations: one between the SGNI and the OGN, and one between the local and national government of Scotland.


\subsubsection{SGNI and OGN}

We believe that the SGNI and Open Government Network (OGN) have mutual interests. Interests that, if joined, add up to the vision of a open SSDI. This requires prior action by both parties. 

The Scottish Geospatial Network Integrator needs the SSDI itself on the agenda, and with it the open SDI. It has defined itself as the leader in its quest integrate and stimulate cooperation among the entire geospatial sector \citep[SGNI]{scottish_geospatial_network_integrator}, which means that the SSDI cannot be ignored. Additionally, while the vision states 'increase the \textbf{use} of geospatial data', the vision does not state anything about the \textbf{re-use} of geodata. Re-use is a hallmark of an SDI, and should also be included in the vision. 

At the same time, The Open Government Network needs to include geospatial data within their action plans. Right now, the mention of 'government data' is too big of a concept for the purpose of an Open SDI. Geospatial data is clearly different than the fiscal reports \& legislative documents the the 'data' definition of the OGN seems to be focused on. 

With their internal visions adjusted, the SGNI and OGN can start a dialogue. The SGNI seeks an Open SDI (even if they don't know it yet), because Open Data has been proven to stimulate the economy \citep{eucommission_psiToOpen}. This aligns with their vision to "unlock data and boost the economy". The OGN On the other hand, clearly upholds the values of citizen participation, open government, and citizen empowerment, which form some of the foundation of an Open SDI. \citep{opengovpartnership_mainpage} \citep{towards_user_oriented_open_data_strateg_2018}. This allegiance to meet collective goals will make the Open SDI an explicit action plan of both parties, instead of the implicit, fragmented vision it is right now.   


\subsubsection{Local Government and National Government of Scotland}

% Local Government \& Government of Scotland
A second discrepancy towards a unified vision for an open SDI is a disconnect between local governments \& the national government in Scotland. Scotland has set out to meet the European PSI directive, but this will have major consequences for the revenue of local governments. the Improvement Service (IS), a company acting as  consultant to local government on subjects such as IT, has already expressed a number of objections to the implementation of Open SDI, and with it parts of the PSI directive \citep{IS_local_SSDI}. They dislike the consequences to revenue, the consequences to local jobs, the consequences to data quality, and the fact that they themselves stand little to gain from an open SDI. 

It is our opinion that these concerns of the local governments are valid and need to be taken into account. To meet these concerns, the Scottish Government needs to state clearly how they will compensate the loss in revenue, state how the local jobs will not be lost when an open SDI is implemented, and they will have to show the IS and local government the benefits of open SDI, even on local level. We believe it is fair to make the Scottish Government accountable for these concerns, for it was their decision to uphold the PSI directive. Lastly, on the issue of data quality, we propose something akin to a license to edit geospatial data, just like anyone needs to be licensed before doing anything that could endanger others. 

With these concessions, we believe the PSI directive can be implemented on a local level, further facilitating the development of an open SDI. 


\subsection{Implementation}

We would like to comment on the implementation of these strategies. A common paradigm within the field of management is that an idea means little without an implementation. we offer two implementation suggestions based on our own analysis of the SSDI.

Firstly, Keeping track of progress is vital for the implementation of strategies on a governmental level \citep{opengovpartnership_mainpage}. Governments can't judge what they can't measure. The Open Government Partnership are a good example of this. For every action plan \& strategy they state, they immediately define clear goals to meet, and define the exact way progress can be measured. 

Secondly, we want to emphasise the importance of small, incremental steps, in regard to the implementation progress of the SDI. the ODI notices a great level of variance among geospatial data providers, in terms of how far the provider had met INSPIRE requirements \citep{odi_geodatainfrastructure}. They note that those who fall behind, only fall further behind. This is why we advocate for slow, gradual change, with a focus on the overdue changes, instead of some parties excelling forward while the rest stays behind. 




